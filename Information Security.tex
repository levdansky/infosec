\documentclass[10pt,a4paper]{book}
%\documentclass[12pt,report,russian]{ncc}
%\usepackage{a4wide}
% Для векторых русских шрифтов в PDF не забудьте установить пакеты cm-super & cm-unicode
\usepackage{cmap}                       % Поддержка поиска русских слов в PDF (pdflatex)
\usepackage[X2, T2A]{fontenc}
%\usepackage[T2, OT1]{fontenc}
\usepackage[utf8]{inputenc}
\usepackage[english,german,italian,latin,russian]{babel}
\usepackage{indentfirst}                % Красная строка в первом абзаце
%\usepackage{misccorr}
%Может быть установлено 8pt, 9pt, 10pt, 11pt, 12pt, 14pt, 17pt, and 20pt
%\usepackage[12pt]{extsizes}
%\usepackage[mag=1000,a4paper,left=3cm,right=2cm,top=2cm,bottom=2cm,noheadfoot]{geometry}

\usepackage{amsmath} % разрешить \texttt и аналогичные в формулах
\usepackage{amssymb, } % дополнительные математические символы
\usepackage{graphicx} % поддержка изображений

%\usepackage{amsfonts, eucal, bm, color, }

\usepackage{algorithm, algorithmic}     % 'algorithm' environments
\floatname{algorithm}{Алгоритм}
\usepackage{multirow}                   % multirow cells in tables
\usepackage{arydshln}                   % dash lines in tables
\usepackage{subfig, float, wrapfig}     % sub figures
\usepackage{caption}                    % titles for figures
\usepackage{tikz}                       % векторная графика внутри TeX
\usepackage{makeidx}                   % index
%\usepackage[xindy]{imakeidx}
\usepackage[totoc=true]{idxlayout}      % балансировка индексов на последней странице, индекс в ToC
\usepackage{enumerate}
\usepackage{fancybox}                   % страница в рамке
%\usepackage{fancyhdr}                  % глава и секция вверху страницы
%\usepackage{layout}
\usepackage[left=1.84cm, right=1.5cm, paperwidth=14cm, top=1.8cm, bottom=2cm, height=19.8cm, paperheight=20cm]{geometry}
\usepackage[parentracker=true,
  backend=biber,
  hyperref=auto,
  language=auto,
  citestyle=gost-numeric,
  defernumbers=true,
  bibstyle=gost-numeric,
  sortlocale=ru_RU
]{biblatex}								% библиография по ГОСТу
\addbibresource{bibliography.bib}

% поддержка гиперссылок; гиперссылки в pdf, должен быть последним загруженным пакетом
\ifx\pdfoutput\undefined
    \usepackage[unicode,dvips]{hyperref}
\else
    \usepackage[pdftex,colorlinks,unicode,bookmarks]{hyperref}
\fi

%\paperwidth=16.8cm \oddsidemargin=0cm \evensidemargin=0cm \hoffset=-0.33cm \textwidth=13.2cm
%\paperheight=24cm \voffset=-0.4cm \topmargin=0cm \headsep=0cm \headheight=0cm \textheight=19.8cm \footskip=0.9cm

% параметры PDF файла
\hypersetup{
    pdftitle={Защита информации},
    pdfauthor={Э. М. Габидулин, А. С. Кшевецкий, А. И. Колыбельников, С. М. Владимиров},
    pdfsubject=учебное пособие,
    pdfkeywords={защита информации, криптография, МФТИ}
}

% добавить точку после номера секции, раздела и т.~д.
\makeatletter
\def\@seccntformat#1{\csname the#1\endcsname.\quad}
\def\numberline#1{\hb@xt@\@tempdima{#1\if&#1&\else.\fi\hfil}}
\makeatother

% перенос слов с тире
%\lccode`\-=`\-
%\defaulthyphenchar=127

% изменить подписи к рисункам, таблицам и т.~д.
\captionsetup{labelsep=period}          % заменить : на .
\captionsetup{textformat=period}        % Подписи завершать точкой
%\captionsetup[table]{position=above}    % вертикальные отступы подписи таблицы для случая, когда подпись вверху
%\captionsetup[figure]{position=below}   % вертикальные отступы подписи рисунка для случая, когда подпись внизу

%% стиль главы и секции вверху страницы
%\pagestyle{fancy}
%%\renewcommand{\chaptermark}[1]{\markboth{#1}{}}
%\renewcommand{\sectionmark}[1]{\markright{#1}{}}
%
%%\fancyhf{}
%%\fancyfoot[СE,CO]{\thepage}
%%\fancyhead[LE]{\textsc{\nouppercase{\leftmark}}}
%\fancyhead[RO]{\textsc{\nouppercase{\rightmark}}}
%
%\fancypagestyle{plain}{ %
%\fancyhf{}                              % remove everything
%\renewcommand{\headrulewidth}{0pt}      % remove lines as well
%\renewcommand{\footrulewidth}{0pt}}

% запретить выходить за границы страницы
\sloppy

\newtheorem{theorem}{Теорема}[section]
\newtheorem{lemma}[theorem]{Лемма}
\newtheorem{definition}[theorem]{Определение}
\newtheorem{property}[theorem]{Утверждение}
\newtheorem{corollary}[theorem]{Следствие}
%\newtheorem{algorithm}[theorem]{Алгоритм}
\newtheorem{remark}[theorem]{Замечание}
\newcommand{\proof}{\noindent\textsc{Доказательство.\ }}

%\newtheorem{example}{\textsc{\textbf{Пример}}}
\newcommand{\example}{\textsc{\textbf{Пример.}} }
\newcommand{\exampleend}

\newcommand{\set}[1]{\mathbb{#1}}
\newcommand{\group}[1]{\mathbb{#1}}
\newcommand{\E}{\group{E}}
\newcommand{\F}{\group{F}}
\newcommand{\GF}[1]{\group{GF}(#1)}
\newcommand{\Gr}{\group{G}}
\newcommand{\R}{\group{R}}
\newcommand{\Z}{\group{Z}}
\newcommand{\MAC}{\textrm{MAC}}
\newcommand{\HMAC}{\textrm{HMAC}}
\newcommand{\PK}{\textrm{PK}}
\newcommand{\SK}{\textrm{SK}}

\newcommand{\langde}[1]{нем. \foreignlanguage{german}{\textit{#1}}}
\newcommand{\langen}[1]{англ. \foreignlanguage{english}{\textit{#1}}}
\newcommand{\langit}[1]{итал. \foreignlanguage{italian}{\textit{#1}}}
\newcommand{\langlat}[1]{лат. \foreignlanguage{latin}{\textit{#1}}}

% Для раздела с задачами
\newcommand{\taskinit}{\newcounter{task-section}\setcounter{task-section}{0}\newcounter{task-number}}
\newcommand{\tasksection}{\addtocounter{task-section}{1}\setcounter{task-number}{0}}
\newcommand{\tasknumber}{\textbf{№ \addtocounter{task-number}{1}\arabic{task-section}.\arabic{task-number}.}~~}

%Наконец, существует способ дублировать знаки операций, который мы приведём безо всяких пояснений. Включив
%\newcommand*{\hm}[1]{#1\nobreak\discretionary{}{\hbox{\mathsurround=0pt #1}}{}}
%в преамбулу, можно написать $a\hm+b\hm+c\hm+d$, при этом в формуле a\hm+b\hm+c\hm+d при переносе знак + будет продублирован.

% Дублирование символов бинарных операций ("+", "-", "="), набранных в строчных формулах, при переносе на другую строку:
%%begin{latexonly}
%\renewcommand\ne{\mathchar"3236\mathchar"303D\nobreak
%      \discretionary{}{\usefont
%      {OMS}{cmsy}{m}{n}\char"36\usefont
%      {OT1}{cmr}{m}{n}\char"3D}{}}
%\begingroup
%\catcode`\+\active\gdef+{\mathchar8235\nobreak\discretionary{}%
% {\usefont{OT1}{cmr}{m}{n}\char43}{}}
%\catcode`\-\active\gdef-{\mathchar8704\nobreak\discretionary{}%
% {\usefont{OMS}{cmsy}{m}{n}\char0}{}}
%\catcode`\=\active\gdef={\mathchar12349\nobreak\discretionary{}%
% {\usefont{OT1}{cmr}{m}{n}\char61}{}}
%\endgroup
%\def\cdot{\mathchar8705\nobreak\discretionary{}%
% {\usefont{OMS}{cmsy}{m}{n}\char1}{}}
%\def\times{\mathchar8706\nobreak\discretionary{}%
% {\usefont{OMS}{cmsy}{m}{n}\char2}{}}
%\mathcode`\==32768
%\mathcode`\+=32768
%\mathcode`\-=32768
%%end{latexonly}

\makeindex

\begin{document}
\selectlanguage{russian}

%\layout

% рамка границ страницы http://www.ctan.org/tex-archive/macros/latex/contrib/fancybox/fancybox-doc.pdf
% сделать поиск по fancypage, thisfancypage
%\thisfancypage{}{\fbox}
%\thisfancypage{\fbox}{}
%\fancypage{}{\fbox}         % закомментировать
%\fancypage{\fbox}{\fbox}    % закомментировать
%\fancypage{\setlength{\fboxsep}{32pt}\fbox}{}

\title{Защита информации \\ Учебное пособие}
\author{Габидулин Эрнст Мухамедович \\ Кшевецкий Александр Сергеевич \\ Колыбельников Александр Иванович \\ Владимиров Сергей Михайлович}
\date{
 %   \textbf{\textsc{Черновой вариант. Может содержать ошибки.}} \\
%    \today
}
\maketitle
\setcounter{page}{3}

\newpage
%\thispagestyle{empty}
\setcounter{tocdepth}{2}
\tableofcontents
%\thispagestyle{empty}
\newpage

%\lhead[\leftmark]{}
%\rhead[]{\rightmark}

\let\oldleftmark=\leftmark
\let\oldrightmark=\rightmark
\renewcommand{\leftmark}{ПРЕДИСЛОВИЕ}
\renewcommand{\rightmark}{ПРЕДИСЛОВИЕ}
\input{foreword}

\chapter{Основные понятия и определения}
\let\leftmark=\oldleftmark
\let\rightmark=\oldrightmark

\section{Краткая история криптографии}

Вслед за возникновением письменности появилась задача обеспечения секретности и подлинности передаваемых сообщений путем так называемой тайнописи. Поскольку государства возникали почти одновременно с письменностью, дипломатия и военное управление требовали секретности.

Данные о первых способах тайнописи весьма обрывочны. В древнеиндийских трактатах можно встретить упоминания о способах преобразования текста, некоторые из которых можно отнести к криптографии. Предполагается, что тайнопись была известна в Древнем Египте и Вавилоне. До нашего времени дошли литературные свидетельства того, что секретное письмо использовалось в Древней Греции. В Древней Спарте использовалась скитала\index{скитала} (<<шифр Древней Спарты>>\index{шифр!Древней Спарты}, рис.~\ref{fig:Skytale}), которая также является одним из древнейших известных криптографических устройств. Скитала представляла собой длинный цилиндр, на который наматывалась полоска пергамента. Текст писали поперёк ленты (вдоль цилиндра). Для расшифровки был необходим цилиндр аналогичного диаметра. Считается, что ещё Аристотель предложил метод криптоанализа скиталы. Не зная точного диаметра оригинального цилиндра, он предложил наматывать пергамент на конус до тех пор, пока текст не начнёт читаться. Аристотеля можно называть одним из первых известных криптоаналитиков.

\begin{figure}[t]
	\centering
	\subfloat[Скитала. Рисунок современного автора. Рисунок участника Wikimedia Commons Luringen, доступно по \href{https://creativecommons.org/licenses/by-sa/3.0/deed.ru}{лицензии CC-BY-SA 3.0}]{\label{fig:Skytale}\includegraphics[width=0.60\textwidth]{pic/Skytale}}
	~~~~
	\subfloat[Аристотель (384~до~н.~э. -- 322~до~н.~э.). Римская копия оригинала Лисиппа]{\includegraphics[width=0.35\textwidth]{pic/Aristotle_Altemps_Inv8575}}
	\caption{Скитала\index{скитала}, <<шифр Древней Спарты>>\index{шифр!Древней Спарты}}
\end{figure}

В Ветхом Завете, в том числе в книге пророка Иеремии (VI~век до~н.~э.), использовалась техника скрытия отдельных кусков текста, получившая название <<атбаш>>\index{шифр!атбаш}.

\begin{itemize}
	\item \texttt{Иер. 25:26}: и всех царей севера, близких друг к другу и дальних, и все царства земные, которые на лице земли, а царь Сесаха выпьет после них
	\item \texttt{Иер. 51:41}: Как взят Сесах, и завоевана слава всей земли! Как сделался Вавилон ужасом между народами!
\end{itemize}

В этих отрывках слово <<Сесах>> относится к государству, про которое не упоминается в других источниках. Но если взять написание слова <<Сесах>> на иврите, заменить первую букву алфавита на последнюю, вторую на предпоследнюю, и так далее, то вместо <<Сесах>> получится <<Бавель>> -- одно из названий города Вавилон. То есть с помощью техники <<атбаш>> авторы манускрипта скрывали отдельные названия, оставляя большую часть текста без шифрования. Возможно это делалось в том числе и для того, чтобы не иметь проблем с распространением текстов на территории, подконтрольной Вавилону. Шифр <<атбаш>>\index{шифр!атбаш} можно рассматривать как пример моноалфавитного афинного шифра (см. раздел~\ref{section-affine-cipher}).

Сразу несколько техник защищённой передачи сообщений связывают с именем Энея Тактика, полководца IV~века до~н.~э.
\begin{itemize}
	\item \textbf{Диск Энея} представлял собой диск небольшого диаметра с отверстиями, которые соответствовали буквам алфавита. Отправитель протягивал нитку через отверстия, тем самым кодируя сообщение. Диск с ниткой отправлялся получателю. Особенностью диска Энея было то, что в случае захвата гонца, последний мог быстро выдернуть нитки из диска, фактически уничтожив передаваемое сообщение;
	\item \textbf{Линейка Энея} представляла собой линейку с отверстиями, соответствующими буквам алфавита. Нитку также продевали через отверстия, тем самым шифруя сообщение. Однако после продевания на нитке завязывали узлы. После окончания нитку снимали с линейки и отправляли получателю. Чтобы восстановить сообщение, получатель должен был иметь линейку с таким же порядком отверстий, как та, на которой текст шифровался. Подобный метод можно назвать моноалфавитным шифром (см. раздел~\ref{section-substitution-cipher}), исходное сообщение -- открытым текстом, нитку с узлами -- шифротекстом, а саму линейку -- ключом шифрования.
	\item Ещё одна техника, \textbf{книжный шифр Энея}, состояла в прокалывании небольших отверстий в книге или манускрипте рядом с буквами, соответствующими буквам исходного сообщения. Этот метод относится уже не к криптографии, а к стеганографии -- науке о скрытии факта передачи сообщения.
\end{itemize}

Ко II~веку до~н.~э. относят изобретение в Древней Греции квадрата Полибия (рис.~\ref{fig:polubios-square}). Метод позволял передавать информацию на большие расстояния с помощью факелов. Каждой букве алфавита ставилось в соответствии два числа от 1 до 5 (номера строки и столбца в квадрате Полибия). Эти числа обозначали количество факелов, которые необходимо поднять на сигнальной башне. Квадрат Полибия относится к методам кодирования информации: переводу информации из одного представления (греческого алфавита) в другое (число факелов) для удобства хранения, обработки или передачи.

\begin{figure}[t]
	\centering
\begin{tabular}{ || c || c | c | c | c | c ||}
\hline
\hline
  & 1 & 2 & 3 & 4 & 5 \\
\hline
\hline
1 & A & B & $\Gamma$ & $\Delta$ & E \\
\hline
2 & Z & H & $\Theta$ & I & K \\
\hline
3 & $\Lambda$ & M & N & $\Xi$ & O  \\
\hline
4 & $\Pi$ & P & $\Sigma$ & T & $\Upsilon$ \\
\hline
5 & $\Phi$ & X & $\Psi$ & $\Omega$ & \\
\hline
\hline
\end{tabular}
  \caption{Квадрат Полибия для греческого алфавита}
  \label{fig:polubios-square}
\end{figure}

Известен метод шифрования, который использовался Гаем Юлием Цезарем (100--44~гг.~до~н.~э.). Он получил название <<шифр Цезаря>>\index{шифр!Цезаря} и состоял в замене каждой буквы текста на другую букву, следующую в алфавите через две позиции (см. раздел~\ref{section-caesar-cipher}). Данный метод относится к классу моноалфавитных шифров.

В VIII веке~н.~э. была опубликована <<Книга тайного языка>> Аль-Халиля аль-Фарахиди, в котором арабский филолог описал технику криптоанализа, сейчас известную как атака по открытому тексту. Он предположил, что первыми словами письма, которое было отправлено византийскому императору, будет фраза <<Во имя Аллаха>>, что оказалось верным и позволило расшифровать оставшуюся часть письма. Абу аль-Кинди (801--873~гг.~н.~э.) в своём <<Трактате о дешифровке криптографических сообщений>> показал, что моноалфавитные шифры, в которых каждому символу кодируемого текста ставится в однозначное соответствие какой-то другой символ алфавита, легко поддаются частотному криптоанализу. В тексте трактата аль-Кинди привёл таблицу частот букв, которую можно использовать для дешифровки шифротекстов на арабском языке, использующих моноалфавитный шифр.

\begin{figure}[t]
	\centering
	\subfloat[Статуя Леона Баттиста Альберти (\langit{Leone Battista Alberti}, 1404--1472) во дворе Уффици. Фото участника it.wiki Frieda, доступно по \href{https://creativecommons.org/licenses/by-sa/3.0/deed.ru}{лицензии CC-BY-SA 3.0}]{\includegraphics[width=0.50\textwidth]{pic/Leon_Battista_Alberti_1}}
	~~~~
	\subfloat[Фрагмент оформления гробницы Иоганна Тритемия (\langlat{Iohannes Trithemius}, 1462--1516)]{\includegraphics[width=0.45\textwidth]{pic/Trithemiusmoredetail}}
	\caption{Отцы западной криптографии}
\end{figure}

Итальянский архитектор Леон Баттиста Альберти, проанализировав использовавшиеся в Европе шифры, предложил для каждого текста использовать не один, а несколько моноалфавитных шифров. Однако Альберти не смог предложить законченной идеи полиалфавитного шифра, хотя его и называют отцом западной криптографии. В истории развития полиалфавитных шифров до XX века также наиболее известны немецкий аббат XVI века Иоганн Тритемий и английский ученый начала XIX века Чарльз Уитстон (\langen{Charles Wheatstone}, 1802--1875). Уитстон изобрел простой и стойкий способ полиалфавитной замены, называемый шифром Плейфера\index{шифр!Плейфера} в честь лорда Плейфера, способствовавшему внедрению шифра. Шифр Плейфера использовался вплоть до Первой мировой войны.

\begin{figure}[t]
	\centering
	\subfloat[<<Энигма>>]{\label{fig:enigma}\includegraphics[width=0.35\textwidth]{pic/EnigmaMachine}}
	~~
	\subfloat[<<Лоренц>> (без кожуха)]{\label{fig:lorenz}\includegraphics[width=0.60\textwidth]{pic/Lorenz-SZ42-2}}
	\caption{Криптографические машины Второй мировой}
\end{figure}

Роторные машины XX века позволяли создавать и реализовывать устойчивые к <<наивному>> взлому полиалфавитные шифры. Примером такой машины является немецкая машина <<Энигма>>\index{Энигма}, разработанная в конце Первой мировой войны (рис.~\ref{fig:enigma}). Период активного применения <<Энигмы>> пришелся на Вторую мировую войну. Хотя роторные машины использовались в промышленных масштабах, криптография, на которой они были основаны, представляла собой всё ещё искусство, а не науку. Отсутствовал научный базис надёжности криптографических инструментов. Возможно, это было одной из причин успеха криптоанализа <<Энигмы>>, который сначала был осуществлён в Польше в <<Бюро шифров>>, а потом и в <<Блетчли-парке>> в Великобритании. Польша впервые организовала курсы криптографии не для филологов и специалистов по немецкому языку, а для математиков, хотя и знающих язык весьма вероятного противника. Трое из выпускников курса — Мариан Реевский, Генрих Зыгальский и Ежи Рожицкий — поступили на службу в «Бюро шифров» и получили первые результаты успешного криптоанализа. Используя математику, электромеханические приспособления и данные французского агента Asche (Ганс-Тило Шмидт), они могли дешифровывать значительную часть сообщений вплоть до лета 1939 года, когда вторжение Германии в Польшу стало очевидным. Дальнейшая работа по криптоанализу <<Энигмы>> в центре британской разведки <<Station~X>>\index{Station X} (<<Блетчли-парк>>\index{Блетчли-парк}) связана с именами таких известных математиков, как Гордон Уэлчман и Алан Тьюринг. Кроме <<Энигмы>> в центре проводили работу над дешифровкой и других шифров, в том числе немецкой шифровальной машины <<Лоренц>> (рис.~\ref{fig:lorenz}). Для целей её криптоанализа был создан компьютер Colossus, имевший 1500 электронных ламп, а его вторая модификация -- Colossus Mark II -- считается первым в мире программируемым компьютером в истории ЭВМ.

Середина XX века считается основной вехой в истории науки о защищённой передаче информации и криптографии. Эта веха связана с публикацией двух работ Клода Шеннона: <<Математическая теория связи>> (<<A Mathematical Theory of Communication>>, 1948, \cite{Shannon:1948:MTCa, Shannon:1948:MTCb}) и <<Теория связи в секретных системах>> (<<Communication Theory of Secrecy Systems>>, 1949, \cite{Shannon:1949:CTS}). В данных работах Шеннон впервые определил фундаментальные понятия в теории информации, а также показал возможность применения этих понятий для защиты информации, тем самым заложив математическую основу современной криптографии.

Кроме того, появление электронно-вычислительных машин кардинально изменило ситуацию в криптографии. С одной стороны, вычислительные способности ЭВМ подняли на совершенно новый уровень возможности реализации шифров, недоступных ранее из-за их высокой сложности. С другой стороны, аналогичные возможности стали доступны и криптоаналитикам. Появилась необходимость не только в создании шифров, но и хоть в каком-нибудь обосновании того, что новые вычислительные возможности не смогут быть использованы для взлома новых шифров.

В 1976 году появился шифр DES (Data Encryption Standard)\index{шифр!DES}, который был принят как стандарт США. DES широко использовался для шифрования пакетов данных при передаче в компьютерных сетях и системах хранения данных. С 90-х годов параллельно с традиционными шифрами, основой которых была булева алгебра, активно развиваются шифры, основанные на операциях в конечном поле. Широкое распространение персональных компьютеров и быстрый рост объёма передаваемых данных в компьютерных сетях привели к замене в 2002 году стандарта DES на более стойкий и быстрый в программной реализации стандарт -- шифр AES (Advanced Encryption Standard)\index{шифр!AES}. Окончательно, DES был выведен из эксплуатации как стандарт в 2005 году.

В беспроводных голосовых сетях передачи данных используются шифры с малой задержкой шифрования и расшифрования на основе посимвольных преобразований -- так называемые \emph{потоковые шифры}\index{шифр!потоковый}.

%Основным их преимуществом является сочетание помехоустойчивого кодирования с криптостойкостью шифра.

Параллельно с разработкой быстрых шифров в 1977 г. появился новый класс криптосистем, так называемые \emph{криптосистемы с открытым ключом}\index{криптосистема!с открытым ключом}. Хотя эти новые криптосистемы намного медленнее (технически сложнее) симметричных, они открыли принципиально новые возможности --  \emph{электронная подпись}, \emph{аутентификация} и \emph{сертификация} составили основу современной защищённой связи в Интернете.

В настоящее время типичное использование криптографии в информационных системах состоит в:
\begin{itemize}
\item цифровой аутентификации пользователей с помощью криптосистем с открытым ключом;
\item создании кратковременных сеансовых ключей;
\item применении быстрых шифров в процессах обмена данными.
\end{itemize}


\section{Модель системы передачи с криптозащитой}
\selectlanguage{russian}

Простая модель системы передачи с криптозащитой представлена на рис.~\ref{pic:Encrypt}, где введены следующие обозначения:
\begin{itemize}
    \item $A$ -- источник информации;
    \item $B$ -- получатель информации, легальный пользователь;
    \item $X$ -- сообщение до шифрования или \emph{открытый текст}\index{открытый текст} (\langen{plaintext}); $\set{M}$ -- множество всех возможных открытых текстов (от слова Message), $X \in \set{M}$;
    \item $K_1$ -- ключ шифрования\index{ключ!шифрования} (\langen{encryption key}); $\set{K}_E$ -- множество всех возможных ключей шифрования, $K_1 \in \set{K}_E$;
    \item $Y$ -- зашифрованное сообщение (\emph{шифротекст}\index{шифротекст}, \langen{ciphertext, cyphertext} или \emph{шифрограмма}\index{шифрограмма}\footnote{Строго говоря, \emph{шифрограмма} -- это \emph{шифротекст} после его \emph{кодирования} для целей передачи по каналу связи}); $\set{C}$ -- множество всех возможных шифротекстов, $Y \in \set{C}$;
    \item $K_2$ -- ключ расшифрования\index{ключ!расшифрования} (\langen{decryption key}); $\set{K}_D$  -- множество возможных ключей расшифрования, зависящее от множества $\set{K}_E$, $K_2 \in \set{K}_D$.
\end{itemize}

\begin{figure}[!thb]
	\centering
	\includegraphics[width=1.0\textwidth]{pic/scheme-of-cipher}
	\caption{Передача информации с криптозащитой\label{pic:Encrypt}}
\end{figure}

\emph{Шифр}\index{шифр} -- это множество обратимых функций отображения $E_{K_1}$\index{функция!шифрования} множества открытых текстов $\set{M}$ на множество шифротекстов $\set{C}$, зависящих от выбранного ключа шифрования $K_1$ из множества $\set{K}_E$:
%обратимое отображение пары из элемента множества открытых текстов $\set{M}$ и элемента множества ключей шифрования $\set{K}_E$ в множество шифротекстов $\set{C}$:
\begin{equation}
    \label{eq:Encryption}
    Y = E_{K_1}(X), ~ X \in \set{M}, ~ K_1 \in \set{K}_E, ~ Y \in \set{C}.
\end{equation}
Можно сказать, что шифрование -- это обратимая функция двух аргументов: сообщения и ключа. Для каждого $K_1$ эта функция должна быть обратимой. Обратимость -- основное условие шифрования, по которому каждому зашифрованному сообщению $Y$ и ключу $K$ соответствует одно исходное сообщение $X$. Легальный пользователь $B$ (на приёмной стороне системы связи)  получает сообщение $Y$ и осуществляет процедуру \emph{расшифрования}\index{расшифрование}.
Следует отличать шифрование от кодирования, так как кодирование -- это процесс сопоставления конкретным сообщениям строго определённой комбинации символов или сигналов, с целью повышения помехоустойчивости передаваемого сигнала.
Расшифрование --  это отображение множества шифротекстов $\set{C}$ в множество открытых текстов $\set{M}$ функцией $D_{K_2}$\index{функция!расшифрования}, зависящей от ключа расшифрования $K_2$ из множества $\set{K}_D$, являющейся обратной к функции $E_{K_1}$.
\begin{equation}
    \label{eq:Decryption}
    D_{K_2}(Y) = X, ~ Y \in \set{C}, ~ K_2 \in \set{K}_D, ~ X \in \set{M}.
\end{equation}

%Система передачи информации с криптозащитой называется \emph{криптосистемой}\index{криптосистема}.(?????)

%В общем случае функция шифрования сюръективна и псевдослучайна, отображая один открытый текст в разные шифротексты. Если функция шифрования биективна, на практике ее инкапсулируют в другую функцию с целью добиться псевдослучайности шифрования одинаковых открытых текстов в разные шифротексты.

%Методы защиты информации зависят от возможных сценариев передачи. Рассмотрим несколько основных вариантов.
Рассмотрим возможные сценарии вмешательства криптоаналитика и организации защиты информации от его действий.
Пусть  $A$ -- источник и $B$ -- получатель сообщений.

\begin{description}
    \item[Сценарий 1.] Пусть $E$ -- \emph{пассивный} криптоаналитик\index{криптоаналитик!пассивный}, который может подслушивать передачу, но не может вмешиваться в процесс передачи. Из пассивности криптоаналитика следует, что $Y = \widetilde{Y}$, и \emph{целостность} информации обеспечена.

Цель защиты --- \emph{обеспечение конфиденциальности}.

Средства защиты -- шифрование с помощью \emph{симметричных} или \emph{асимметричных } криптосистем.

Дополнительные задачи -- при большом числе пользователей должна быть решена задача \emph{генерации и доставки секретных ключей} всем пользователям.

    \item[Сценарий 2.] Пусть $E$ -- \emph{активный} криптоаналитик\index{криптоаналитик!активный}, который может изменять, удалять и вставлять сообщения или их части.

    Цель защиты -- \emph{обеспечение конфиденциальности} и  \emph{обеспечение целостности}.

Средства защиты --  шифрование и добавление \emph{имитовставки}\index{имитовставка} (message authentication code -- $\MAC$), позволяющего обнаружить нарушение целостности.

    \item[Сценарий 3.] Пусть $E$ -- активный криптоаналитик, который может изменять, удалять и вставлять сообщения или их части, дополнительно к этому легальные пользователи $A$ и $B$ не доверяют друг другу.

Цель защиты -- \emph{аутентификация} пользователей и документов.

Средства -- \emph{электронная подпись} и протокол идентификации (аутентификации) пользователей.
\end{description}

%%Возможно вмешательство нелегального пользователя $E$, называемого \emph{криптоаналитиком}.
%%
%%
%%Если $X = \widetilde{X}$, то вмешательство криптоаналитика  $E$ не изменило передаваемое сообщение, и \emph{целостность} информации обеспечена. Если криптоаналитик не получил информацию, содержащуюся в сообщении, то обеспечена \emph{конфиденциальность}.
%%
%%Если в этой системе возможна двусторонняя передача, то есть от $A$ к $B$ и от $B$ к $A$, то говорят о взаимном обмене информацией между легальными пользователями.
%
%Секретность информации в современных шифрах обеспечивается секретным ключом, в то время как сам алгоритм криптосистемы является общеизвестным. Исторический опыт, например, система шифрования A5/1 в GSM, показывает, что секретность алгоритма шифрования \emph{ослабляет} криптостойкость шифра, а не увеличивает, из-за того, что система становится малоизученной.


\section{Классификация криптосистем}

\input{classification_by_symmetry}

\subsection{Шифры замены и перестановки}

Шифры, по способу преобразования открытого текста в шифротекст, разделяются на шифры замены и шифры перестановки.

\input{substitution_ciphers}

\subsubsection{Шифры перестановки}
\selectlanguage{russian}

Шифры \textbf{перестановки} реализуются следующим образом. Берут открытый текст, например буквенный, и разделяют на блоки определённой длины $x_1, x_2, \dots, x_m$. Затем осуществляется перестановка позиций блока (вместе с символами). Перестановки могут быть однократные и многократные. Частный случай перестановки -- сдвиг. Приведём пример:
\begin{center}
    секрет $\xrightarrow{\text{сдвиг}}$ ретсек $\xrightarrow{\text{перестановка}}$ рскете.
\end{center}
Ключ такого шифра указывает изменение порядка номеров позиций блока при шифровании и расшифровании.

Существуют так называемые \textbf{маршрутные перестановки}. Используется какая-либо геометрическая фигура, например, прямоугольник. Запись открытого текста ведётся по одному \emph{маршруту}, например, по строкам, а считывание для шифрования осуществляется по другому маршруту, например по столбцам. Ключ шифра определяет эти маршруты.
В случае, когда рассматривается перестановка блока текста фиксированной длины, перестановку можно рассматривать как замену.

В полиалфавитных шифрах при шифровании открытый текст разбивается на блоки (последовательности) длины $n$, где $n$ -- \textbf{период}. Этот параметр выбирает \emph{криптограф} и держит его в секрете.

Поясним процедуру шифрования полиалфавитным шифром. Запишем шифруемое сообщение в матрицу по столбцам определённой длины. Пусть открытый текст таков: <<Игры различаются по содержанию, характерным особенностям, а также по тому, какое место они занимают в жизни детей>>. Зададим $n=4$ и запишем этот текст в матрицу размера $(4 \times 24)$:

\begin{center} \resizebox{\textwidth}{!}{ \begin{tabular}{|*{24}{c|}}
    \hline
    и&р&и&т&о&е&н&а&т&ы&о&н&я&а&п&м&к&е&о&а&а&ж&и&е \\
    г&а&ч&с&с&р&и&р&е&м&б&о&м&к&о&у&о&с&н&н&ю&и&д&й \\
    р&з&а&я&о&ж&ю&а&р&о&е&с&а&ж&т&к&е&т&и&и&т&з&е& \\
    ы&л&ю&п&д&а&х&к&н&с&н&т&т&е&о&а&м&о&з&м&в&н&т& \\
    \hline
\end{tabular} } \end{center}

Выбираем $4$ различных моноалфавитных шифра.

Первую строку

\begin{center} \resizebox{\textwidth}{!}{ \begin{tabular}{|*{24}{c|}}
    \hline
    и&р&и&т&о&е&н&а&т&ы&о&н&я&а&п&м&к&е&о&а&а&ж&и&е \\
    \hline
\end{tabular} } \end{center}

шифруем, используя первый шифр. Вторую строку

\begin{center} \resizebox{\textwidth}{!}{ \begin{tabular}{|*{24}{c|}}
    \hline
    г&а&ч&с&с&р&и&р&е&м&б&о&м&к&о&у&о&с&н&н&ю&и&д&й \\
    \hline
\end{tabular} } \end{center}

шифруем, используя второй шифр, и т.~д.

Выполняя расшифрование, легальный пользователь знает период. Он записывает принятую шифрограмму по строкам в матрицу с длиной строки равной периоду, к каждому столбцу применяет соответствующий ключ и расшифровывает сообщение, зная соответствующие шифры.

Шифры перестановки можно рассматривать как частный случай шифров замены, если отождествить один блок перестановки с одним символом большого алфавита.


\input{composite_ciphers}

\subsection{Примеры современных криптографических примитивов}

Приведём примеры названий некоторых современных криптографических примитивов, из которых строят системы защиты информации:
\begin{itemize}
    \item DES\index{шифр!DES}, AES, ГОСТ 28147-89, Blowfish\index{шифр!Blowfish}, RC5\index{шифр!RC5}, RC6\index{шифр!RC6} -- блоковые симметричные шифры, скорость обработки -- десятки мегабайт в секунду,
    \item A5/1, A5/2, A5/3\index{шифр!A5}, RC4\index{шифр!RC4} -- потоковые симметричные шифры с высокой скоростью, семейство A5 применяется в мобильной связи GSM, RC4 -- в компьютерных сетях для SSL соединения между браузером и веб-сервером,
    \item RSA\index{шифр!RSA} -- криптосистема с открытым ключом для шифрования,
    \item RSA\index{электронная подпись!RSA}, DSA\index{электронная подпись!DSA}, ГОСТ Р 34.10-2001\index{электронная подпись!ГОСТ Р 34.10-2001} -- криптосистемы с открытым ключом для электронной подписи,
    \item MD5\index{хэш-функция!MD5}, SHA-1\index{хэш-функция!SHA-1}, SHA-2\index{хэш-функция!SHA-2}, ГОСТ Р 34.11-94\index{хэш-функция!ГОСТ Р 34.11-94} -- криптографические хэш-функции.
\end{itemize}

\section{Методы криптоанализа и типы атак}
\selectlanguage{russian}

Нелегальный пользователь-криптоаналитик получает информацию путем дешифрования. Сложность этой процедуры определяется числом стандартных операций, которые надо выполнить для достижения цели. \emph{Двоичной сложностью}\index{сложность!двоичная} (или битовой сложностью) алгоритма называется количество двоичных операций, которые необходимо выполнить для его завершения.
% Наиболее сложным является дешифрование полиалфавитных шифров.

Попытка криптоаналитика $E$ получить информацию называется \emph{атакой} или криптоатакой\index{атака}. Как правило, легальным пользователям нужно обеспечить защиту информации на протяжении от нескольких дней до 100 лет. Если попытка атаки оказалась удачной для нелегального пользователя $E$, и информация получена или может быть получена в ближайшем будущем, то такое событие называется  \emph{взломом криптосистемы}\index{взлом криптосистемы} или \emph{вскрытием криптосистемы}. Метод вскрытия криптосистемы называется \emph{криптоанализом}\index{криптоанализ}. Криптосистема называется \emph{криптостойкой}\index{криптостойкость}, если число стандартных операций для ее взлома превышает возможности современных вычислительных средств в течение всего времени ценности информации (до 100 лет).

В общем случае, в криптоанализе под \emph{взломом} криптосистемы понимается построение алгоритма криптоатаки для получения доступа к информации с количеством операций, меньшим, чем планировалось при создании этой криптосистемы. Взлом криптосистемы -- это не обязательно реально осуществленное извлечение информации, так как количество операций для извлечения информации может быть вычислительно недостижимым как в настоящее время, так и в течение всего времени защиты.
%, но предполагается достижимым в будущем.

Рассмотрим основные сценарии работы криптоаналитика $E$. В первом сценарии криптоаналитик может осуществлять подслушивание и (или) перехват сообщений. Его вмешательство не нарушает целостности информации: $Y=\widetilde{Y}$. Эта роль криптоаналитика называется \emph{пассивной}. Так как он получает доступ к информации, то здесь нарушается конфиденциальность.

Во втором сценарии роль криптоаналитика \emph{активная}. Он может подслушивать, перехватывать сообщения и преобразовывать их по своему усмотрению: задерживать, искажать с помощью перестановок пакетов, устраивать обрыв связи, создавать новые сообщения и т.~п. Так что в этом случае выполняется условие $Y \neq \widetilde{Y}$. Это значит, что одновременно нарушается целостность и конфиденциальность передаваемой информации.

Приведём примеры пассивных и активных атак.
\begin{itemize}
    \item Атака <<\emph{человек посередине}>>\index{атака!<<человек посередине>>} (\langen{man-in-the-middle}) подразумевает криптоаналитика, который разрывает канал связи, встраиваясь между $A$ и $B$, получает сообщения от $A$ и от $B$, а от себя отправляет новые, фальсифицированные сообщения. В результате $A$ и $B$ не замечают, что общаются с $E$, а не друг с другом.
    \item Атака \emph{воспроизведения}\index{атака!воспроизведения} (\langen{replay attack}) -- когда криптоаналитик может записывать и в будущем воспроизводить шифротексты, имитируя легального пользователя.
    \item Атака на \emph{различение} сообщений\index{атака!на различение} означает, что криптоаналитик, наблюдая одинаковые шифротексты, может извлечь информацию об идентичности исходных открытых текстов.
    \item Атака на \emph{расширение} сообщений\index{атака!на расширение} означает, что криптоаналитик может дополнить шифротекст осмысленной информацией без знания секретного ключа;
    \item \emph{Фальсификация} шифротекстов\index{атака!фальсификацией} криптоаналитиком без знания секретного ключа.
\end{itemize}

Часто для нахождения секретного ключа криптоатаки строят в предположениях о доступности дополнительной информации. Приведём примеры.
\begin{itemize}
    \item Атака на основе известного открытого текста\index{атака!с известным открытым текстом} (\langen{chosen plaintext attack, CPA}) предполагает возможность криптоаналитику выбирать открытый текст и получать для него соответствующий шифротекст.
    \item Атака на основе известного шифротекста\index{атака!с известным шифротекстом} (\langen{chosen ciphertext attack, CCA}) предполагает возможность криптоаналитику выбирать шифротекст и получать для него соответствующий открытый текст.
\end{itemize}

Обязательным требованием к современным криптосистемам является устойчивость ко всем известным типам атак: пассивным, активным и с дополнительной информацией.


%Приведём примеры возможных вариантов работы активного криптоаналитика.
%\begin{itemize}
%\item Криптоаналитик имеет $m$ шифрованных сообщений $Y_{1},Y_{2},\ldots Y_{m}$ и пытается определить ключ или прочитать открытый текст $X_{1},X_{2},\ldots X_{m}.$
%\item Криптоаналитик имеет несколько пар открытого и шифрованного текстов
%
%$(Y_{1},X_{1}),(Y_{2}X_{2}),\ldots (Y_{m}X_{m})$ и пытается дешифровать остальной текст или определить алгоритм шифрования или определить ключ.
%\item
%\item
%\item
%\end{itemize}

Для защиты информации от активного криптоаналитика и обеспечения целостности дополнительно к шифрованию сообщений применяют имитовставку\index{имитовставка}. Для неё используют обозначение $\MAC$ (\langen{message authentication code}). Как правило, $\MAC$ строится на основе хэш-функций, которые будут описаны далее.

Существуют ситуации, когда пользователи $A$ и $B$ не доверяют друг другу. Например, $A$ -- банк, $B$ -- получатель денег. $A$ утверждает, что деньги переведены, $B$ утверждает, что не переведены. Решение задачи аутентификации и неотрицаемости состоит в обеспечении \emph{электронной подписью}\index{электронная подпись} каждого из абонентов. Предварительно надо решить задачу о генерировании и распределении секретных ключей.

В общем случае системы защиты информации должны обеспечивать:
\begin{itemize}
    \item конфиденциальность (защита от наблюдения),
    \item целостность (защита от изменения),
    \item аутентификацию (защита от фальсификации пользователя и сообщений),
    \item доказательство авторства информации (доказательство авторства и защита от его отрицания)
\end{itemize}
как со стороны получателя, так и со стороны отправителя.

Важным критерием для выбора степени защиты является сравнение стоимости реализации взлома для получения информации и экономического эффекта от владения ею. Очевидно, что если стоимость взлома превышает ценность информации, взлом нецелесообразен.

%Сценарии защиты информации
%   Сценарий 1. A -- передающая сторона. B -- принимающая сторона. E -- пассивный
%криптоаналитик, который может подслушивать передачу, но не может вмешиваться
%в процесс передачи. Цель защиты: обеспечение конфиденциальности. Средства
%-- методы шифрования с секретным ключом (симметричные системы шифрования)
%и методы шифрования с открытым ключом (асимметричные системы шифрования).
%Сценарий 2. E -- активный криптоаналитик, который может изменять, удалять и вставлять
%сообщения или их части. Цель защиты -- обеспечение конфиденциальности (не
%всегда) и обеспечение целостности. Средства -- методы шифрования и добавление
%имитовставки\index{имитовставка} (Message Autentication Code -- $\MAC$).
%Сценарий 3. A и B не доверяют друг другу. Цель защиты -- аутентификация пользователя.
%Средства -- электронная подпись.


\input{The_minimum_key_lengths}

\chapter{Классические шифры}

В главе приведены наиболее известные \emph{классические} шифры, которыми можно было пользоваться до появления роторных машин. К ним относятся такие шифры, как: шифр Цезаря\index{шифр!Цезаря}, шифр Плейфера\index{шифр!Плейфера}, шифр Хилла\index{шифр!Хилла}, шифр Виженера\index{шифр!Виженера}. Они наглядно демонстрируют различные классы шифров.

\section{Моноалфавитные шифры}\label{section-substitution-cipher}\index{шифр!моноалфавитный|(}
\selectlanguage{russian}

Преобразования открытого текста в шифротекст могут быть описаны различными функциями. Если функция преобразования является аддитивной, то и соответствующий шифр называется \emph{аддитивным}. Если это преобразование является аффинным, то шифр называется \emph{аффинным}.

\subsection{Шифр Цезаря}\label{section-caesar-cipher}\index{шифр!Цезаря}

Известным примером простого шифра замены является \emph{шифр Цезаря}. Процедура шифрования состоит в следующем (рис.~\ref{fig:caesar}). Записывают все буквы латинского алфавита в стандартном порядке
    \[ A B C D E \dots Z. \]
Делают циклический сдвиг влево, например, на три буквы, и записывают все буквы во втором ряду, начиная с четвёртой буквы $D$. Буквы первого ряда заменяют соответствующими (как показано стрелкой на рисунке) буквами второго ряда. После такой замены слова не распознаются теми, кто не знает ключа. Ключом $K$ является первый символ сдвинутого алфавита.

\begin{figure}[thb]
\[ \begin{array}{ccccccccccc}
    \text{A} & \text{B} & \text{C} & \text{D} & \text{E} & & \text{V} & \text{W} & \text{X} & \text{Y} & \text{Z} \\
    \downarrow & \downarrow & \downarrow & \downarrow & \downarrow & \dots & \downarrow & \downarrow & \downarrow & \downarrow & \downarrow \\
    \text{D} & \text{E} & \text{F} & \text{G} & \text{H} & & \text{Y} & \text{Z} & \text{A} & \text{B} & \text{C} \\
\end{array} \]
	\caption{Шифр Цезаря\index{шифр!Цезаря}}
	\label{fig:caesar}
\end{figure}

\example
В русском языке сообщение \texttt{изучайтекриптографию} посредством шифрования с ключом $K = \text{\texttt{г}}$ (сдвиг вправо на 3 символа по алфавиту) преобразуется в \texttt{лкцъгмхзнултхсёугчлб}.
\exampleend

Недостатком любого шифра замены является то, что в шифрованном тексте сохраняются все частоты появления букв открытого текста и корреляционные связи между буквами. Они существуют в каждом языке. Например, в русском языке чаще всего встречаются буквы $A$ и $O$. Для дешифрования криптоаналитик имеет возможность прочитать открытый текст, используя частотный анализ букв шифротекста. Для <<взлома>> шифра Цезаря достаточно найти одну пару букв -- одну замену.

\subsection{Аддитивный шифр перестановки}\index{шифр!перестановки аддитивный}

Рисунок~\ref{fig:caesar-additiv} поясняет \emph{аддитивный шифр} перестановки на алфавите. Все 26 букв латинского алфавита нумеруют по порядку от 0 до 25. Затем номер буквы меняют в соответствии с уравнением
    \[ y = x + b \mod 26, \]
где $x$ -- прежний номер, $y$ -- новый номер, $b$ -- заданное целое число, определяющее сдвиг номера и известное только легальным пользователям. Очевидно, что шифр Цезаря является примером аддитивного шифра.

\begin{figure}[thb]
\[ \begin{array}{ccccccccccc}
    \text{A} & \text{B} & \text{C} & \text{D} & \text{E} & & \text{V} & \text{W} & \text{X} & \text{Y} & \text{Z} \\
    \downarrow & \downarrow & \downarrow & \downarrow & \downarrow & \dots & \downarrow & \downarrow & \downarrow & \downarrow & \downarrow \\
    0 & 1 & 2 & 3 & 4 & & 21 & 22 & 23 & 24 & 25 \\
    \downarrow & \downarrow & \downarrow & \downarrow & \downarrow & \dots & \downarrow & \downarrow & \downarrow & \downarrow & \downarrow \\
    3 & 4 & 5 & 6 & 7 & & 24 & 25 & 0 & 1 & 2 \\
    \downarrow & \downarrow & \downarrow & \downarrow & \downarrow & \dots & \downarrow & \downarrow & \downarrow & \downarrow & \downarrow \\
    \text{D} & \text{E} & \text{F} & \text{G} & \text{H} & & \text{Y} & \text{Z} & \text{A} & \text{B} & \text{C} \\
\end{array} \]
	\caption{Шифр Цезаря\index{шифр!Цезаря} как пример аддитивного шифра\index{шифр!перестановки аддитивный}}
	\label{fig:caesar-additiv}
\end{figure}

\subsection{Аффинный шифр}\label{section-affine-cipher}\index{шифр!афинный}

Аддитивный шифр является частным случаем \emph{аффинного шифра}. Правило шифрования сообщения имеет вид
    \[ y = a x + b \mod n. \]
Здесь производится умножение номера символа $x$ из алфавита, $x\in \set\{ 0, 1, 2, \dots, N \leq n-1 \}$, на заданное целое число $a$ и сложение с числом $b$ по модулю целого числа $n$. Ключом является $K = (a, b)$.

Расшифрование осуществляется по формуле
    \[ x = (y - b) a^{-1} \mod n. \]

Чтобы обеспечить обратимость в этом шифре, должен существовать единственный обратный элемент $a^{-1}$ по модулю $n$. Для этого должно выполняться условие $\gcd(a,n) = 1$, то есть $a$ и $n$ должны быть взаимно простыми числами ($\gcd$ -- обозначение термина с английского greatest common divisor -- наибольший общий делитель, $\text{НОД}$). Очевидно, что для <<взлома>> такого шифра достаточно найти две пары букв -- две замены.

\index{шифр!моноалфавитный|)}


\section{Биграммные шифры замены}\index{шифр!биграммный}
\selectlanguage{russian}

Если при шифровании преобразуются по две буквы открытого текста, то такой шифр называется \textbf{биграммным}\index{шифр!биграммный} шифром замены. Первый биграммный шифр был изобретен аббатом Иоганном Тритемием и опубликован в 1508 году. Другой биграммный шифр изобретен в 1854 году Чарльзом Витстоном. Лорд Лайон Плейфер внедрил этот шифр в государственных службах Великобритании, и шифр был назван шифром Плейфера\index{шифр!Плейфера}.

Опишем шифр Плейфера\index{шифр!Плейфера}. Составляется таблица для английского алфавита (буквы \texttt{I}, \texttt{J} отождествляются), в которую заносятся буквы перемешанного алфавита, например в виде таблицы, представленной ниже. Часто перемешивание алфавита реализуется с помощью начального слова. В нашем примере начальное слово \texttt{playfir}. Таблица имеет вид

\begin{center}
    \begin{tabular}{ccccc}
        p & l & a & y & f  \\
        i & r & b & c & d  \\
        e & g & h & k & m  \\
        n & o & q & s & t  \\
        u & v & w & x & z  \\
    \end{tabular}
\end{center}

Буквы открытого текста разбиваются на пары. Правила шифрования каждой пары состоят в следующем.

\begin{itemize}
    \item Если буквы пары не лежат в одной строке или в одном столбце таблицы, то они заменяются буквами, образующими с исходными буквами вершины прямоугольника. Первой букве пары соответствует буква таблицы, находящаяся в том же столбце. Пара букв открытого текста \texttt{we} заменяется двумя буквами таблицы \texttt{hu}. Пара букв открытого текста \texttt{ew} заменяется двумя буквами таблицы \texttt{uh}.
    \item Если буквы пары открытого текста расположены в одной строке таблицы, то каждая буква заменяется соседней справа буквой таблицы. Например, пара \texttt{gk}  заменяется двумя буквами \texttt{hm}. Если одна из этих букв -- крайняя правая в таблице, то ее <<правым соседом>> считается крайняя левая в этой строке. Так, пара \texttt{to} заменяется буквами \texttt{nq}.
    \item Если буквы пары лежат в одном столбце, то каждая буква заменяется соседней буквой снизу. Например, пара \texttt{lo} заменяется парой \texttt{rv}. Если одна из этих букв крайняя нижняя, то ее <<нижним соседом>> считается крайняя верхняя буква в этом столбце таблицы. Например, пара \texttt{kx} заменяется буквами \texttt{sy};
    \item Если буквы в паре одинаковые, то между ними вставляется определённая буква, называемая <<буквой-пустышкой>>. После этого разбиение на пары производится заново.
\end{itemize}

\example
Используем шифр Плейфера\index{шифр!Плейфера} и зашифруем сообщение \texttt{Wheatstone was the inventor}. Исходное сообщение, разбитое на биграммы, показано в первой строке таблицы. Результат шифрования, так же разбитый на биграммы, приведён во второй строке.
\begin{center} \begin{tabular}{|*{12}c|}
    \hline
    wh & ea & ts & to & ne & wa & st & he & in & ve & nt & or \\
    \hline
    aq & ph & nt & nq & un & ab & tn & kg & eu & gu & on & vg \\
    \hline
\end{tabular} \end{center}
\exampleend

Шифр Плейфера\index{шифр!Плейфера} не является криптографически стойким. Несложно найти ключ, если известны пара открытого и соответствующего ему шифротекста. Если известен только шифротекст, криптоаналитик может проанализировать соответствие между частотой появления биграмм в шифротексте и известной частотой появления биграмм в языке, на котором написано сообщение. Такой частотный анализ помогает дешифрованию.


\section{Полиграммный шифр замены Хилла}\index{шифр!Хилла|(}
\selectlanguage{russian}

Если при шифровании преобразуются более двух букв открытого текста, то шифр называется \emph{полиграммным}\index{шифр!полиграммный}. Первый полиграммный шифр предложил Лестер Хилл в 1929 году (\langen{Lester Sanders Hill}, \cite{Hill:1929, Hill:1931}). Это был первый шифр, который позволял оперировать более чем с тремя символами за один такт.

В шифре Хилла текст предварительно преобразуют в цифровую форму и разбивают на последовательности (блоки) по $n$ последовательных цифр. Такие последовательности называются \emph{$n$-граммами}. Выбирают обратимую по модулю $m$  $(n \times n)$-матрицу $\mathbf{A} = (a_{ij})$, где  $m$ -- число букв в алфавите. Выбирают случайный $n$-вектор $\mathbf{f} = (f_1,  \dots, f_n)$. После чего  $n$-грамма открытого текста $\mathbf{x} = (x_1, x_2,  \dots, x_n)$ заменяется $n$-граммой шифрованного текста $\mathbf{y} = (y_1, y_2,  \dots, y_n)$ по формуле
    \[ \mathbf{y} = \mathbf{x} \mathbf{A} + \mathbf{f} \mod m. \]
Расшифрование проводится по правилу
    \[ \mathbf{x} = (\mathbf{y} - \mathbf{f}) \mathbf{A}^{-1} \mod m. \]

\example
Приведём пример шифрования с помощью шифра Хилла. Преобразуем английский алфавит в числовую форму (m = 26) следующим образом:
\[ \text{a} \rightarrow 0, ~ \text{b} \rightarrow 1, \text{c} \rightarrow 2, \dots, \text{z} \rightarrow 25. \]
%\[ \begin{array}{cccccccccccccc}
%    a & b & c & d & e & f & g & h & i & j &  k &  l &  m &  n \\
%    0 & 1 & 2 & 3 & 4 & 5 & 6 & 7 & 8 & 9 & 10 & 11 & 12 & 13
%\end{array} \]
%\[ \begin{array}{cccccccccccc}
%     o &  p &  q &  r &  s &  t &  u &  v &  w &  x &  y &  z  \\
%    14 & 15 & 16 & 17 & 18 & 19 & 20 & 21 & 22 & 23 & 24 & 25
%\end{array} \]

Выберем для примера $n=2$. Запишем фразу <<Wheatstone was an inventor>> из предыдущего примера (первая строка таблицы). Каждой букве поставим в соответствие ее номер в алфавите (вторая строка):
\begin{center} \resizebox{\textwidth}{!}{ \begin{tabular}{|*{12}c|}
    \hline
    w,h & e,a & t,s & t,o & n,e & w,a & s,t & h,e & i,n & v,e & n,t & o,r \\
    \hline
    22,7 & 4,0 & 19,18 & 19,14 & 13,4 & 22,0 & 18,19 & 7,4 & 8,13 & 21,4 & 13,19 & 14,17 \\
    \hline
\end{tabular} } \end{center}

Выберем матрицу шифрования $A$ в виде
\[
    \mathbf{A} = \left( \begin{array}{cc}
        5 & 8 \\
        3 & 5 \\
    \end{array} \right).
\]

Эта матрица обратима по $\mod 26$, так как ее определитель равен $1$ и взаимно прост с числом букв английского алфавита $m=26$. Обратная матрица равна
\[
    \mathbf{A}^{-1} = \left( \begin{array}{cc}
        5  & 18 \\
        23 & 5
    \end{array} \right) \mod 26.
\]

Выберем вектор $\mathbf{f} = (4, 2)$. Первая числовая пара открытого текста  $\mathbf{x} = (\text{w}, \text{h}) = (22, 7)$  зашифрована в виде
\[
    \mathbf{y} = \mathbf{x} \mathbf{A} + \mathbf{f} =
        (22, 7)
        \left( \begin{array}{cc}
            5 & 8 \\
            3 & 5
        \end{array} \right) +
        (4, 2) = (14, 3) \mod 26
\]
или в буквенном виде  $(\text{o}, \text{d})$.

Повторяя вычисления для всех пар, получим полный шифрованный текст в числовом виде (третья строка) или в буквенном виде (четвертая строка):
\begin{center} \resizebox{\textwidth}{!}{ \begin{tabular}{|*{12}c|}
    \hline
    w,h & e,a & t,s & t,o & n,e & w,a & s,t & h,e & i,n & v,e & n,t & o,r \\
    22,7 & 4,0 & 19,18 & 19,14 & 13,4 & 22,0 & 18,19 & 7,4 & 8,13 & 21,4 & 13,19 & 14,17 \\
    \hline
    14,3 & 24,22 & 9,21 & 3,9 & 23,1 & 10,8 & 12,19 & 19,23 & 18,3 & 11,15 & 13,20 & 2,19 \\
    o,d & y,w & j,v & d,j & x,b & k,i & m,t & t,x & s,d & l,p & n,u & c,t \\
    \hline
\end{tabular} } \end{center}
\exampleend

Криптосистема Хилла уязвима к частотному криптоанализу\index{криптоанализ!частотный}, который основан на вычислении частот последовательностей символов. Рассмотрим пример взлома простого варианта криптосистемы Хилла.

\example В английском языке $m = 26$,
    \[ a \rightarrow 0, ~ b \rightarrow 1, ~ \dots, ~ z \rightarrow 25. \]
При шифровании использована криптосистема Хилла с матрицей второго порядка c нулевым вектором $\mathbf{f}$. Наиболее часто встречающиеся в шифротексте биграммы -- RH и NI, в то время как в исходном языке они TH и HE (артикль THE). Найдём матрицу секретного ключа, составив уравнения
\[
    \begin{array}{l}
        R = 17 = -9 \mod 26, ~~ H = 7 \mod 26, ~~ N = 13 \mod 26, \\
        I = 8 \mod 26, ~~ T = 19 = -7 \mod 26, ~~ E=4 \mod 26; \\
    \end{array}
\] \[
    \left( \begin{array}{cc}
        \text{R} & \text{H} \\
        \text{N} & \text{I}
    \end{array} \right) =
    \left( \begin{array}{cc}
        \text{T} & \text{H} \\
        \text{H} & \text{E}
    \end{array} \right) \cdot
    \left( \begin{array}{cc}
        k_{1,1} & k_{1,2} \\
        k_{2,1} & k_{2,2}
    \end{array} \right) \mod 26;
\] \[
    \left( \begin{array}{cc}
        -9 & 7 \\
        13 & 8
    \end{array} \right) =
    \left( \begin{array}{cc}
        -7 & 7 \\
        7 & 4
    \end{array} \right) \cdot
    \left( \begin{array}{cc}
        k_{1,1} & k_{1,2} \\
        k_{2,1} & k_{2,2}
    \end{array} \right) \mod 26;
\]

Стоит обратить внимание на то, что числа 4, 8, 13 не имеют обратных по модулю 26.

\[
    D = \det \left( \begin{array}{cc} -7 & 7 \\ 7 & 4 \end{array} \right) = -7 \cdot 4 - 7 \cdot 7 = 1 \mod 26.
\] \[
    \left( \begin{array}{cc} -7 & 7 \\ 7 & 4 \end{array} \right)^{-1} =
    D^{-1} \left( \begin{array}{cc} 4 & -7 \\ -7 & -7 \end{array} \right) =
    \left( \begin{array}{cc} 4 & -7 \\ -7 & -7 \end{array} \right) \mod 26.
\] \[
    \left( \begin{array}{cc} k_{1,1} & k_{1,2} \\ k_{2,1} & k_{2,2} \end{array} \right) =
    \left( \begin{array}{cc} 4 & -7 \\ -7 & -7 \end{array} \right) \cdot
    \left( \begin{array}{cc} -9 & 7 \\ 13 & 8 \end{array} \right) =
\] \[
    = \left( \begin{array}{cc} 3 & -2 \\ -2 & -1 \end{array} \right) \mod 26.
\]
Найденный секретный ключ
\[
    \left( \begin{array}{cc} \text{D} & \text{Y} \\ \text{Y} & \text{Z} \end{array} \right).
\]
\exampleend

\index{шифр!Хилла|)}


% \subsection{Омофонные замены}
%
% Омофонными заменами называют криптопримитивы, в основе которых лежит замена групп символов открытого текста $M$ на группу символов $C$ с использованием ключа $K$. Такой метод шифрования вносит неоднозначность между $M$ и $C$, это позволяет защититься от методов частотного криптоанализа.
%  \subsection{шифрокоды}
%  Шифрокоды -- это класс шифров сочетающих в себе свойства кодов и помехозащищённости со свойствами шифра и обеспечения конфиденциальности.

\section{Шифр гаммирования Виженера}
\selectlanguage{russian}

Шифр, который известен под именем Виженера, впервые описал Джованни Батиста Беллазо (Giovan Battista Bellasо) в своей книге <<La cifra>>.

Рассмотрим один из вариантов этого шифра. В самом простом случае квадратом \textbf{Виженера}  называется таблица из циклически сдвинутых копий латинского алфавита, в которой буквы J и V исключены. Первая строка и первый столбец -- буквы латинского алфавита в их обычном порядке. В строках таблицы порядок букв сохраняется, за исключением циклических переносов. Представим эту таблицу:

\begin{center} \resizebox{\textwidth}{!}{ \begin{tabular}{|c|*{24}c|}
    \hline
    $\quad \downarrow ~ \rightarrow$ & \textbf{A} & \textbf{B} & \textbf{C} & \textbf{D} & \textbf{E} & \textbf{F} & \textbf{G} & \textbf{H} & \textbf{I} & \textbf{K} & \textbf{L} & \textbf{M} & \textbf{N} & \textbf{O} & \textbf{P} & \textbf{Q} & \textbf{R} & \textbf{S} & \textbf{T} & \textbf{U} & \textbf{X} & \textbf{Y} & \textbf{Z} & \textbf{W} \\
    \hline
    \textbf{A} & A & B & C & D & E & F & G & H & I & K & L & M & N & O & P & Q & R & S & T & U & X & Y & Z & W \\
    \textbf{B} & B & C & D & E & F & G & H & I & K & L & M & N & O & P & Q & R & S & T & U & X & Y & Z & W & A \\
    \textbf{C} & C & D & E & F & G & H & I & K & L & M & N & O & P & Q & R & S & T & U & X & Y & Z & W & A & B \\
    \textbf{D} & D & E & F & G & H & I & K & L & M & N & O & P & Q & R & S & T & U & X & Y & Z & W & A & B & C \\
    \textbf{E} & E & F & G & H & I & K & L & M & N & O & P & Q & R & S & T & U & X & Y & Z & W & A & B & C & D \\
    \textbf{F} & F & G & H & I & K & L & M & N & O & P & Q & R & S & T & U & X & Y & Z & W & A & B & C & D & E \\
    \textbf{G} & G & H & I & K & L & M & N & O & P & Q & R & S & T & U & X & Y & Z & W & A & B & C & D & E & F \\
    \textbf{H} & H & I & K & L & M & N & O & P & Q & R & S & T & U & X & Y & Z & W & A & B & C & D & E & F & G \\
    \textbf{I} & I & K & L & M & N & O & P & Q & R & S & T & U & X & Y & Z & W & A & B & C & D & E & F & G & H \\
    \textbf{K} & K & L & M & N & O & P & Q & R & S & T & U & X & Y & Z & W & A & B & C & D & E & F & G & H & I \\
    \textbf{L} & L & M & N & O & P & Q & R & S & T & U & X & Y & Z & W & A & B & C & D & E & F & G & H & I & K \\
    \textbf{M} & M & N & O & P & Q & R & S & T & U & X & Y & Z & W & A & B & C & D & E & F & G & H & I & K & L \\
    \textbf{N} & N & O & P & Q & R & S & T & U & X & Y & Z & W & A & B & C & D & E & F & G & H & I & K & L & M \\
    \textbf{O} & O & P & Q & R & S & T & U & X & Y & Z & W & A & B & C & D & E & F & G & H & I & K & L & M & N \\
    \textbf{P} & P & Q & R & S & T & U & X & Y & Z & W & A & B & C & D & E & F & G & H & I & K & L & M & N & O \\
    \textbf{Q} & Q & R & S & T & U & X & Y & Z & W & A & B & C & D & E & F & G & H & I & K & L & M & N & O & P \\
    \textbf{R} & R & S & T & U & X & Y & Z & W & A & B & C & D & E & F & G & H & I & K & L & M & N & O & P & Q \\
    \textbf{S} & S & T & U & X & Y & Z & W & A & B & C & D & E & F & G & H & I & K & L & M & N & O & P & Q & R \\
    \textbf{T} & T & U & X & Y & Z & W & A & B & C & D & E & F & G & H & I & K & L & M & N & O & P & Q & R & S \\
    \textbf{U} & U & X & Y & Z & W & A & B & C & D & E & F & G & H & I & K & L & M & N & O & P & Q & R & S & T \\
    \textbf{X} & X & Y & Z & W & A & B & C & D & E & F & G & H & I & K & L & M & N & O & P & Q & R & S & T & U \\
    \textbf{Y} & Y & Z & W & A & B & C & D & E & F & G & H & I & K & L & M & N & O & P & Q & R & S & T & U & X \\
    \textbf{Z} & Z & W & A & B & C & D & E & F & G & H & I & K & L & M & N & O & P & Q & R & S & T & U & X & Y \\
    \textbf{W} & W & A & B & C & D & E & F & G & H & I & K & L & M & N & O & P & Q & R & S & T & U & X & Y & Z \\
    \hline
\end{tabular} } \end{center}

Здесь первый столбец используется для ключевой последовательности, а первая строка -- для открытого текста. Общая схема шифрования такова: выбирается некоторая ключевая последовательность, которая периодически повторяется в виде длинной строки. Под ней соответственно каждой букве записываются буквы открытого текста в виде второй строки. Буква ключевой последовательности указывает строку в квадрате Виженера, буква открытого текста указывает столбец в квадрате. Соответствующая буква, стоящая в квадрате на пересечении строки и столбца, заменяет букву открытого текста в шифротексте. Приведём примеры.

\example
Ключевая последовательность состоит из периодически повторяющегося ключевого слова, известного обеим сторонам. Пусть ключевая последовательность состоит из периодически повторяющегося слова THIS, а открытый текст -- слова COMMUNICATIONSYSTEMS (см. таблицу). Пробелы между словами опущены.
\begin{center} \resizebox{\textwidth}{!}{ \begin{tabular}{|l|*{20}c|}
    \hline
    Ключ            & T & H & I & S & T & H & I & S & T & H & I & S & T & H & I & S & T & H & I & S \\
    Открытый текст  & C & O & M & M & U & N & I & C & A & T & I & O & N & S & Y & S & T & E & M & S \\
    Шифротекст      & X & X & U & E & O & U & R & U & T & B & R & G & G & A & F & L & N & M & U & L \\
    \hline
\end{tabular} } \end{center}
Результат шифрования приведён в третьей строке: на пересечении строки $T$ и столбца $C$ стоит буква $X$, на пересечении строки $H$ и столбца $O$ стоит буква $X$, на пересечении строки $I$ и столбца  $M$ стоит буква $U$ и т.~д.
\exampleend

Виженер считал возможным в качестве ключевой последовательности использовать открытый текст с добавлением начальной буквы, известной легальным пользователям. Этот вариант используется во втором примере.

\example
Ключевая последовательность образуется с помощью открытого текста. Стороны договариваются о первой букве ключа, а следующие буквы состоят из открытого текста. Пусть в качестве первой буквы выбрана буква  $T$. Тогда для предыдущего примера таблица шифрования имеет вид:
\begin{center} \resizebox{\textwidth}{!}{ \begin{tabular}{|l|*{20}c|}
    \hline
    Ключ            & T & C & O & M & M & U & N & I & C & A & T & I & O & N & S & Y & S & T & E & M \\
    Открытый текст  & C & O & M & M & U & N & I & C & A & T & I & O & N & S & Y & S & T & E & M & S \\
    Шифротекст      & X & Q & A & Z & G & H & X & L & C & T & C & Y & B & F & P & P & M & Z & Q & E \\
    \hline
\end{tabular} } \end{center}
\exampleend

\example
Пусть ключевая последовательность образуется с помощью шифротекста. Стороны договариваются о первой букве ключа. В отличие от предыдущего случая, следующая буква ключа -- это результат
шифрования первой буквы текста и т.~д. Пусть в качестве первой буквы выбрана буква  $T$. Тогда приведенная в предыдущем примере таблица шифрования примет такой вид:
\begin{center} \resizebox{\textwidth}{!}{ \begin{tabular}{|l|*{20}c|}
    \hline
    Ключ            & T & X & K & X & H & C & P & Z & A & A & T & C & Q & D & X & S & L & E & I & U \\
    Открытый текст  & C & O & M & M & U & N & I & C & A & T & I & O & N & S & Y & S & T & E & M & S \\
    Шифротекст      & X & K & X & H & C & P & Z & A & A & T & C & Q & D & X & S & L & E & I & U & N \\
    \hline
\end{tabular} } \end{center}
\exampleend


\section[Криптоанализ полиалфавитных шифров]{Криптоанализ полиалфавитных \protect\\ шифров}
\selectlanguage{russian}

При дешифровании полиалфавитных шифров криптоаналитику надо сначала определить период, затем преобразовать шифрограмму в матрицу для предполагаемого периода и использовать для каждого столбца методы криптоанализа моноалфавитных шифров. При неудаче надо изменить предполагаемый период.

Известно несколько методов криптоанализа для нахождения периода. Из них наиболее популярными являются метод Касиски, автокорреляционный метод и метод индекса совпадений.


\subsection{Метод Касиски}

Метод Касиски (\langde{Friedrich Wilhelm Kasiski}, 1805--1881, \cite{Kasiski:1863}) состоит в том, что в шифротексте находят одинаковые сегменты длины не менее трёх символов и вычисляют расстояние между начальными символами последовательных сегментов. Далее находят наибольший общий делитель этих расстояний. Считается, что предполагаемый период $n$ является кратным этому значению. Обычно, нахождение периода осуществляется в несколько этапов.

После того как выбирается наиболее правдоподобное значение периода, криптоаналитик переходит к дешифрованию. Приведём пример использования метода Касиски.

\example
Пусть шифруется следующий текст без учета знаков препинания и различия строчных и прописных букв. Пробелы оставлены в тексте для удобства чтения, хотя при шифровании пробелы были опущены.

\begin{quote}
    \noindent \texttt{Игры различаются по содержанию, характерным особенностям, а также по тому какое место они занимают в жизни детей их воспитании и обучении Каждый отдельный вид игры имеет многочисленные варианты Дети очень изобретательны Они усложняют и упрощают известные игры придумывают новые правила и детали Например сюжетно ролевые игры создаются самими детьми но при некотором руководстве воспитателя Их основой является самодеятельность Такие игры иногда называют творческими сюжетно ролевыми играми Разновидностью сюжетно ролевой игры являются строительные игры и игры драматизации В практике воспитания нашли свое место и игры с правилами которые создаются для детей взрослыми К ним относятся дидактические подвижные и игры забавы В основе их лежит четко определённое программное содержание дидактические задачи и целенаправленное обучение Для хорошо организованной жизни детей в детском саду необходимо разнообразие игр так как только при этих условиях будет обеспечена детям возможность интересной и содержательной деятельности Многообразие типов видов форм игр неизбежно как неизбежно многообразие жизни которую они отражают как неизбежно многообразие несмотря на внешнюю схожесть игр одного типа модели}
\end{quote}

Для шифрования выберем период $n=4$ и следующие 4 моноалфавитных шифра замены.

\begin{center} \begin{tabular}{|lcl|}
    \hline
    абвгдежзийклмнопрстуфхцчшщъыьэюя & -- & алфавит \\
    йклмнопрстуфхцчшщъыьэюяабвгдежзи & -- & 1-й шифр \\
    гаэъчфсолиевяьщцурнкздбюышхтпмйж & -- & 2-й шифр \\
    бфзънаужщмятешлюсдчкэргцйьпвхиыо & -- & 3-й шифр \\
    пъерыжсьзтэиуюйфякхалцбмчвншгощд & -- & 4-й шифр \\
    \hline
\end{tabular} \end{center}

Тогда шифрованный текст примет следующий вид (в шифротексте пробелов нет, они вставлены для удобства чтения).

\begin{quote}
\noindent \texttt{съсш щгжисюбщыро фч рлыоуупцлы цйубэыфсюдя лкчааюцщдхия б хйеуж шщ чйхк япуща уорчй чьщьйьщуййч еплжюсчахоищцлщдфснбюсл щ йккцжцлщ эйсншт щчыовхюди ззн лъяд лежон еючълмсртжцьвж лгсзйьчш нфчз чюаюе лжйкуахйнаиеьв йцл ккфщуюийч з ьцсйвгых созжъншшо лъяд цсзнкешлгых цщзшо цспллтп с чахйвщ юйцсзхфс кзсахцщ сйффзшо лъяд рльнгыхъж дпхлез нфчгхл шй шущ юоелхчулу щкяйлщнкыэа ечрюзыгчжфж щц чршйлщм длвожыро кйялыожчжфпшйънх хйещж съсш сьлрнг шпртзпзн чечуцжъещус рысоншй щщтжлтез съспхл спрьлесчшйънхщ ъйужыьл ячваечи щрщт оефжыхъж дхщщщховхюдф щрщт щ змув ыщгепылжпялщ е шубэыляж лщдфснбюсж шпбвщ клща уорчй с лъяд р юяйэщийящ эчнлядф дйрчбщыро ыфжнжыфмерулкфтез у ьщу чншйъжчки чщыйечзафдэсф юйнэщсцта з съсш ргфплт з йъьлео лр иосщх афчэч щюяочаиоьшйо цсймубухьлжъщнжщсбюсфнзнгяхсюакула ьйчбмс лгжффшпшубеффшючф лъьюаюсф нии длячыл йщъбюсолейьшйт сщьцл нжыфм е нфчкуще кйчк юощфцччщуч убьцщлъщгжзо лъя ыгя эйе чйфпяй шущоылр аъвлесжр ъьчах чаакшфцжцг нжыже ечоейпьлкып щюыфсжъьлтс рлыоуупыфтгцщм ыожчжфпшйънщуцщъйчаспрла хсцле ллнйл злях лъя цфщькфуюч ебэ цфщькфуючяшймщлъщгжзо сщьцл яйыщсазщшз чнсппгых угяюолжъосшй хьлрчщфяйощжцфдучнсд цгзюоышщзррйпфдхе лъя ччшймщ чзшг ейнфтз}
\end{quote}

Теперь проведём криптоанализ, используя метод Касиски. Предварительно подсчитаем число появлений каждой буквы в шифротексте. Эти данные приведём в таблице, где $i$ в первой строке означает букву алфавита, а $f_{i}$ во второй строке  -- это число появлений этой буквы в шифротексте. Всего в нашем шифротексте имеется $L=1036$ букв.

\begin{center} \resizebox{\textwidth}{!}{ \begin{tabular}{|c||c|c|c|c|c|c|c|c|c|c|c|c|c|c|c|c|}
    \hline
    $i$     & А & Б & В & Г & Д & Е & Ж & З & И & Й & К & Л & М & Н & О & П \\
    $f_{i}$ & 26 & 15 & 11 & 21 & 20 & 36 & 42 & 31 & 13 & 56 & 23 & 70 & 10 & 33 & 36 & 25 \\
    \hline
\end{tabular} } \end{center}

\begin{center} \resizebox{\textwidth}{!}{ \begin{tabular}{|c||c|c|c|c|c|c|c|c|c|c|c|c|c|c|c|c|}
    \hline
    $i$     & Р & С & Т & У & Ф & Х & Ц & Ч & Ш & Щ & Ъ & Ы & Ь & Э & Ю & Я \\
    $f_{i}$ & 28 & 54 & 15 & 36 & 45 & 32 & 31 & 57 & 35 & 72 & 32 & 35 & 27 & 11 & 30 & 28 \\
    \hline
\end{tabular} } \end{center}

В рассматриваемом примере проведенный анализ показал следующее.
\begin{itemize}
    \item Сегмент СЪС встречается в позициях $1, 373, 417, 613$. Соответствующие расстояния равны
        \[ \begin{array}{l}
            373 - 1 = 372 = 4 \cdot 3 \cdot 31, \\
            417 - 373= 44 = 4 \cdot 11, \\
            613 - 417 = 196 = 4 \cdot 49. \\
        \end{array} \]
        Наибольший общий делитель равен $4$. Делаем вывод, что период кратен $4$.
    \item Сегмент ЩГЖ встречается в позициях $5, 781, 941$. Соответствующие расстояния равны
        \[ \begin{array}{l}
            781 - 5 = 776 = 8 \cdot 97, \\
            941 - 781 = 160 = 32 \cdot 5. \\
        \end{array} \]
        Делаем вывод, что период кратен $8$, что не противоречит выводу для предыдущих сегментов (кратность $4$).
    \item Сегмент ЫРО встречается в позициях $13, 349, 557$. Соответствующие расстояния равны
        \[ \begin{array}{l}
            349 - 13 = 336 = 16 \cdot 3 \cdot 7, \\
            557 - 349 = 208 = 16 \cdot 13. \\
        \end{array} \]
        Делаем вывод, что период кратен 4.
\end{itemize}

Предположение о том, что период $n=4$, оказалось правильным.
\exampleend


\subsection{Автокорреляционный метод}

Автокорреляционный метод состоит в том, что исходный шифротекст $C_{1},C_{2},  \ldots, C_{L}$ выписывается в строку, а под ней выписываются строки, полученные сдвигом вправо на $t =1, 2, 3, \ldots$ позиций. Для каждого $t$ подсчитывается число  $n_{t}$ индексов $i \in \left[ {1,L - t} \right]$, таких, что $C_i  = C_{i + t}$.

Вычисляются автокорреляционные коэффициенты
    \[ \gamma_t  = \frac{n_t}{L - t}. \]
Для сдвигов $t$, кратных периоду, коэффициенты должны быть заметно больше, чем для $t$, не кратных периоду.

\example
Для рассматриваемой криптограммы выделим те значения $t$, для которых $\gamma _t  > 0,05.$ Получим ряд чисел:

\begin{quote}
    \noindent \texttt{4, 12, 16, 24, 28, 32, 36, 40, 44, 48, 52, 56, 64, 68, 72, 76, 80, 84, 88, 92, 96, 104, 108, 112, 116, 124, 128, 132, 140, 148, 152, 156, 160, 164, 168, 172, 176, 180, 184, 188, 192, 196, 200, 204, 208, 216, 220, 224, 228, 252, 256, 260, 264, 268, 272, 276, 280, 284, 288, 292, 296, 300, 304, 308, 312, 316, 320, 324, 328, 344, 348, 356, 364, 368, 372, 376, 380, 384, 388, 396, 400, 404, 408, 412, 420, 424, 432, 436, 440, 448, 452, 456, 460, 462, 468, 472, 476, 480, 484, 496, 500, 508, 512, 516.}
\end{quote}

Все эти числа, кроме 462, делятся на 4. Выбираем значение $n=4$, которое верно и совпадает со значением, полученным по методу Касиски.
\exampleend


\subsection{Метод индекса совпадений}

Метод индекса совпадений был описан Фридманом в 1922 году (\langen{William Frederick Friedman}, 1891--1969, \cite{Friedman:1922}). При применении метода индекса совпадений подсчитывают число появлений букв в случайной последовательности
    \[ \mathbf{X} = (X_1 ,X_2 ,  \ldots , X_L ) \]
и вычисляют вероятность того, что два случайных элемента этой последовательности совпадают. Эта величина называется \textbf{индексом совпадений} и обозначается $I_{c}(\mathbf{x}),$ где
    \[ I_{c} (\mathbf{x}) = \frac{{\sum\limits_{i = 1}^A {f_i (f_i  - 1)} }} {{L(L - 1)}}, \]
$f_{i}$ -- число появлений буквы $i$  в последовательности $\mathbf{x}$, $A$ -- число букв в алфавите.

Значение этого индекса используется в криптоанализе полиалфавитных шифров для приближенного определения периода по формуле
    \[ m \approx \frac{{k_p  - k_r }} {{I_{c} (\mathbf{x}) - k_r  + \frac{{k_p  - I_{c} (\mathbf{x})}} {L}}}, \]
где
    \[ k_r  = \frac{1}{A}, ~~ k_p  = \sum\limits_{i=1}^A p_i^2, \]
$p_i $ -- частота появления буквы $i$ в естественном языке.
Теоретическое обоснование метода индекса совпадений не является простым. Оно приведено в Приложении~\ref{chap:coincide-index} к данному пособию.

\example
В рассматриваемом выше примере приведены значения $f_{i}$. Для русского языка
    \[ A=32, ~ k_{r} = \frac{1}{32} \approx 0.03125, ~ k_{p} \approx 0.0529. \]
Проведя вычисления, получаем $m \approx 3.376$. Так что полученное по формуле приближенное значение $3.376$ достаточно близко к значению периода $n=4$.
\exampleend

С развитием ЭВМ многие классические полиалфавитные шифры перестали быть устойчивыми к криптоатакам.


\chapter{Совершенная криптостойкость}\index{криптостойкость!совершенная|(}
\selectlanguage{russian}

Рассмотрим модель криптосистемы, в которой Алиса выступает источником сообщений $m \in \group{M}$. Алиса использует некоторую функцию шифрования, результатом вычисления которой является шифротекст $c \in \group{C}$:

	\[c = E_{K_1}\left(m\right).\]

Шифротекст $c$ передаётся по открытому каналу легальному пользователю Бобу, причём по пути он может быть перехвачен нелегальным пользователем (криптоаналитиком) Евой.

Боб, обладая ключом расшифрования $K_2$, расшифровывает сообщение с использованием функции расшифрования:
	\[m' = D_{K_2}\left(c \right).\]

Рассмотрим теперь исходное сообщение, передаваемый шифротекст и ключи шифрования (и расшифрования, если они отличаются) в качестве случайных величин, описывая их свойства с точки зрения теории информации. Далее полагаем, что в криптосистеме ключи шифрования и расшифрования совпадают.

Будем называть криптосистему \emph{корректной}, если она обладает следующими свойствами.
\begin{itemize}
	\item Легальный пользователь имеет возможность однозначно восстановить исходное сообщение, то есть
					\[H \left( M | C, K \right) = 0, \]
					\[m' = m.\]
	\item Выбор ключа шифрования не зависит от исходного сообщения
					\[ I \left( K ; M \right) = 0, \]
					\[ H \left( K | M \right) = H \left( K \right). \]
\end{itemize}

Второе свойство является в некотором виде условием на возможность отделить ключ шифрования от данных и алгоритма шифрования.

\section[Определения]{Определения совершенной криптостойкости}

Понятие совершенной секретности (или стойкости) введено американским ученым Клодом Шенноном. В конце Второй мировой войны он закончил работу, посвященную теории связи в секретных системах\cite{Shannon:1949:CTS}. Эта работа вошла составной частью в собрание его трудов, вышедшее в русском переводе в 1963 году.~\cite{Shannon:1963} Понятие о стойкости шифров по Шеннону связано с решением задачи криптоанализа по одной криптограмме.

Криптосистемы совершенной стойкости могут применяться как в современных вычислительных сетях, так и для шифрования любой бумажной корреспонденции. Основной проблемой применения данных шифров для шифрования больших объёмов информации является необходимость распространения ключей объёмом не меньшим, чем передаваемые сообщения.

\begin{definition}\label{perfect_by_probabilities}
Будем называть криптосистему \emph{совершенно криптостойкой}, если апостериорное распределение вероятностей исходного случайного сообщения $m_i \in \group{M}$ при регистрации случайного шифротекста $c_k \in \group{C}$ совпадает с априорным распределением~\cite{Gultyaeva:2010}:

	\[\forall m_j \in \group{M}, c_k \in \group{C}: P \left( m = m_j | c = c_k \right) = P \left( m = m_j \right).\]
\end{definition}

Данное условие можно переформулировать в терминах статистических свойств сообщения, ключа и шифротекста как случайных величин:

\begin{definition}\label{perfect_by_enthropy}
Будем называть криптосистему \emph{совершенно криптостойкой}, если условная энтропия сообщения\index{энтропия!условная}\index{энтропия!открытого текста} при известном шифротексте равна безусловной:
	\[H \left( M | C \right) = H \left( M \right),\]
	\[I \left( M; C \right) = 0.\]
\end{definition}

Можно показать, что определения~\ref{perfect_by_probabilities} и~\ref{perfect_by_enthropy} тождественны.

\section[Условие]{Условие совершенной криптостойкости}

Найдём оценку количества информации, которое содержит шифротекст $C$ относительно сообщения $M$
\[ I(M; C) = H(M) - H(M | C). \]
Очевидны следующие соотношения условных и безусловных энтропий~\cite{GabPil:2007}:
\[H(K|C)=H(K|C)+H(M|K,C)=H(M,K|C),\]
\[H(M,K|C)=H(M|C)+H(K|M,C)\geq H(M|C),\]
\[H(K)\geq H(K|C)\geq H(M|C).\]
Отсюда получаем:
 \[ I(M; C) = H(M) - H(M | C)\geq H(M)-H(K). \]
Из последнего неравенства следует, что взаимная информация между сообщением и шифротекстом равна нулю, если энтропия ключа не меньше энтропии сообщений. С другой стороны, взаимная информация между сообщением и шифротекстом равна нулю, если они статистически независимы. Таким образом, условием совершенной криптостойкости является неравенство
\[ H(M) \leq H(K).\]
%Если утверждение верно, то количество информации в шифротексте относительно открытого текста $I(M; C)$ равно нулю:
%  \[ I(M; C) = H(M) - H(M | C) = 0, \]
%так как для статистически независимых величин условная энтропия равна безусловной энтропии, то есть $H(M) = H(M | C)$.

%Функцию шифрования обозначим $E: \{ M, K \} \rightarrow C$. Процедура шифрования состоит из следующих шагов.
%\begin{itemize}
%    \item Легальный пользователь $A$ выбирает ключ $k \in K$ и секретно сообщает его легальному пользователю $B$ (дополнительная задача -- распределение ключей).
%    \item По открытому сообщению $m \in M$ и выбранному ключу $k$ вычисляют шифрованное сообщение $c = E_k(m) \in C$.
%\end{itemize}

%Основное требование при шифровании состоит в том, чтобы при выбранном ключе $k$ вычисление $c$  было легкой задачей для любого сообщения $m$.

%Функцию расшифрования обозначим $D: \{ C, K \} \rightarrow M$. Процедура расшифрования состоит из следующих шагов.
%\begin{itemize}
%    \item Легальный пользователь $B$ получает от $A$ секретный ключ $k \in K$.
 %   \item $B$ по принятому шифрованному сообщению $c \in C$ и известному ключу $k$ вычисляет открытое сообщение $m = D_k(c) \in M$.
%\end{itemize}

%Основное требование: при выбранном ключе $k$ вычисление $m$ должно быть легкой задачей для любого $c$. С другой стороны, при неизвестном ключе $k$ вычисление открытого сообщения $m$ по известному шифрованному сообщению $c$ должно быть трудной задачей для любого $c$.

%Криптостойкость шифра оценивается числом операций, необходимым для определения: открытого текста $m$ по шифротексту $c$, либо ключа шифрования $k$ по открытому тексту $m$ и шифротексту $c$.

%$M, C, K$ интерпретируются как случайные величины.
%Пусть заданы распределения вероятностей $P_m(M), P_c(C), P_k(K)$. По определению шифрование $C = E_K(M)$ -- детерминированная функция своих аргументов.
%Если при выбранном шифре оказалось, что открытый текст $M$ и шифротекст $C$ -- статистически независимые случайные величины, то считается, что такая система обладает совершенной криптостойкостью.


%\subsection{Длина ключа}

%Пусть сообщения $m\in M$ и ключи $r\in K$ являются независимыми случайными величинами. Это значит, что их совместная вероятность $P_{mk}(M, K)$ равна произведению отдельных вероятностей:
%\[P_{mk}(M, K) = P_m(M) \cdot P_k(K).\]
%Пусть $C = E_K(M)$ -- множество шифрованных текстов, $M = D_K(C)$ -- множество расшифрованных текстов. Можно найти вероятности $P_c(C), P_{mck}(M,C,K)$.

%Используя известные соотношения о безусловной и условной энтропии~\cite{GabPil:2007}, оценим энтропию открытых текстов $M$ с учетом статистической независимости $M$ и $C$:
 %   \[ H(M) = H(M | C) \leq H(MK | C) = H(K | C) + H(M | CK) = \]     \[ = H(K | C) \leq H(K). \]

%Так как энтропия открытого текста при заданном шифротексте и известном ключе равна нулю, то $H(M|CK)=0$. В результате получаем     \[ H(M) \leq H(K). \]

Обозначим длины сообщений и ключа как $L(M)$ и $L(K)$ соответственно. Известно, что $H(M) \leq L(M)$~\cite{GabPil:2007}. Равенство $H(M) = L(M)$ достигается, когда сообщения состоят из статистически независимых и равновероятных символов. Такое же свойство выполняется и для случайных ключей $H(K) \leq L(K)$. Таким образом, достаточным условием совершенной криптостойкости системы можно считать неравенство
 \[ L(M) \leq L(K)\]
при случайном выборе ключа.

%С другой стороны, энтропия открытого текста $H(M)$ характеризует длину последовательности для описания случайной величины $M$ (открытого сообщения), а $H(K)$ характеризует длину последовательности для описания ключа. Следовательно, совершенная криптостойкость возможна только тогда, когда длина ключа не меньше, чем длина шифруемого сообщения, то есть     \[ H(M) \leq H(K). \] Как правило, длина сообщения заранее неизвестна и ограничена большим числом. Выбрать ключ длины не меньшей, чем возможное сообщение не представляется возможным или рациональным, и один и тот же ключ (или его преобразования) используется многократно для шифрования блоков сообщения фиксированной длины. То есть, $H(K) \ll H(M)$.

На самом деле сообщение может иметь произвольную (заранее не ограниченную) длину. Поэтому генерация и главным образом доставка легальным пользователям случайного и достаточно длинного ключа становятся критическими проблемами. Практическим решением этих проблем является многократное использование одного и того же ключа при условии, что его длина гарантирует вычислительную невозможность любой известной атаки на подбор ключа.

\index{криптостойкость!совершенная|)}

\section{Криптосистема Вернама}\index{криптосистема!Вернама|(}

Приведём пример системы с совершенной криптостойкостью.

Пусть сообщение представлено двоичной последовательностью длины $N$:
    \[ m = (m_1, m_2, \dots, m_N). \]
Распределение вероятностей сообщений $P_m(m)$ может быть любым. Ключ также представлен двоичной последовательностью $ k = (k_1, k_2, \dots, k_N)$ той же длины, но с равномерным распределением
    \[ P_k(k) = \frac{1}{2^N} \]
для всех ключей.

Шифрование в криптосистеме Вернама осуществляется путем покомпонентного суммирования по модулю алфавита последовательностей открытого текста и ключа:
    \[ C = M \oplus K = (m_1 \oplus k_1, ~ m_2 \oplus k_2, \dots,  m_N \oplus k_N). \]

Легальный пользователь знает ключ и осуществляет расшифрование:
    \[ M =C \oplus K = (m_1 \oplus k_1, ~ m_2 \oplus k_2, \dots, m_N \oplus k_N). \]

Найдём вероятностное распределение $N$-блоков шифротекстов, используя формулу
    \[ P(c = a) = P(m \oplus k = a) = \sum_{m} P(m) P(m \oplus k = a | m) = \]
    \[ = \sum_{m} P(m) P(k \oplus m) = \sum_{m} P(m) \frac{1}{2^N} = \frac{1}{2^N}. \]

Получили подтверждение известного факта: сумма двух случайных величин, одна из которых имеет равномерное распределение, является случайной величиной с равномерным распределением. В нашем случае распределение ключей равномерное, поэтому распределение шифротекстов тоже равномерное.

Запишем совместное распределение открытых текстов и шифротекстов:
    \[ P(m = a, c = b) ~=~ P(m = a) ~ P(c = b | m = a). \]

Найдём условное распределение
    \[ P(c = b | m = a) ~=~ P(m \oplus k = b | m = a) ~= \]
    \[ =~ P(k = b \oplus a | m = a) ~=~ P(k = b \oplus a) ~=~ \frac{1}{2^N}, \]
так как ключ и открытый текст являются независимыми случайными величинами. Итого:
    \[ P(c=b | m=a) = \frac{1}{2^N}. \]

Подстановка правой части этой формулы в формулу для совместного распределения даёт
    \[ P(m=a,c=b)=P(m=a)\frac{1}{2^N}, \]
что доказывает независимость шифротекстов и открытых текстов в этой системе. По доказанному выше, количество информации в шифротексте относительно открытого текста равно нулю. Это значит, что рассмотренная криптосистема Вернама обладает совершенной секретностью (криптостойкостью) при условии, что для каждого $N$-блока (сообщения) генерируется случайный (одноразовый) $N$-ключ.

\index{криптосистема!Вернама|)}

\section{Расстояние единственности}\label{section_unicity_distance}\index{расстояние единственности}
\selectlanguage{russian}
\index{расстояние единственности}

Использование ключей с длиной, сопоставимой с размером текста, имеет смысл только в очень редких случаях, когда есть возможность предварительно обменяться ключевой информацией большого объёма, много большего, чем планируемый объём передаваемой информации. Но в большинстве случаев использование абсолютно надёжных систем оказывается неэффективным как с экономической, так и с практической точек зрения. Если двум сторонам нужно постоянно обмениваться большим объёмом информации, и они смогли найти надёжный канал для передачи ключа, то ничего не мешает воспользоваться этим же каналом для передачи самой информации сопоставимого объёма.

В подавляющем большинстве криптосистем размер ключа много меньше размера открытого текста, который нужно передать. Попробуем оценить теоретическую надёжность подобных систем, исходя из статистических теоретико-информационных соображений.

В реальной ситуации длина ключа может быть много меньше длины открытого текста, поскольку передача ключа при больших объёмах текста будет затруднена большим объёмом ключа. Это означает, что энтропия ключа\index{энтропия!ключа} может быть много меньше энтропии открытого текста\index{энтропия!открытого текста}: $H(K) \ll H(M)$. Для таких ситуаций важным понятием является \textbf{расстояние единственности}\index{расстояние единственности}, впервые предложенным в работах Клода Шеннона.~\cite{Golomb:2002, Schneier:2011}

\begin{definition}\label{definition:unicity_distance}
\textbf{Расстоянием единственности}\index{расстояние единственности} называется количество символов шифротекста, которое необходимо для однозначного восстановления открытого текста.
\end{definition}

Пусть зашифрованное сообщение (шифротекст) $C$ состоит из $N$ символов $L$-буквенного алфавита:
	\[C = (C_1, C_2, \dots, C_N).\]

Определим функцию $h(n)$ как условную энтропию\index{энтропия!условная} ключа при перехвате криптоаналитиком $n$ символов шифротекста:
\[ \begin{array}{l}
    h ( 0 ) = H(K), \\
    h ( 1 ) = H(K | C_1), \\
    h ( 2 ) = H(K | C_1, C_2), \\
    \dots \\
    h ( n ) = H(K | C_1, C_2, \dots, C_n), \\
    \dots
\end{array} \]

Функция $h(n)$ называется \emph{функцией неопределённости ключа}\index{функция!неопределённости ключа}. Она является невозрастающей функцией числа перехваченных символов $n$. Если для некоторого значения $n_u$ окажется, что $h ( n_u ) = 0$, то это будет означать, что ключ $K$ является детерминированной функцией первых $n_u$ символов шифротекста $C_1, C_2, \dots, C_{n_u}$, и, при неограниченных вычислительных возможностях, используемый ключ $K$ может быть определён. Число $n_u$ и будет являться \emph{расстоянием единственности}. Полученное $n_u$ соответствует определению~\ref{definition:unicity_distance}, так как для корректной криптосистемы однозначное определение ключа также означает и возможность получить открытый текст однозначным способом.

Найдём типичное поведение функции $h(n)$ и значение расстояния единственности $n_u$. Используем следующие предположения:
\begin{itemize}
    \item Криптограф всегда стремится спроектировать систему таким образом, чтобы символы шифрованного текста имели равномерное распределение, и, следовательно, энтропия шифротекста\index{энтропия!шифротекста} имела максимальное значение:
            \[ H(C_1 C_2 \dots C_n) \approx n \log_2 L, ~ n = 1, 2, \dots, N. \]
    \item Имеет место соотношение
            \[ H(C | K) = H(C_1 C_2 \dots C_N | K)  =  H(M), \]
        которое следует из цепочки равенств
            \[ H(MCK) = H(M) + H(K | M) + H(C | MK) = H(M) + H(K), \]
        так как
            \[ H(K | M) = H(K), ~~ H(C | MK) = 0, \]
            \[H(MCK) = H(K) + H(C | K) + H(M | CK) = H(K) + H(C | K), \]
        поскольку
            \[ H(M | CK) = 0. \]
    \item Предполагается, что для любого $n \le N$ приближенно выполняется соотношение
        \[ H(C_n | K) \approx \frac{1}{N} H(M), \]
        \[ H(C_1 C_2\dots C_n | K) \approx \frac{n}{N} H(M). \]
\end{itemize}

Вычислим энтропию $H(C_1 C_2 \dots C_n ; K)$ двумя способами:
    \[ H( C_1 C_2 \dots C_n ; K ) = H(C_1 C_2 \dots C_n) + H(K | C_1 C_2 \dots C_n) \approx \]
        \[ \approx n \log_2 L + h(n), \]
    \[ H( C_1 C_2 \dots C_n ; K ) = H(K) + H(C_1 C_2 \dots C_n | K) \approx \]
        \[ \approx H(K) + \frac{n}{N} H(M). \]

Отсюда следует, что
    \[ h(n) \approx H(K) + n \left( \frac{H(M)}{N} - \log_2 L \right) \]
и
    \[ n_u = \frac{H(K)}{ \left( 1 - \frac{H(M)}{N \log_2 L} \right) \log_2 L} = \frac{H(K)}{\rho \log_2 L}. \]
Здесь
    \[ \rho = 1 - \frac{H(M)}{N \log_2 L} \]
означает избыточность источника открытых текстов\index{избыточность!открытого текста}.

Если избыточность источника измеряется в битах на символ, ключ шифрования выбирается случайным образом из всего множества ключей $\{0, 1\}^{l_K}$, где $l_K$ -- длина ключа в битах, и расстояние единственности $n$ также получается в битах, то формула значительно упрощается:

\begin{equation}\label{eq:unicity_distance_simple_frac}
n_u \approx \frac{l_K}{\rho}.
\end{equation}

Взяв нижнюю границу $H(M)$ энтропии\index{энтропия!открытого текста} одного символа английского текста как $1{,}3$ бит/символ~\cite{Shannon:1951, Schneier:2002}, получим:

	\[ \rho _{en} \approx 1 - \frac{ 1{,}3 }{ \log _2 {26} } \approx 0{,}72.\]

Для русского текста с энтропией $H(M)$, примерно равной $3{,}01$ бит/символ~\cite{Lebedev:1958}\footnote{Следует отметить, что для английского текста значение $1{,}3$ представляет собой суммарную оценку для всего текста, в то время как оценка $3{,}01$ для русского текста получена Лебедевым и Гармашем из анализа \textbf{частот трёхбуквенных сочетаний} в отрывке текста Л. Н. Толстого <<Война и мир>> длиной в 30 тыс. символов. Соответствующая оценка для английского текста, также приведённая в работе Шеннона, примерно равна $3{,}0$}, получаем:

	\[ \rho _{ru} \approx 1 - \frac{ 3{,}0 }{ \log _2 {32} } \approx 0{,}40.\]

Однако, если предположить, что текст передаётся в формате простого текстового файла (\langen{plain text}) в стандартной кодировке UTF-8 (один байт на английский символ и два -- на кириллицу), то значения избыточности становятся примерно равны $0{,}83$ для английского и $0{,}81$ для русского языков:

	\[ \rho _{en, UTF-8} \approx 1 - \frac{ 1{,}3 }{ \log _2 {2^{8}} } \approx 0{,}83,\]
	\[ \rho _{ru, UTF-8} \approx 1 - \frac{ 3{,}0 }{ \log _2 {2^{16}} } \approx 0{,}81.\]

Подставляя полученные числа в выражение~\ref{eq:unicity_distance_simple_frac} для шифров DES\index{шифр!DES} и AES\index{шифр!AES}, получаем таблицу~\ref{table:unicity_distances}.

\begin{table}[!ht]
	\centering
		\begin{tabular}{|| l | r | r ||}
			\hline
			\hline
			\text{Блочный шифр} & \text{Английский текст} & \text{Русский текст} \\
			\hline
			\hline
			\text{Шифр DES\index{шифр!DES},} & \text{ $\approx~67$ бит;} & \text{$\approx~69$ бит;} \\
			\text{ключ 56 бит} & \text{ 2 блока данных} & \text{2 блока данных} \\
			\hline
			\text{Шифр AES\index{шифр!AES},} & \text{ $\approx~153$ бит;} & \text{$\approx~158$ бит;} \\
			\text{ключ 128 бит} & \text{ 3 блока данных} & \text{3 блока данных} \\
			\hline
			\hline
		\end{tabular}
  \caption{Расстояния единственности для шифров DES\index{шифр!DES} и AES\index{шифр!AES} для английского и русского текстов в формате простого текстового файла и кодировке UTF-8}
	\label{table:unicity_distances}
\end{table}

Полученные данные с теоретической точки зрения означают, что, когда криптоаналитик будет подбирать ключ к зашифрованным данным, трёх блоков данных ему будет достаточно, чтобы сделать вывод о правильности выбора ключа расшифрования и корректности дешифровки, если известно, что в качестве открытого текста выступает простой текстовый файл. Если открытым текстом является случайный набор данных, то криптоаналитик не сможет отличить правильно расшифрованный набор данных от неправильного, и расстояние единственности, в соответствии с выводами выше (для нулевой избыточности источника), оказывается равным бесконечности.

Улучшить ситуацию для легального пользователя помогает предварительное сжатие открытого текста с помощью алгоритмов архивации, что уменьшает его избыточность\index{избыточность!открытого текста} (а также уменьшает размер и ускоряет процесс шифрования в целом). Однако расстояние единственности не становится нулевым, так как в результате работы алгоритмов архивации присутствуют различные константные сигнатуры, а для многих текстов можно заранее предсказать примерные словари сжатия, которые будут записаны как часть открытого текста. Более того, используемые на практике программы безопасной передачи данных вынуждены так или иначе встраивать механизмы хотя бы частичной быстрой проверки правильности ключа расшифрования (например, добавлением известной сигнатуры в начало открытого текста). Делается это для того, чтобы сообщить легальному получателю об ошибке ввода ключа, если такая ошибка случится.

Соображения выше показывают, что для одного ключа расшифрования, так или иначе, процедура проверки его корректности является быстрой. Чтобы значительно усложнить работу криптоаналитику, множество ключей, которые требуется перебрать, должно быть большой величиной (например, от $2^{80}$). Это можно сделать, во-первых, увеличением битовой длины ключа, во-вторых, аккуратной разработкой алгоритма шифрования, чтобы криптоаналитик не смог <<отбросить>> часть ключей без их полной проверки.

Несмотря на то, что теоретический вывод о совершенной криптостойкости для практики неприемлем, так как требует большого объёма ключа, сравнимого с объёмом открытого текста, разработанные идеи находят успешное применение в современных криптосистемах. Вытекающий из идей Шеннона принцип выравнивания апостериорного распределения символов в шифротекстах используется в современных криптосистемах с помощью многократных итераций (раундов), включающих замены и перестановки.



\chapter{Блоковые шифры}\label{chapter-block-ciphers}\index{шифр!блоковый|(}

\section{Введение и классификация}\label{section-block-ciphers-intro}
\selectlanguage{russian}

Блоковые шифры являются основой современной криптографии. Многие криптографические примитивы -- криптографически стойкие генераторы псевдослучайной последовательности (см. главу~\ref{chapter-crypto-random}), криптографические функции хэширования (см. главу~\ref{chapter-hash-functions}) -- так или иначе основаны на блоковых шифрах. А использование медленной криптографии с открытым ключом было бы невозможно по практическим соображениям без быстрых блоковых шифров.

Блоковые шифры можно рассматривать как функцию преобразования строки фиксированной длины в строку аналогичной длины\footnote{В случае использования недетерминированных алгоритмов, дающих новый результат при каждом шифровании, длина выхода будет больше. Меньше длина выхода быть не может, так как будет невозможно однозначно восстановить произвольное сообщение.} с использованием некоторого ключа, а также соответствующую ей функцию расшифрования:
\[\begin{array}{l}
	C = E_K\left( M \right), \\
	M'= D_K\left( C \right).
\end{array}\]

Данные функции необходимо дополнить требованиями корректности, производительности и надёжности. Во-первых, функция расшифрования должна однозначно восстанавливать произвольное исходное сообщение:

\[ \forall k \in \group{K}, m \in \group{M}: D_k \left( E_k\left( m \right) \right) = m. \]

Во-вторых, функции шифрования и расшифрования должны быстро выполняться легальными пользователями (знающими ключ). В-третьих, должно быть невозможно найти открытый текст сообщения по шифротексту без знания ключа, кроме как полным перебором всех возможных ключей расшифрования. Также, что менее очевидно, надёжная функция блокового шифра не должна давать возможность найти ключ шифрования (расшифрования), даже если злоумышленнику известны пары открытого текста и шифротекста. Последнее свойство защищает от атак на основе известного открытого текста\index{атака!с известным открытым текстом} и на основе известного шифротекста\index{атака!с известным шифротекстом}, а также активно используется при построении криптографических функций хэширования в конструкции Миагучи~---~Пренеля\index{конструкция!Миагучи~---~Пренеля}. То есть:
\begin{itemize}
	\item $C = f \left( M, K \right)$ и $M = f \left( C, K \right)$ должны вычисляться быстро (легальные операции);
	\item $M = f \left( C \right)$ и $C = f \left( M \right)$ должны вычисляться не быстрее, чем $\left| \group{K} \right|$ операций расшифрования (шифрования), при условии, что злоумышленник может отличить корректное сообщение (см. выводы к разделу~\ref{section_unicity_distance});
	\item $K = f \left( M, C \right)$ должно вычисляться не быстрее, чем $\left| \group{K} \right|$ операций шифрования;
\end{itemize}

Если размер ключа достаточно большой (от 128 бит и выше), то функцию блокового шифрования, удовлетворяющую указанным выше условиям, можно называть надёжной.

Блоковые шифры делят на два больших класса по методу построения.
\begin{itemize}
	\item Шифры, основанные на SP-сетях (сети замены-пере\-становки), основанные на \emph{обратимых} преобразованиях с открытым текстом. При разработке таких шифров криптограф должен следить за тем, чтобы каждая из производимых операций была и криптографически надёжна, и обратима при знании ключа.
	\item Шифры, в той или иной степени построенные на ячейке Фейстеля. В данных шифрах используется конструкция под названием <<ячейка Фейстеля>>, которая по методу построения уже обеспечивает обратимость операции шифрования легальным пользователем при знании ключа. Криптографу при разработке функции шифрования остаётся сосредоточиться на надёжности конструкции.
\end{itemize}

\begin{figure}[!b]
	\centering
	\includegraphics[width=1\textwidth]{pic/block-cipher}
  \caption{Общая структура раундового блокового шифра. С помощью функции ключевого расписания из ключа $K$ получается набор раундовых ключей $K1, K2, \dots$. Открытый текст $M$ разбивается на блоки $M1, M2, \dots$, каждый из которых проходит несколько раундов шифрования, используя соответствующие раундовые ключи. Результаты последних раундов шифрования каждого из блоков объединяются в шифротекст $C$ с помощью одного из режима сцепления блоков}
  \label{fig:block-cipher}
\end{figure}

Все современные блоковые шифры являются \emph{раундовыми} (см. рис.~\ref{fig:block-cipher}). То есть блок текста проходит через несколько одинаковых (или похожих) преобразований, называемых \emph{раундами шифрования}. У~функции шифрования также могут существовать начальный и завершающий раунды, отличающиеся от остальных (обычно -- отсутствием некоторых преобразований, которые не имеют смысла для <<крайних>> раундов).

Аргументами каждого раунда являются результат предыдущего раунда (для самого первого -- часть открытого текста) и \emph{раундовый ключ}\index{ключ!раундовый}. Раундовые ключи получаются из оригинального ключа шифрования с помощью процедуры, получившей название алгоритма \emph{ключевого расписания}\index{ключевое расписание} (также встречаются названия <<расписание ключей>>, <<процедура расширения ключа>> и др.; \langen{key schedule}). Функция ключевого расписания является важной частью блокового шифра. На потенциальной слабости этой функции основаны такие криптографические атаки, как атака на основе связанных ключей\index{атака!на связанных ключах} и атака скольжения\index{атака!скольжения}.

После прохождения всех раундов шифрования блоки $C1$, $C2$,~$\dots$ объединяются в шифротекст $C$ с помощью одного из режимов сцепления блоков (см. раздел~\ref{chapter-block-chaining}). Простейшим примером режима сцепления блоков является режим электронной кодовой книги\index{режим!электронной кодовой книги}, когда блоки $C1$, $C2$,~$\dots$ просто конкатенируются в шифротекст $C$ без дополнительной обработки.

К числовым характеристикам блокового шифра относят:
\begin{itemize}
	\item размер входного и выходного блока;
	\item размер ключа шифрования;
	\item количество раундов.
\end{itemize}

Также надёжные блоковые шифры обладают \emph{лавинным эффектом}\index{лавинный эффект} (\langen{avalanche effect}): изменение одного бита в блоке открытого текста или ключа приводит к полному изменению соответствующего блока шифротекста.


\input{lucifer}

\section{Ячейка Фейстеля}
\selectlanguage{russian}

Одним из основных методов построения современных блоковых шифров является ячейка \textbf{Фейстеля} (Feistel), изображенная на рисунке~\ref{fig:Feistel}. Главная особенность шифрования с ячейкой Фейстеля состоит в том, что обратимость шифрования (т.~е. расшифрование) не зависит от обратимости преобразования $F$ внутри ячейки. Широкое применение ячеек Фейстеля в шифрах 1970--90-х годов вызвано бурным развитием персональных компьютеров. Шифрование выполняется \textbf{раундами}, на каждом раунде выполняется одно и то же преобразование ячейки Фейстеля, но с разными ключами. Общее количество раундов 16--32.

\begin{figure}[!ht]
    \centering
    \includegraphics[width=0.6\textwidth]{pic/feistel}
    \caption{Ячейка Фейстеля\label{fig:Feistel}}
\end{figure}

Здесь введены обозначения: $X$ -- блок двоичных символов, который записан в регистр памяти, состоящий из двух частей, $X = (L,R)$, где $L$ -- начальное содержимое левого регистра, $\tilde{L}$ -- содержимое левого регистра сдвига после преобразования, $R$ -- начальное содержимое правого регистра, $\tilde{R}$ -- содержимое правого регистра после преобразования, $K$ -- ключ шифрования, задающий преобразование $F(K,R)$. Знак $\oplus$ определяет операцию побитового суммирования по модулю 2, то есть операцию XOR. Перекрестные линии указывают на замену частей регистра. После одного элементарного преобразования содержимое правого регистра заменяется содержимым левого регистра и наоборот. $\tilde{L},\tilde{R}$ -- результат элементарного шифрования, выполненного за один раунд. $L_{1}$ -- содержимое правого регистра после замены, $R_{1}$ -- содержимое левого регистра после замены. Основным шифрующим преобразованием является функция $F$.

Ячейка Фейстеля -- произведение двух перестановок $T$ и $G$, где $T$ -- замена левой части на правую и наоборот. Запишем преобразование $Y=TG(X,L)$, выполняемое этой ячейкой:
\[
  \begin{array}{l}
    \tilde{X} = (\tilde{L}, \tilde{R}) = (L \oplus F(K,R), R) \equiv G(X, L), \\
    Y = TGX. \\
  \end{array}
\]

Если дважды применим перестановку, то получим снова открытый текст:
\[
    \begin{array}{l}
        \tilde{\tilde{L}} = \tilde{L} \oplus F(K, \tilde{R}) = (L \oplus F(K,R) \oplus F(K,R)) = L, \\
        \tilde{\tilde{R}} = R.\\
    \end{array}
\]

Многократное применение преобразования $Y=TG(X,L)$ с различными ключами представим в виде
\[
  \begin{array}{l}
    Y_1 = T G_1 X,\\
    Y_2 = T G_2 Y_1 = T G_2 T G_1 X, \\
    \ldots, \\
    Y_{m-1} = T G_{m-1} Y_{m-2} = T G_{m-1} T G_{m-2} \ldots T G_1 X.\\
  \end{array}
\]
В первом уравнении показан результат первого шифрования с ключом $K_{1}$, во втором уравнении -- результат шифрования с ключом $K_{2}$ и т.~д., в $(m-1)$-м уравнении -- результат с ключом $K_{m-1}$. В последнем, $(m)$-м уравнении, перестановку $T$ можно не использовать:
\[
   Y_{m}= G_{m} Y_{m-1} = G_{m} T G_{m-1} \ldots T G_{1} X.\\
\]

Как видно из приведенных соотношений, пара величин -- содержимое регистра и первый ключ $X, K_{1}$ -- влияет на все позиции шифрованного текста. Полностью разрушается статистическая структура исходного текста за счет преобразований, вызывающих \emph{лавинный эффект}\index{лавинный эффект}. \textbf{Лавинный эффект} -- это распространение <<влияния>> одного бита открытого текста (или ключа) на все остальные биты шифруемого блока за определённое количество раундов.
%Для ячеек Фейстеля это количество равно 2.

Одной из характеристик блокового шифра является число раундов, за которое достигается полная диффузия (конфузия) -- зависимость всех битов выхода (входа) от всех битов входа (выхода). Вход -- это открытый текст и ключ.

Криптостойкость ячейки Фейстеля подтверждается тем фактом, что не существует примеров ее взлома (в случае шифра DES взлом был сделан полным перебором 56-битового ключа, а не взломом самой криптосистемы; например, российский стандарт ГОСТ 28147-89 на ячейке Фейстеля с 256-битовым ключом не взломан).

Рассмотрим процедуру расшифрования. Легальный пользователь знает все ключи и последовательность их применения. Он выполняет следующие операции. Имеем шифрованное сообщение $Y_{m}$. На первом шаге вычисляет:
\[
    G_{m} Y_{m} = G_{m} G_{m} Y_{m-1} = Y_{m-1}.
\]
На втором шаге использует найденное сообщение $Y_{m-1}$ и аналогично находит $Y_{m-2}$:
\[
    G_{m-1} T Y_{m-1} = G_{m-1} T T G_{m-1} Y_{m-2} = Y_{m-2}.
\]
Продолжает этот процесс до получения $Y_{1}$. После этого находит $X$:
\[
    G_{1} T Y_{1} = G_{1} T T G_{1} X = X.
\]
Как показали эти операции, вычислительная сложность устройства расшифрования ячейки Фейстеля такая же, как сложность устройства шифрования.

Раундовые блоковые шифры должны обеспечивать \emph{диффузию}, при которой каждый бит входа и ключа влияет на все биты выхода, и \emph{конфузию}, при которой каждый бит выхода нелинейно зависит от всех битов входа и ключа.

Основные свойства, которыми должна обладать функция $F$:
\begin{itemize}
    \item создание лавинного эффекта;
    \item нелинейность по отношению к операции XOR.
\end{itemize}

Как правило, функция $F$ включает таблицы перестановки $P$ и подстановки групп бит, так называемые s-блоки (от слова substitution), и функцию перестановки, перемешивающую биты между последовательно исполняемыми $s$-блоками. В совокупности эти действия и обеспечивают требуемые свойства ячейки.


\input{GOST_28147-89}

\section{Стандарт шифрования США AES}\index{шифр!AES|(}
\selectlanguage{russian}

До 2001 г. стандартом шифрования данных в США был DES\index{шифр!DES} (аббревиатура от Data Encryption Standard), который был принят в 1980 году. Входной блок открытого текста и выходной блок шифрованного текста DES составляли по 64 бита каждый, длина ключа -- 56 бит (до процедуры расширения). Алгоритм основан на ячейке Фейстеля\index{ячейка Фейстеля} с $s$-блоками и таблицами расширения и перестановки бит. Количество раундов -- 16.

Для повышения криптостойкости и замены стандарта DES был объявлен конкурс на новый стандарт AES (аббревиатура от Advanced Encryption Standard). Победителем конкурса стал шифр Rijndael. Название составлено с использованием первых слогов фамилий его создателей (Rijmen and Daemen). В русскоязычном варианте читается как <<Рэндал>>~\cite{Kiwi:1999}. Шифр был утвержден в качестве стандарта FIPS 197 в ноябре 2001 г. и введён в действие 26 мая 2002 года~\cite{FIPS-PUB-197}.

AES -- это раундовый\index{шифр!раундовый} блоковый\index{шифр!блоковый} шифр с переменной длиной ключа (128, 192 или 256 бит) и фиксированной длиной входного и выходного блоков (128 бит).

\subsection[Состояние, ключ шифрования и число раундов]{Состояние, ключ шифрования и число \protect\\ раундов}

Различные преобразования воздействуют на результат промежуточного шифрования, называемый \textit{состоянием} ($\mathsf{State}$). Состояние представлено $(4 \times 4)$-матрицей из байт.

\textit{Ключ шифрования раунда} ($\mathsf{Key}$) также представляется прямоугольной $(4 \times \mathsf{Nk})$-матрицей из байт $k_{i,j}$, где $\mathsf{Nk}$ равно длине ключа, разделенной на 32, то есть 4, 6 или 8.

Эти представления приведены ниже.
\[
    \mathsf{State} = \left[ \begin{array}{cccc}
        a_{0,0} & a_{0,1} & a_{0,2} & a_{0,3} \\
        a_{1,0} & a_{1,1} & a_{1,2} & a_{1,3} \\
        a_{2,0} & a_{2,1} & a_{2,2} & a_{2,3} \\
        a_{3,0} & a_{3,1} &a_{3,2} & a_{3,3}  \\
    \end{array} \right],
\] \[
    \mathsf{Key} = \left[ \begin{array}{cccc}
        k_{0,0} & k_{0,1} & k_{0,2} & k_{0,3} \\
        k_{1,0} & k_{1,1} & k_{1,2} & k_{1,3} \\
        k_{2,0} & k_{2,1} & k_{2,2} & k_{2,3} \\
        k_{3,0} & k_{3,1} & k_{3,2} & k_{3,3} \\
    \end{array} \right].
\]

Иногда блоки символов интерпретируются как одномерные последовательности из 4-байтых векторов, где каждый вектор является соответствующим столбцом прямоугольной таблицы. В этих случаях таблицы можно рассматривать как наборы из 4, 6 или 8 векторов, нумеруемых в диапазоне $0 \dots 3, 0 \dots 5$ или $0 \dots 7$. В тех случаях, когда нужно пометить индивидуальный байт внутри 4-байтого вектора, используется обозначение $(a, b, c, d)$, где $a, b, c, d$ соответствуют байтам в одной из позиций $0, 1, 2, 3$ в столбце или векторе.

\textit{Входные} и \textit{выходные} блоки шифра AES рассматриваются как последовательности 16 байт $(a_0, a_1, \dots, a_{15})$. Преобразование входного блока $(a_0, \dots, a_{15})$ в исходную $(4 \times 4)$ матрицу состояния $\mathsf{State}$ или конечной матрицы состояния в выходную последовательность проводится по правилу (запись по столбцам):
    \[ a_{i,j} = a_{i + 4j}, ~ i = 0 \dots 3, ~ j = 0 \dots 3. \]

Аналогично ключ шифрования может рассматриваться как последовательность байт $(k_0, k_1, \dots, k_{4 \cdot \mathsf{Nk} - 1})$, где $\mathsf{Nk} = 4, 6, 8$. Число байт в этой последовательности равно 16, 24 или 32, а номера этих байт находятся в интервалах $0 \dots 15, ~ 0 \dots 23$ или $0 \dots 31$ соответственно. $(4 \times \mathsf{Nk})$-матрица ключа шифрования $\mathsf{Key}$ задаётся по правилу:
    \[ k_{i,j} = k_{i + 4j}, ~ i = 0 \dots 3, ~ j = 0 \dots \mathsf{Nk} - 1. \]

Число раундов $\mathsf{Nr}$ зависит от длины ключа. Его значения приведены в таблице ниже.

\begin{center}
    \begin{tabular}{|l|c|c|c|}
    \hline
    Длина ключа, биты           &128 & 192 & 256 \\
    $\mathsf{Nk}$               & 4  & 6   & 8 \\
    Число раундов $\mathsf{Nr}$ & 10 & 12 & 14 \\
    \hline
    \end{tabular}
\end{center}


\subsection{Операции в поле}

При переходе от одного раунда к другому матрицы \textit{состояния} и \textit{ключа шифрования раунда} подвергаются ряду преобразований. Преобразования могут осуществляться над:
\begin{itemize}
    \item отдельными байтами или парами байт (необходимо определить операции сложения и умножения);
    \item столбцами матрицы, которые рассматриваются как 4-мерные векторы с соответствующими байтами в качестве элементов;
    \item строками матрицы.
\end{itemize}

В алгоритме шифрования AES байты рассматриваются как элементы поля $\GF{2^8}$, а вектор-столбцы из четырех байт -- как многочлены третьей степени над полем $\GF{2^8}$. В Приложении~\ref{chap:discrete-math} дано подробное описание этих операций.

Хотя определение операций дано через их математическое представление, в реализациях шифра AES активно используются таблицы с заранее вычисленными результатами операций над отдельными байтами, включая взятие обратного элемента и перемножение элементов в поле $\GF{2^8}$ (на что требуется 256 байт и 65 Кбайт памяти соответственно).

\subsection{Операции одного раунда шифрования}

В каждом раунде шифра AES, кроме последнего раунда, производятся следующие 4 операции:
\begin{itemize}
  \item замена байт, $\mathsf{SubBytes}$;
  \item сдвиг строк, $\mathsf{ShiftRows}$;
  \item перемешивание столбцов, $\mathsf{MixColumns}$;
  \item добавление текущего ключа, $\mathsf{AddRoundKey}$.
\end{itemize}

В последнем раунде исключается операция <<перемешивание столбцов>>. В обозначениях, близких к языку С, можно записать программу в следующем виде:
\[
    \begin{array}{l}
        \mathsf{Round(State, RoundKey)} \{ \\
        ~~~~ \mathsf{SubBytes(State)}; \\
        ~~~~ \mathsf{ShiftRows(State)}; \\
        ~~~~ \mathsf{MixColumns(State)}; \\
        ~~~~ \mathsf{AddRoundKey(State, RoundKey)}; \\
        \} \\
    \end{array}
\]
Последний раунд слегка отличается, и его можно записать в следующем виде:
\[
    \begin{array}{l}
        \mathsf{Round(State, RoundKey)} \{ \\
        ~~~~ \mathsf{SubBytes(State)}; \\
        ~~~~ \mathsf{ShiftRows(State)}; \\
        ~~~~ \mathsf{AddRoundKey(State, RoundKey)}; \\
        \} \\
    \end{array}
\]
В этих обозначениях все <<функции>>, а именно: $\mathsf{Round}$, $\mathsf{SubBytes}$, $\mathsf{ShiftRows}$, $\mathsf{MixColumns}$ и $\mathsf{AddRoundKey}$ -- воздействуют на матрицы, определяемые указателем $\mathsf{(State, RoundKey)}$. Сами преобразования описаны в следующих разделах.


\subsubsection{Замена байт $\mathsf{SubBytes}$}

Нелинейная операция <<замена байт>> действует независимо на каждый байт $a_{i,j}$ текущего состояния. Таблица замены (или $s$-блок) является обратимой и формируется последовательным применением двух преобразований.

\begin{enumerate}
    \item Сначала байт $a$ представляется как элемент $a(x)$ поля Галуа $\GF{2^8}$ и заменяется на обратный элемент $a^{-1} \equiv a^{-1}(x)$ в поле. Байт $\mathrm{'00'}$, для которого обратного элемента не существует, переходит сам в себя.
    \item Затем к обратному байту $a^{-1} = (x_0, x_1, x_2, x_3, x_4, x_5, x_6, x_7)$ применяется аффинное преобразование над полем $\GF{2}$ следующего вида:
        \[
            \left[  \begin{array}{c}
                y_{0} \\ y_{1} \\ y_{2} \\ y_{3} \\ y_{4} \\ y_{5} \\ y_{6} \\ y_{7} \\
            \end{array} \right] = \left[ \begin{array}{cccccccc}
                1 & 0 & 0  & 0 & 1 & 1 & 1 & 1 \\
                1 & 1 & 0  & 0 & 0 & 1 & 1 & 1 \\
                1 & 1 & 1  & 0 & 0 & 0 & 1 & 1 \\
                1 & 1 & 1  & 1 & 0 & 0 & 0 & 1 \\
                1 & 1 & 1  & 1 & 1 & 0 & 0 & 0 \\
                0 & 1 & 1  & 1 & 1 & 1 & 0 & 0 \\
                0 & 0 & 1  & 1 & 1 & 1 & 1 & 0 \\
                0 & 0 & 0  & 1 & 1 & 1 & 1 & 1  \
            \end{array} \right] \cdot \left[ \begin{array}{c}
                x_{0} \\ x_{1} \\ x_{2} \\ x_{3} \\ x_{4} \\ x_{5} \\ x_{6} \\ x_{7} \\
            \end{array} \right] + \left[ \begin{array}{c}
                1 \\ 1 \\ 0 \\ 0 \\ 0 \\ 1 \\ 1 \\ 0 \\
            \end{array} \right].
        \]
\end{enumerate}

В полиномиальном представлении это аффинное преобразование имеет вид:
\[Y(z)=(z^4)X(z)(1+z+z^2+z^3+z^4)\mod(1+z^8) + F(z).\]
Применение описанных операций $s$-блока ко всем байтам текущего состояния обозначено
    \[ \mathsf{SubBytes(State)}. \]

Обращение операции $\mathsf{SubBytes(State)}$ также является заменой байт. Сначала выполняется обратное аффинное преобразование, а затем от полученного байта берётся обратный.


\subsubsection{Сдвиг строк $\mathsf{ShiftRows}$}

Для выполнения операции <<сдвиг строк>> строки в таблице текущего состояния циклически сдвигаются влево. Величина сдвига различна для различных строк. Строка $0$ не сдвигается вообще. Строка $1$ сдвигается на $C_1=1$ позицию, строка $2$ –- на $C_2=2$ позиции, строка $3$ -– на $C_3=3$ позиции.
%Величины $C1,C2$ и $C3$ зависят от $Nb$. Их значения приведены в табл.~\ref{tab:AES-shift-rows}.
%
%\begin{table}[!ht]
%    \centering
%    \begin{tabular}{|c|c|c|c|}
%        \hline
%        Nb & C1 & C2 & C3 \\
%        \hline
%        4  & 1  & 2  & 3  \\
%        \hline
%        6  & 1  & 2  & 3  \\
%        \hline
%        8  & 1  & 3  & 4  \\
%        \hline
%    \end{tabular}
%    \caption{Сдвиг $C$ и длина блока $Nb$.}
%    \label{tab:AES-shift-rows}
%\end{table}


\subsubsection{Перемешивание столбцов $\mathsf{Mix Columns}$}

При выполнении операции <<перемешивание столбцов>> столбцы матрицы текущего состояния рассматриваются как многочлены над полем $\GF{2^8}$ и умножаются по модулю многочлена $y^4 +1$ на фиксированный многочлен $\mathbf{c}(y)$, где
    \[ \mathbf{c}(y) = \mathrm{'03'} y^3 + \mathrm{'01'} y^2 + \mathrm{'01'} y + \mathrm{'02'}. \]
Этот многочлен взаимно прост с многочленом $y^4 + 1$ и, следовательно, обратим. Перемножение удобнее проводить в матричном виде. Если $\mathbf{b}(y) = \mathbf{c}(y) \otimes \mathbf{a}(y)$, то
\[
    \left[ \begin{array}{c}
        b_{0} \\ b_{1} \\ b_{2} \\ b_{3} \\
    \end{array}\right] =  \left[ \begin{array}{cccc}
        \mathrm{'02'} & \mathrm{'03'} & \mathrm{'01'} & \mathrm{'01'} \\
        \mathrm{'01'} & \mathrm{'02'} & \mathrm{'03'} & \mathrm{'01'} \\
        \mathrm{'01'} & \mathrm{'01'} & \mathrm{'02'} & \mathrm{'03'} \\
        \mathrm{'03'} & \mathrm{'01'} & \mathrm{'01'} & \mathrm{'02'} \\
    \end{array} \right] \cdot \left[ \begin{array}{c}
        a_{0} \\ a_{1} \\ a_{2} \\ a_{3} \\
     \end{array} \right].
\]

Обратная операция состоит в умножении на многочлен $\mathbf{d}(y)$, обратный многочлену $\mathbf{c}(y)$ по модулю $y^4 + 1$, то есть
\[
    (\mathrm{'03'} y^{3} + \mathrm{'01'} y^{2} + \mathrm{'01'} y + \mathrm{'02'}) \otimes \mathbf{d}(y) = \mathrm{'01'}.
\]
Этот многочлен равен
\[
    \mathbf{d}(y) = \mathrm{'0B'} y^3 + \mathrm{'0D'} y^2 + \mathrm{'09'} y + \mathrm{'0E'}.
\]


\subsubsection{Добавление ключа раунда $\mathsf{AddRoundKey}$}

Операция <<Добавление ключа раунда>> состоит в том, что к матрице текущего состояния добавляется по модулю $2$ матрица ключа текущего раунда. Обе матрицы должны иметь одинаковые размеры. Матрица ключа раунда вычисляется с помощью процедуры \emph{расширения ключа}, описанной ниже. Операция <<Добавление ключа раунда>> обозначается $\mathsf{AddRoundKey(State, RoundKey)}$.

\[
    \left[ \begin{array}{cccc}
        a_{0,0} & a_{0,1} & a_{0,2} & a_{0,3} \\
        a_{1,0} & a_{1,1} & a_{1,2} & a_{1,3} \\
        a_{2,0} & a_{2,1} & a_{2,2} & a_{2,3} \\
        a_{3,0} & a_{3,1} & a_{3,2} & a_{3,3}
    \end{array} \right]
    \oplus
    \left[ \begin{array}{cccc}
        k_{0,0} & k_{0,1} & k_{0,2} & k_{0,3} \\
        k_{1,0} & k_{1,1} & k_{1,2} & k_{1,3} \\
        k_{2,0} & k_{2,1} & k_{2,2} & k_{2,3} \\
        k_{3,0} & k_{3,1} & k_{3,2} & k_{3,3}
    \end{array} \right] =
\] \[
    = \left[ \begin{array}{cccc}
        b_{0,0} & b_{0,1} & b_{0,2} & b_{0,3} \\
        b_{1,0} & b_{1,1} & b_{1,2} & b_{1,3} \\
        b_{2,0} & b_{2,1} & b_{2,2} & b_{2,3} \\
        b_{3,0} & b_{3,1} & b_{3,2} & b_{3,3}
    \end{array} \right].
\]


\subsection{Процедура расширения ключа}

Матрица ключа текущего раунда получается из исходного ключа шифра с помощью специальной процедуры, состоящей из расширения ключа и выбора раундового ключа. Основные принципы этой процедуры состоят в следующем.
\begin{itemize}
    \item Суммарная длина ключей всех раундов равна длине блока, умноженной на увеличенное на 1 число раундов. Для блока длины 128 бит и 10 раундов общая длина всех ключей раундов равна 1408;
    \item С помощью ключа шифра находят \textit{расширенный ключ}.
    \item Ключи \emph{раунда} выбираются из \emph{расширенного} ключа по правилу: ключ первого раунда состоит из первых 4-х столбцов матрицы расширенного ключа, второй ключ –- из следующих 4-х столбцов и т.~д.
\end{itemize}

Расширенный ключ –- это матрица $\mathsf{W}$, состоящая из $4 \cdot (\mathsf{Nr} + 1)$ 4-байтных вектор-столбцов, каждый столбец $i$ обозначается $\mathsf{W}[i]$.

Далее рассматривается только случай, когда ключ шифра состоит из $16$ байт. Первые $\mathsf{Nk} = 4$ столбца содержат ключ шифра. Остальные столбцы вычисляются рекурсивно из столбцов с меньшими номерами.

Для $\mathsf{Nk} = 4$ имеем 16-байтый ключ
\[
    \mathsf{Key} = (\mathsf{Key}[0], \mathsf{Key}[1], \dots, \mathsf{Key}[15]).
\]
Приведём алгоритм расширения ключа для $\mathsf{Nk} = 4$.
\begin{algorithm}[iht]
    \caption{$\mathsf{KeyExpansion}(\mathsf{Key}, \mathsf{W})$\label{alg:AES-key-exp}}
    \begin{algorithmic}
        \FOR{ $i=0$ \TO $\mathsf{Nk} - 1$}
            \STATE $\mathsf{W}[i] = (\mathsf{Key}[4i], ~ \mathsf{Key}[4i+1], ~ \mathsf{Key}[4i+2], ~ \mathsf{Key}[4i+3])^T$;
        \ENDFOR
        \FOR{ $i = \mathsf{Nk}$ \TO $4 \cdot (\mathsf{Nr} + 1) - 1$}
            \STATE $\mathsf{temp} = \mathsf{W}[i-1]$;
            \IF{ ($i = 0 \mod \mathsf{Nk}$)}
                \STATE $\mathsf{temp} = \mathsf{SubWord}(\mathsf{RotWord}(\mathsf{temp})) ~ \oplus ~ \mathsf{Rcon}[i / \mathsf{Nk}]$;
            \ENDIF
            \STATE $\mathsf{W}[i] = \mathsf{W}[i - \mathsf{Nk}] ~ \oplus ~ \mathsf{temp}$;
        \ENDFOR
    \end{algorithmic}
\end{algorithm}

%\[
%    \begin{array}{l}
%        \mathsf{KeyExpansion}(\mathsf{Key}, \mathsf{W}) \{ \\
%        ~~~~ \mathsf{for ~ (i = 0; ~ i < Nk = 4; ~ i++)} \\
%        ~~~~~~~~ \mathsf{W[i] = (Key[4 \cdot i], ~ Key[4*i+1], ~ Key[4*i+2], ~ Key[4*i+3]);} \\
%        ~~~~ \mathsf{for ~ (i = Nk; ~ i < 4 * (Nr + 1); ~ i++)} ~ \{ \\
%        ~~~~~~~~ \mathsf{temp = W[i-1];} \\
%        ~~~~~~~~ \mathsf{if ~ (i ~ \% ~ Nk ~ == ~ 0)} \\
%        ~~~~~~~~~~~~ \mathsf{temp = SubWord(RotWord(temp))} ~ \oplus ~ \mathsf{Rcon[i / Nk];} \\
%        ~~~~~~~~ \mathsf{W[i] = W[i - Nk]} ~ \oplus ~ \mathsf{temp;} \\
%        ~~~~ \} \\
%        \} \\
%    \end{array}
%\]

Здесь $\mathsf{SubWord}(\mathsf{W}[i])$ обозначает функцию, которая применяет операцию <<замена байт>> (или s-блок) $\mathsf{SubBytes}$ к каждому из 4-х байт столбца $\mathsf{W}[i]$. Функция $\mathsf{RotWord}(\mathsf{W}[i])$  осуществляет циклический сдвиг вверх байт столбца $\mathsf{W}[i]$: если $\mathsf{W}[i] = (a, b, c, d)^T$, то $\mathsf{RotByte}(\mathsf{W}[i]) = (b, c, d, a)^T$. Векторы-константы $\mathsf{Rcon}[i]$ определены ниже.

Как видно из этого описания, первые $\mathsf{Nk} = 4$ столбца заполняются ключом шифра. Все следующие столбцы $\mathsf{W}[i]$ равны сумме по модулю 2 предыдущего столбца $\mathsf{W}[i-1]$ и столбца $\mathsf{W}[i-4]$. Для столбцов $\mathsf{W}[i]$ с номерами $i$, кратными $\mathsf{Nk} = 4$, к столбцу $\mathsf{W}[i-1]$ применяются операции $\mathsf{RotWord(W)}$ и $\mathsf{SubWord(W)}$, а затем производится суммирование по модулю 2 со столбцом $\mathsf{W}[i-4]$ и константой раунда $\mathsf{Rcon}[i ~/~ 4]$.

%Для $\mathsf{Nk}>6$ имеем
%\[
%\begin{array}{l}
% \mathsf{KeyExpansion\,(byte\,Key\,[4*Nk]\,\, word \,\, W[Nb*(Nr+1)])}\\
%  \{\\
% \quad\quad \mathsf{for\,\,(i=0;\,\, i<Nk;\,\,i++)} \\
%  \qquad \quad\quad\quad \mathsf{W[i]=(Key[4*i];Key[4*i+1];Key[4*i+2];Key[4*i+3]);}\\
%  \quad\quad \mathsf{for \,\,(i=Nk;\,\,i<Nb*(Nr+1);\,\,i++)}\\
%  \quad\quad \{ \\
%  \quad \quad\quad\quad \mathsf{temp=W[i-1]}; \\
%  \quad \quad\quad\quad \mathsf{if\,\,(i\quad\% \quad Nk==0)}\\
%  \qquad \qquad \qquad \quad \mathsf{temp=SubByte(RotByte(temp))\quad\widehat{\,}\quad Rcon[i/Nk]};\\
%\quad \quad\quad\quad \mathsf{else \,\,if\,\,(i\quad\% \quad Nk==4)}\\
% \qquad \qquad \qquad \quad \mathsf{temp=SubByte(temp)};\\
%  \quad \quad\quad\quad \mathsf{W[i]=W[i-Nk] \quad\widehat{\,}\quad temp};\\
%  \quad\quad \} \\
%  \}\, \\
%\end{array}
%\]
%Различие между этими двумя случаями состоит в том, что во втором случае к столбцу $\mathsf{W[i-1]}$ применяются операции
% $\mathsf{RotByte(W)}$ и $\mathsf{SubByte(W)}$, если $\mathsf{i-4}$ кратно $\mathsf{Nk}$.\\

Константы раундов определяются следующим образом:
    \[ \mathsf{Rcon}[i] = (\mathsf{RC}[i], \mathrm{'00'}, \mathrm{'00'}, \mathrm{'00'})^T, \]
где байт $\mathsf{RC}[1] = \mathrm{'01'}$, а байты $\mathsf{RC}[i] = \alpha^{i-1}, ~ i = 2, 3, \dots$. Байт $\alpha = \mathrm{'02'}$ –- это примитивный элемент поля $\GF{2^8}$.

\example
Пусть $\mathsf{Nk} = 4$. В этом случае ключ шифра имеет длину 128 бит. Найдем столбцы расширенного ключа. Столбцы $\mathsf{W}[0], \mathsf{W}[1], \mathsf{W}[2], \mathsf{W}[3]$ непосредственно заполняются битами ключа шифра. Номер следующего столбца $\mathsf{W}[4]$ кратен $\mathsf{Nk}$, поэтому
\[
    \mathsf{W}[4] = \mathsf{SubWord}(\mathsf{RotWord}(\mathsf{W}[3])) \oplus \mathsf{W}[0] \oplus
        \left[ \begin{array}{c}
            \mathrm{'01'} \\ \mathrm{'00'} \\ \mathrm{'00'} \\ \mathrm{'00'} \\
        \end{array} \right].
\]
Далее имеем
\[
    \begin{array}{l}
        \mathsf{W}[5] = \mathsf{W}[4] \oplus \mathsf{W}[1], \\
        \mathsf{W}[6] = \mathsf{W}[5] \oplus \mathsf{W}[2], \\
        \mathsf{W}[7] = \mathsf{W}[6] \oplus \mathsf{W}[3].  \\
    \end{array}
\]
Затем
\[
    \mathsf{W}[8] = \mathsf{SubWord}(\mathsf{RotWord}(\mathsf{W}[7])) \oplus \mathsf{W}[4] \oplus
        \left[ \begin{array}{c}
            \alpha \\
            \mathrm{'00'}\\
            \mathrm{'00'}\\
            \mathrm{'00'}\\
        \end{array} \right] ,
\] \[
    \begin{array}{l}
        \mathsf{W}[9] = \mathsf{W}[8] \oplus \mathsf{W}[5], \\
        \mathsf{W}[10] = \mathsf{W}[9] \oplus \mathsf{W}[6], \\
        \mathsf{W}[11] = \mathsf{W}[10] \oplus \mathsf{W}[7] \\
    \end{array}
\]
и т.~д.
\exampleend

%\example
%Пусть $\mathsf{Nk=6}.$ В этом случае ключ шифра имеет длину 192 бита. Найдем столбцы расширенного ключа. Столбцы $\mathsf{W[0],W[1],W[2],W[3],W[4],W[5]}$ непосредственно заполняются
%битами ключа шифра. Номер следующего столбца $\mathsf{W[6]}$ кратен $\mathsf{Nk}$, поэтому
%\[
%\begin{array}{ccccccc}
% \mathsf{W[6]} & = & \mathsf{SubByte(RotByte(W[5]))} &\oplus  &  \mathsf{W[0]} & \oplus  & \left[ \begin{array}{c}
% \mathsf{`01'} \\
%  \mathsf{`00'}\\
%  \mathsf{`00'}\\
%  \mathsf{`00'}\\
%\end{array}
%\right]    \\
%\end{array}
%\].
%
%Далее имеем
%\[
%\begin{array}{ccc}
% \mathsf{W[7]=W[6]}\oplus \mathsf{W[1]}; & \mathsf{W[8]=W[7]}\oplus \mathsf{W[2]}; & \mathsf{W[9]=W[8]}\oplus \mathsf{W[3]}; \\
% \mathsf{W[10]=W[9]}\oplus \mathsf{W[4]}; &\mathsf{ W[11]=W[10]}\oplus \mathsf{W[5]}.\\
%\end{array}
%\]
%Затем
%\[
%\begin{array}{ccccccc}
% \mathsf{W[12]} & = & \mathsf{SubByte(RotByte(W[11]))} &\oplus  &  \mathsf{W[6]} & \oplus  & \left[ \begin{array}{c}
% \mathsf{\alpha} \\
%  \mathsf{`00'}\\
%  \mathsf{`00'}\\
%  \mathsf{`00'}\\
%\end{array}
%\right] ,   \\
%\end{array}
%\]
%\[
%\begin{array}{ccc}
% \mathsf{W[13]=W[12]}\oplus \mathsf{W[7]}; & \mathsf{W[14]= W[13]}\oplus \mathsf{W[8]};  & \mathsf{W[15]=W[14]}\oplus \mathsf{W[9]},  \\
%\end{array}
%\]
%и т.~д.
%\exampleend
%
%\example
%Пусть $\mathsf{Nk=8}.$ В этом случае ключ шифра имеет длину $256$ бита. Найдем столбцы расширенного ключа. Столбцы
%$\mathsf{W[0],W[1],W[2],W[3],W[4],W[5],W[6],W[7]}$  непосредственно заполняются битами ключа шифра. Номер следующего столбца
%$\mathsf{W[8]}$ кратен $\mathsf{Nk}$, поэтому
%\[
%\begin{array}{ccccccc}
% \mathsf{W[8]} & = & \mathsf{SubByte(RotByte(W[7]))} &\oplus  &  \mathsf{W[0]} & \oplus  & \left[ \begin{array}{c}
% \mathsf{`01'} \\
%  \mathsf{`00'}\\
%  \mathsf{`00'}\\
%  \mathsf{`00'}\\
%\end{array}
%\right]    \\
%\end{array}
%\].
%Далее имеем
%\[
%\begin{array}{ccc}
%\mathsf{ W[7]=W[6]}\oplus \mathsf{W[1]}; & \mathsf{W[8]=W[7]}\oplus \mathsf{W[2]}; & \mathsf{W[9]=W[8]}\oplus \mathsf{W[3]}; \\
%\mathsf{ W[10]=W[9]}\oplus \mathsf{W[4]}; & \mathsf{W[11]=W[10]}\oplus \mathsf{W[5]}.\\
%\end{array}
%\]
%Номер следующего столбца $\mathsf{W[12]}$ равен $12$. Так как $12-4$ кратно $\mathsf{Nk}$, то
%\[
%\begin{array}{ccc}
%\mathsf{ W[12]=SubByte(RotByte(W[11]))}\oplus \mathsf{W[4]}; & \mathsf{W[13]=W[12]}\oplus \mathsf{W[5]}; & \mathsf{W[14]=W[13]}\oplus \mathsf{W[6]}; \\
%\mathsf{ W[15]=W[14]}\oplus \mathsf{W[7]}. &  &\\
%\end{array}
%\]
%Затем
%\[
%\begin{array}{ccccccc}
% \mathsf{W[16]} & = & \mathsf{SubByte(RotByte(W[15]))} &\oplus  &  \mathsf{W[8]} & \oplus  & \left[ \begin{array}{c}
% \mathsf{\alpha} \\
%  \mathsf{`00'}\\
%  \mathsf{`00'}\\
%  \mathsf{`00'}\\
%\end{array}
%\right] ,   \\
%\end{array}
%\]
%\[
%\begin{array}{ccc}
% \mathsf{W[17]=W[16]}\oplus \mathsf{W[9]}; & \mathsf{W[18]=W[17]}\oplus \mathsf{W[10]};  &\mathsf{ W[19]=W[18]}\oplus \mathsf{W[10]}, \\
%\end{array}
%\]
%
%\[
%\begin{array}{ccc}
%\mathsf{ W[20]=SubByte(RotByte(W[19]))}\oplus \mathsf{W[12]}; & \mathsf{W[21]=W[20]}\oplus \mathsf{W[13]}; & \mathsf{W[22]=W[21]}\oplus \mathsf{W[14]}; \\
%\mathsf{ W[23]=W[22]}\oplus \mathsf{W[15]}, &  &\\
%\end{array}
%\]
%и т.~д.

Ключ $i$-го раунда состоит из столбцов матрицы расширенного ключа
\[
    \mathsf{RoundKey} = (\mathsf{W}[4(i-1)], \mathsf{W}[4(i-1) + 1], \ldots, \mathsf{W}[4i-1]).
\]
%Если длина блока равна 192 битам $Nb=6$, то ключ 5-го раунда состоит из столбцов $W[24],W[25],W[26],W[27],W[28],W[29].$
%\exampleend

В настоящее время американский стандарт шифрования AES де-факто используется во всём мире в негосударственных системах передачи данных, если позволяет законодательство страны. C 2010 г. процессоры Intel поддерживают специальный набор инструкций для шифра AES.

\index{шифр!AES|)}


\section{Режимы работы блоковых шифров}\label{chapter-block-chaining}
\selectlanguage{russian}

Открытый текст $M$, представленный как двоичный файл, перед шифрованием разбивают на части $M_1, M_2, \dots, M_n$, называемые пакетами. Предполагается, что размер в битах каждого пакета существенно превосходит длину блока шифрования, которая равна 64 бит для российского стандарта и 128 для американского стандарта AES.

В свою очередь, каждый пакет $M_i$ разбивается на блоки размера, равного размеру блока шифрования:
    \[ M_i = \left[ M_{i,1}, M_{i,2}, \dots, M_{i,n_i} \right]. \]
Число блоков $n_i$ в разных пакетах может быть разным. Кроме того, последний блок пакета $M_{i,n_i}$ может иметь размер, меньший размера блока шифрования. В этом случае для него применяют процедуру дополнения (удлинения) до стандартного размера. Процедура должна быть обратимой: после расшифрования последнего блока пакета лишние байты должны быть обнаружены и удалены. Некоторые способы дополнения:
\begin{itemize}
  \item добавить один байт со значением $128$, а остальные байты пусть будут нулевые;
  \item определить, сколько байт надо добавить к последнему блоку, например $b$ и добавить $b$ байт со значением $b$ в каждом.
\end{itemize}
В дальнейшем предполагается, что такое дополнение сделано для каждого пакета. При шифровании блоков внутри одного пакета первый индекс в нумерации блоков опускается, то есть вместо обозначения $M_{i,j}$ используется $M_j$.

Для шифрования всего открытого текста $M$ и, следовательно, всех пакетов используется один и тот же  \emph{сеансовый} ключ шифрования  $K$. Процедуру передачи одного пакета будем называть \emph{сеансом}.

Существует несколько режимов работы блоковых шифров: режим электронной кодовой книги, режим шифрования зацепленных блоков, режим обратной связи, режим шифрованной обратной связи, режим счетчика. Рассмотрим особенности каждого из этих режимов.


\subsection{Электронная кодовая книга}

В режиме электронной кодовой книги (\langen{Electronic Code Book, ECB}) открытый текст в пакете разделен на блоки
    \[ \left[ M_1, M_2, \dots, M_{n-1}, M_n \right]. \]

В процессе шифрования каждому блоку $M_j$ соответствует свой шифротекст $C_j$, определяемый с помощью ключа $K$:
    \[ C_j = E_K(M_j), ~ j = 1, 2, \dots, n. \]

Если в открытом тексте есть одинаковые блоки, то в шифрованном тексте им также соответствуют одинаковые блоки. Это даёт дополнительную информацию для криптоаналитика, что является недостатком этого режима. Другой недостаток состоит в том, что криптоаналитик может подслушивать, перехватывать, переставлять, воспроизводить ранее записанные блоки, нарушая конфиденциальность и целостность информации. Поэтому при работе в режиме электронной кодовой книги нужно вводить аутентификацию сообщений.

Шифрование в режиме электронной кодовой книги не использует сцепление блоков и синхропосылку\index{синхропосылка} (вектор инициализации)\index{вектор инициализации}. Поэтому для данного режима применима атака на различение сообщений, так как два одинаковых блока или два одинаковых открытых текста шифруются идентично.

На рис.~\ref{fig:ecb-demo} приведен пример шифрования графического файла морской звезды в формате BMP, 24 бит цветности на пиксель (рис.~\ref{fig:starfish}), блоковым шифром AES с длиной ключа 128 бит в режиме электронной кодовой книги  (рис.~\ref{fig:starfish-aes-128-ecb}). В начале зашифрованного файла был восстановлен стандартный заголовок формата BMP. Как видно, в зашифрованном файле изображение все равно различимо.
\begin{figure}[!ht]
    \centering
    \subfloat[Исходный рисунок]{\label{fig:starfish} \includegraphics[width=0.45\textwidth]{pic/starfish}}
    ~~~
    \subfloat[Рисунок, зашифрованный AES-128]{\label{fig:starfish-aes-128-ecb} \includegraphics[width=0.45\textwidth]{pic/starfish-aes-128-ecb}}
    \caption{Шифрование в режиме электронной кодовой книги\label{fig:ecb-demo}}
\end{figure}
BMP файл в данном случае содержит в самом начале стандартный заголовок (ширина, высота, количество цветов), и далее идёт массив 24-битовых значений цвета пикселей, взятых построчно сверху вниз. В массиве много последовательностей нулевых байт, так как пиксели белого фона кодируются 3 нулевыми байтами. В AES размер блока равен 16 байт и, значит, каждые $\frac{16}{3}$ подряд идущих пикселей белого фона шифруются одинаково, позволяя различить изображение в зашифрованном файле.

%На рис.~\ref{fig:ecb-demo} приведен пример шифрования графического файла логотипа Википедии в формате BMP, 24 бит цветности на пиксель (рис.~\ref{fig:wikilogo}), блоковым шифром AES с длиной ключа 128 бит в режиме электронной кодовой книги  (рис.~\ref{fig:wikilogo-aes-128-ecb}). В начале зашифрованного файла был восстановлен стандартный заголовок BMP формата. Как видно, на зашифрованном рисунке возможно даже прочитать надпись.
%\begin{figure}[!ht]
%    \centering
%    \subfloat[Исходный рисунок]{\label{fig:wikilogo}\includegraphics[width=0.45\textwidth]{pic/wikilogo}}
%    ~~~
%    \subfloat[Рисунок, зашифрованный AES-128]{\label{fig:wikilogo-aes-128-ecb}\includegraphics[width=0.45\textwidth]{pic/wikilogo-aes-128-ecb}}
%    \caption{Шифрование в режиме электронной кодовой книги.}
%    \label{fig:ecb-demo}
%\end{figure}

%Возможно воссоздание структуры информации -- например, пингвин на рис.~\ref{fig:tux-ecbmode}. Картинка с пингвином записана в формате BMP и зашифрована DES в режиме электронной кодовой книги.
%\begin{figure}[!ht]
%    \centering
%    \includegraphics[width=0.3\textwidth]{pic/tux-ecb}
%    \caption{Картинка с пингвином, зашифрованная в режиме электронной кодовой книги.}
%    \label{fig:tux-ecbmode}
%\end{figure}


\subsection{Сцепление блоков шифротекста}

В режиме сцепления блоков шифротекста (\langen{Cipher Block Chaining, CBC}) перед шифрованием текущего блока открытого текста предварительно производится его суммирование по модулю 2 с предыдущим блоком зашифрованного текста, что и осуществляет <<сцепление>> блоков. Процедура шифрования имеет вид
\[ \begin{array}{l}
    C_1 = E_K(M_1 \oplus C_0), \\
    C_j = E_K(M_j \oplus C_{j-1}), ~ j = 1, 2, \dots,  n,
\end{array} \]
где $C_0 = \textrm{IV}$ --  вектор, называемый вектором инициализации (обозначение $\textrm{IV}$ от Initialization Vector). Другое название -- синхропосылка.

Благодаря сцеплению, \emph{одинаковым} блокам открытого текста соответствуют \emph{различные} шифрованные блоки. Это затрудняет криптоаналитику статистический анализ потока шифрованных блоков.

На приемной стороне расшифрование осуществляется по правилу
\[ \begin{array}{l}
    D_K(C_j) = M_j \oplus C_{j-1}, ~ j=1, 2, \dots, n,\\
    M_{j} = D_K(C_j) \oplus C_{j-1}.
\end{array} \]

Блок $C_0 = \textrm{IV}$ должен быть известен легальному получателю шифрованных сообщений. Обычно криптограф выбирает его случайно и вставляет на первое место в поток шифрованных блоков. Сначала передают блок $C_0$, а затем шифрованные блоки $C_1, C_2, \ldots, C_n$.

В разных пакетах блоки $C_0$ должны выбираться независимо. Если их выбрать одинаковыми, то возникают проблемы, аналогичные проблемам в режиме ECB. Например, часто первые нешифрованные блоки $M_1$ в разных пакетах бывают одинаковыми. Тогда одинаковыми будут и первые шифрованные блоки.

Однако случайный выбор векторов инициализации также имеет свои недостатки. Для выбора такого вектора необходим хороший генератор случайных чисел. Кроме того, каждый пакет удлиняется на один блок.

Нужны такие процедуры выбора $C_0$ для каждого сеанса передачи пакета, которые известны криптографу и легальному пользователю. Одним из решений является использование так называемых \emph{одноразовых меток}. Каждому сеансу присваивается уникальное число. Его уникальность состоит в том, что оно используется только один раз и никогда не должно повторяться в других пакетах. В англоязычной научной литературе оно обозначается как \emph{Nonce}, то есть сокращение от <<Number used once>>\index{одноразовая метка}.

Обычно одноразовая метка состоит из номера сеанса и дополнительных данных, обеспечивающих уникальность. Например, при двустороннем обмене шифрованными сообщениями одноразовая метка может состоять из номера сеанса и индикатора направления передачи. Размер одноразовой метки должен быть равен размеру шифруемого блока. После определения одноразовой метки $\textrm{Nonce}$ вектор инициализации вычисляется в виде
    \[ C_0 = \textrm{IV} = E_K(\textrm{Nonce}). \]

Этот вектор используется в данном сеансе для шифрования открытого текста в режиме CBC. Заметим, что блок $C_0$ передавать в сеансе не обязательно, если приемная сторона знает заранее дополнительные данные для одноразовой метки. Вместо этого достаточно вначале передать только номер сеанса в открытом виде. Приемная сторона добавляет к нему дополнительные данные и вычисляет блок $C_0$, необходимый для расшифрования в режиме CBC. Это позволяет сократить издержки, связанные с удлинением пакета. Например, для шифра AES длина блока $C_0$ равна $16$ байт. Если число сеансов ограничить величиной $2^{32}$ (вполне приемлемой для большинства приложений), то для передачи номера пакета понадобится только $4$ байта.


\subsection{Обратная связь по выходу}

В предыдущих режимах входными блоками для устройств шифрования были непосредственно блоки открытого текста.
В режиме обратной связи по выходу (OFB от Output FeedBack) блоки открытого текста непосредственно на вход устройства шифрования не поступают. Вместо этого устройство шифрования генерирует псевдослучайный поток байт, который суммируется по модулю $2$ с открытым текстом для получения шифрованного текста. Шифрование осуществляют по правилу:
\[ \begin{array}{l}
    K_0 = \textrm{IV}, \\
    K_j = E_K(K_{j-1}), ~ j = 1, 2, \dots, n, \\
    C_j = K_j \oplus M_j.
\end{array} \]

Здесь текущий ключ $K_j$ есть результат шифрования предыдущего ключа $K_{j-1}$. Начальное значение $K_0$ известно криптографу и легальному пользователю. На приемной стороне расшифрование выполняют по правилу:
\[ \begin{array}{l}
    K_0 = \textrm{IV}, \\
    K_j = E_K(K_{j-1}), ~ j = 1, 2, \dots, n, \\
    M_j = K_j \oplus C_j.
\end{array} \]

Как и в режиме CBC, вектор инициализации $\textrm{IV}$ может быть выбран случайно и передан вместе с шифрованным текстом, либо вычислен на основе одноразовых меток. Здесь особенно важна уникальность вектора инициализации.

Достоинство этого режима состоит в полном совпадении операций шифрования и расшифрования. Кроме того, в этом режиме не надо проводить операцию дополнения открытого текста.


\subsection{Обратная связь по шифрованному тексту}

В режиме обратной связи по шифрованному тексту (CFB от Cipher FeedBack) ключ $K_j$ получается с помощью процедуры шифрования предыдущего шифрованного блока $C_{j-1}$. Может быть использован не весь блок $C_{j-1}$, а только часть его. Как и в предыдущем случае, начальное значение ключа $K_0$ известно криптографу и легальному пользователю:
\[ \begin{array}{l}
    K_0 = \textrm{IV}, \\
    K_j = E_K(C_{j-1}), ~ j = 1, 2, \dots, n,\\
    C_j = K_j \oplus M_j.
\end{array} \]

У этого режима нет особых преимуществ по сравнению с другими режимами.


\subsection{Счетчик}

В режиме счетчика (CTR от Counter) правило шифрования имеет вид, похожий на режим обратной связи по выходу (OFB), но позволяющий вести независимое (параллельное) шифрование и расшифрование блоков:
\[ \begin{array}{l}
    K_j = E_K(\textrm{Nonce} \| j - 1), ~ j = 1, 2, \dots, n, \\
    C_j = M_j \oplus K_j,
\end{array} \]
где $\textrm{Nonce} \| j - 1$ -- конкатенация битовой строки одноразовой метки $\textrm{Nonce}$ и номера блока уменьшенного на единицу $j-1$.
%Для стандарта AES значение $\textrm{Nonce}$ занимает 16 бит, номер блока -- 48 бит. С одним ключом выполняется шифрование $2^{48}$ блоков.

Правило расшифрования идентичное:
\[ \begin{array}{l}
    M_j = C_j \oplus K_j. \\
\end{array} \]


\section{Некоторые свойства блоковых шифров}

\input{feistel_network_reversibility}

\subsection{Схема Фейстеля без s-блоков}
\selectlanguage{russian}

Пусть функция $F$ является простой линейной комбинацией некоторых битов правой части и ключа раунда относительно операции XOR. Тогда можно записать систему линейных уравнений битов выхода $y_i$ относительно битов входа $x_i$ и ключа $k_i$ после всех 16 раундов в виде:
    \[ y_i = (\sum_{i=0}^{n_1} a_i x_i) \oplus (\sum_{i=0}^{n_2} b_i k_i), \],
где суммирование производится по модулю 2, коэффициенты $a_i$ и $b_i$ известны и равны 0 или 1, количество битов в блоке открытого текста равно $n_1$, количество битов ключа равно $n_2$.

Имея открытый текст и шифротекст, легко найти ключ. Без знания открытых текстов, выполняя XOR шифротекстов, найдем XOR открытых текстов. Во-первых, это атака на различение сообщений. Во-вторых, часто известны форматы сообщений, отдельные поля или распределение символов открытого текста, что приводит к атаке перебором с учетом множества уравнений, полученных XOR шифротекстов.

$s$-блоки замены создают нелинейность в уравнениях выхода $y_i$ относительно сообщения и ключа.


\subsubsection[Схема Фейстеля в ГОСТ 28147-89 без $s$-блоков]{Схема Фейстеля в~ГОСТ~28147-89 без~$s$-блоков}

В отличие от устаревшего алгоритма DES, блоковый шифр ГОСТ без $s$-блоков намного сложнее для взлома, так как для него нельзя записать систему линейных уравнений:
\[
    \begin{array}{l}`
        L_1 = R_0, \\
        R_1 = L_0 \oplus ((R_0 \boxplus K_1) \lll 11), \\
    \end{array}
\] \[
    \begin{array}{l}
        L_2 = R_1 = L_0 \oplus ((R_0 \boxplus K_1) \lll 11), \\
        R_2 = L_1 \oplus (R_1 \boxplus K_2)  = \\
        ~~~~~= R_0 \oplus (((L_0 \oplus ((R_0 \boxplus K_1) \lll 11)) \boxplus K_2) \lll 11). \\
    \end{array}
\]

Операция $\boxplus$ нелинейна по XOR. Например, только на трех операциях $\oplus$, $\boxplus$ и $\lll f(R_i)$ без использования $s$-блоков построен блоковый шифр RC5, который по состоянию на 2010 г. не был взломан.


\subsection{Лавинный эффект}
\selectlanguage{russian}

\subsubsection{Лавинный эффект в DES}

Оценим число раундов, за которое в DES достигается полный лавинный эффект\index{лавинный эффект}, предполагая \emph{случайное} расположение бит перед расширением, $s$-блоками ($s$ -- substitute, блоки замены) и XOR.

Пусть на входе правой части $R_i$ содержится $r$ бит, на которые уже распространилось влияние одного бита, выбранного вначале. После расширения получим
    \[ n_1 \approx \min(1.5 \cdot r, 32) \]
зависимых бит. Предполагая случайные попадания в 8 $s$-блоков, мы увидим, что, согласно задаче о размещении, биты попадут в
    \[ s_2 = 8 \left( 1 - \left( 1 - \frac{1}{8} \right)^{n_1} \right) \approx 8 \left( 1 - e^{-\frac{n_1}{8}} \right) \]
$s$-блоков. Одно из требований NSA к $s$-блокам заключалось в том, чтобы изменение каждого бита входа \emph{изменяло} 2 бита выхода. Мы предположим, что каждый бит входа $s$-блока \emph{влияет} на все 4 бита выхода. Зависимыми станут
    \[ n_2 = 4 \cdot s_2 = 32 \left( 1 - e^{-\frac{n_1}{8}} \right) \]
бит. При дальнейшем XOR с величиной $L_i$, содержащей $l$ зависимых бит, результатом будет
    \[ n_3 \approx n_2 + l  - \frac{n_2 l}{32} \]
зависимых бит.

\begin{table}[!ht]
    \centering
    \caption{Распространение влияния 1 бита левой части в DES\label{tab-DES-avalance-effect}}
    \begin{tabular}{||c||c||c|c|c||}
        \hline
        \multirow{3}{*}{Раунд} & $L_i$ & \multicolumn{3}{|c||}{$R_i$} \\
        \cline{2-5}
        & & Расширение & $s$-блоки & $R_{i+1} = f(R_i) \oplus L_i$ \\
        & $l$ & $r \rightarrow n_1$ & $n_1 \rightarrow n_2$ & $(n_2, l) \rightarrow n_3$ \\
        \hline \hline
        0 & 1 & 0 & 0 & 0 \\
        1 & 0 & 0 & 0 & $(0,1) \rightarrow 1$ \\
        2 & 1 & $1 \rightarrow 1.5$ & $1.5 \rightarrow 5.5$ & $(5.5, 0) \rightarrow 5.5$ \\
        3 & 5.5 & $5.5 \rightarrow 8.2$ & $8.2 \rightarrow 20.5$ & $(20.5, 1) \rightarrow 20.9$ \\
        4 & 20.9 & $20.9 \rightarrow 31.3$ & $31.3 \rightarrow 32$ & $(32, 20.9) \rightarrow 32$ \\
        5 & 32 & 32 & 32 & 32 \\
      \hline
    \end{tabular}
\end{table}

В таблице~\ref{tab-DES-avalance-effect} приводится расчет распространения 1 бита левой части. Посчитано число зависимых бит по раундам в предположении об их случайном расположении и том, что каждый бит на входе $s$-блока \emph{влияет} на все биты выхода. Полная диффузия достигается за 5 раундов, что совпадает с экспериментальной проверкой. Для достижения максимального лавинного эффекта требуется аккуратно выбрать расширение, $s$-блоки, а также перестановку в функции $F$.


\subsubsection{Лавинный эффект в ГОСТ 28147-89}

Лавинный эффект\index{лавинный эффект} по входу обеспечивается $(4 \times 4)$ $s$-блоками и циклическим сдвигом влево на $11 \neq 0 \mod 4$.

\begin{table}[!ht]
    \centering
    \caption{Распространение влияния 1 бита левой части в ГОСТ 28147-89\label{tab:GOST-avalance-effect}}
    \resizebox{\textwidth}{!}{ \begin{tabular}{||c||c|c|c|c|c|c|c|c||c|c|c|c|c|c|c|c||}
        \hline
        \multirow{2}{*}{Раунд} & \multicolumn{8}{|c||}{$L_i$} & \multicolumn{8}{|c||}{$R_i$} \\
        \cline{2-17}
              & 1 & 2 & 3 & 4 & 5 & 6 & 7 & 8   &   1 & 2 & 3 & 4 & 5 & 6 & 7 & 8 \\
        \hline \hline
        0     &   &   &   &   &   &   &   & 1   &     &   &   &   &   &   &   &   \\
        1     &   &   &   &   &   &   &   &     &     &   &   &   &   &   &   & 1 \\
        2     &   &   &   &   &   &   &   & 1   &     &   &   &   & 3 & 1 &   &   \\
        3     &   &   &   &   & 3 & 1 &   &     &     & 3 & 4 & 1 &   &   &   & 1 \\
        4     &   & 3 & 4 & 1 &   &   &   & 1   &   4 & 1 &   &   & 3 & 1 & 3 & 4 \\
        5     & 4 & 1 &   &   & 3 & 1 & 3 & 4   &     & 3 & 4 & 4 & 4 & 4 & 4 & 1 \\
        6     &   & 3 & 4 & 4 & 4 & 4 & 4 & 1   &   4 & 4 & 4 & 4 & 4 & 3 & 3 & 4 \\
        7     & 4 & 4 & 4 & 4 & 4 & 3 & 3 & 4   &   4 & 4 & 4 & 4 & 4 & 4 & 4 & 4 \\
        8     & 4 & 4 & 4 & 4 & 4 & 4 & 4 & 4   &   4 & 4 & 4 & 4 & 4 & 4 & 4 & 4 \\
      \hline
    \end{tabular} }
\end{table}

Из таблицы~\ref{tab:GOST-avalance-effect} видно, что на каждом раунде число зависимых битов увеличивается, в среднем, на 4 в результате сдвига и попадания выхода $s$-блока предыдущего раунда в два $s$-блока следующего раунда. Показано распространение зависимых битов в группах по 4 бита в левой и правой частях без учета сложения с ключом раунда. Предполагается, что каждый бит на входе $s$-блока влияет на все биты выхода. Число раундов для достижения полного лавинного эффекта без учета сложения с ключом -- 8. Экспериментальная проверка для $s$-блоков, используемых Центробанком РФ, показывает, что требуется 8 раундов.


\subsubsection{Лавинный эффект в AES}

В первом раунде один бит оказывает влияние на один байт в операции <<замена байт>> и затем на столбец из четырех байт в операции <<перемешивание столбцов>>\index{лавинный эффект}.

Во втором раунде операция <<сдвиг строк>> сдвигает байты измененного столбца на разное число байт по строкам, в результате этого получаем диагональное расположение измененных байт, то есть в каждой строке присутствует по измененному байту. Далее, в результате операции <<перемешивание столбцов>> изменение распространяется от байта в столбце на весь столбец и, следовательно, на всю матрицу.

Диффузия по входу достигается за 2 раунда.


\subsection{Двойное и тройное шифрования}\index{шифрование!двойное}\index{шифрование!тройное}
\selectlanguage{russian}

В конце XX-века, когда ненадёжность существующего стандарта DES\index{шифр!DES} уже была очевидна, а нового стандарта ещё не было, стали распространены техники двойного и тройного шифрования, когда к одному блоку текста последовательно применяется несколько преобразований на разных ключах.

Например, двойное шифрование\index{шифрование!двойное} 2DES\index{шифр!2DES} использует два разных ключа $K_1$ и $K_2$ для шифрования одного блока текста дважды:

\[ E_{K1, K2} \left( M \right) \equiv E_{K1} \left( E_{K2} \left( M \right) \right). \]

Так как функция шифрования\index{функция!шифрования} DES не образует группу\index{группа} (\cite{Kaliski:Rivest:Sherman:1988, Campbell:Wiener:1993}), то данное преобразование не эквивалентно однократному шифрованию на каком-нибудь третьем ключе. То есть для произвольных $K_1$ и $K_2$ нельзя подобрать такой $K_3$, что:

\[E_{K1} \left( E_{K2} \left( M \right) \right) \equiv E_{K3} \left( M \right).\]

Тем самым размер ключевого пространства (количество различных ключей шифрования, если считать за ключ пару $K_1$ и $K_2$) увеличивается с $2^{56}$ до $2^{112}$ (без учёта проверочных бит). Однако, из-за атаки <<встреча посередине>>\index{атака!встреча посередине} (\langen{meet in the middle}) фактическая криптостойкость увеличилась не более, чем до $2^{57}$.

Тройной DES (\langen{triple DES, 3DES})\index{шифрование!тройное}\index{шифр!3DES} использует тройное преобразование. Причём в качестве второй функции используется функция \emph{расшифрования}:

\[ E_{K1, K2, K3} \left( M \right) \equiv E_{K1} \left( D_{K2} \left( E_{K3} \left( M \right) \right) \right). \]

\begin{itemize}
	\item Вариант $K_1 \neq K_2 \neq K_3$ является наиболее защищённым, ключевое пространство увеличивается до $2^{168}$;
	\item Вариант $K_1 \neq K_2$, $K_1 = K_3$ увеличивает ключевое пространство до $2^{112}$, но защищён от атаки <<встреча посередине>>\index{атака!встреча посередине}, в отличие от 2DES\index{шифр!2DES};
	\item Вариант $K_1 = K_2 = K_3$ эквивалентен однократному преобразованию DES\index{шифр!DES}. Его можно использовать для целей совместимости.
\end{itemize}

Оценим сложность атак на 2DES\index{шифр!2DES} и 3DES\index{шифр!3DES}.

\subsubsection{Атака на двойное шифрование}

%Для упрощения записи введём обозначение последовательного шифрования $E_{K_1}( E_{K_2}( \dots E_{K_n}(M) \dots)) \equiv E{K_1} \circ E_{K_2} (M)$ и назовём его суперпозицией функций $E_{K_i}$.

Атака основана на предположении, что у криптоаналитика есть возможность получить шифротекст для любого открытого текста\index{атака!с известным шифротекстом} (Chosen Plaintext Attack, CPA), либо открытый текст по шифротексту\index{атака!с известным открытым текстом} (Chosen Ciphertext Attack, CCA), но неизвестен ключ шифрования, который и нужно найти.

Шифрование в 2DES\index{шифр!2DES}:
    \[ C = E_{K_1}( E_{K_2}(M)). \]
Запишем $D_{K_1}(C) = E_{K_2}(M)$. Пусть время одного шифрования $T_E$, время одного сравнения блоков $T_{=} \approx 2^{-10} T_E$.

Атака для нахождения ключей без использования памяти занимает время
    \[ T = 2^{56 + 56} (T_E + T_{=}) \approx 2^{112} T_E. \]

Можно заранее вычислить значения $E_{K_2}(M)$ для всех ключей и построить таблицу: индекс -- $E_{K_2}(M)$, значения поля -- набор ключей $K_2$, которые соответствуют этому значению. Атака для нахождения ключей требует времени
    \[ T = 2 \cdot 2^{56} T_E + 2^{56} T_{=} \approx 2^{57} T_E \]
и памяти $M = 56 \cdot 2^{56} \approx 2^{64}$ бит $= 26$ GiB, учитывая прямой доступ по значению к возможным ключам. При нахождении соответствия берётся другая пара (открытый текст, шифротекст) и проверяется равенство для определения правильные ключи или нет.

По отношению к CCA и CPA криптостойкость 2DES\index{шифр!2DES} эквивалентна обычному DES\index{шифр!DES} с использованием 26 GiB памяти.

\subsubsection{Атака на тройное шифрование}

Атака для нахождения ключей (CCA\index{атака!с известным шифротекстом}, CPA\index{атака!с известным открытым текстом}) на наиболее стойкий вариант 3DES\index{шифр!3DES} (все три ключа $K_1$, $K_2$ и $K_3$ выбираются независимо) требует время $T \approx 2^{168} T_E$ без использования дополнительной памяти.

Для построения таблицы запишем
    \[ D_{K_2}( D_{K_1}( C)) = E_{K_3} (M). \]
Таблица строится аналогично 2DES\index{шифр!2DES} для $E_{K_3}(M)$. С использованием памяти атака занимают время $T = 2^{112} T_E$ и память $M = 26$ GiB.


\index{шифр!блоковый|)}

\input{stream-ciphers}

\input{hash-functions}

\input{public-key}

\chapter{Распространение ключей}\index{протокол!распространения ключей}
\selectlanguage{russian}

Задачей распространения ключей между двумя пользователями является создание секретных псевдослучайных сеансовых ключей шифрования и аутентификация сообщений. Пользователи предварительно создают и обмениваются ключами аутентификации один раз. В дальнейшем для создания защищённой связи пользователи производят взаимную аутентификацию и вырабатывают сеансовые ключи\index{ключ!сеансовый}.

\section[Трёхэтапный протокол Шамира]{Трёхэтапный протокол Шамира на коммутативных шифрах}
\selectlanguage{russian}

Предположим, что две стороны $A$ и $B$ соединены незащищённым каналом связи. Каждая из этих сторон имеет свой секретный ключ: $A$ имеет ключ $K_A$, $B$ имеет ключ $K_B$. Сторона $A$ должна создать общий секретный ключ $K$ и передать стороне $B$.

Для решения этой задачи используют трёхэтапный протокол Шамира с тремя <<замками>>. \textbf{Протокол Шамира}\index{протокол!Шамира} построен на \emph{коммутативных} функциях шифрования, для которых выполняется условие:
    \[ E_{K_{B}} (E_{K_{A}}(K))=E_{K_{A}} (E_{K_{B}}(K)). \]

Протокол предполагает следующие процедуры.
\begin{enumerate}
    \item $A$ создаёт секретный ключ $K$, шифрует его своей системой шифрования с помощью своего ключа $K_A$ и посылает сообщение стороне $B$:
        \[ A \rightarrow B: ~ E_{K_A}(K). \]
    \item $B$ получает это сообщение, шифрует его с помощью своего ключа $K_B$ и посылает сообщение стороне $A$:
        \[ A \leftarrow B: ~ E_{K_B}( E_{K_A}( K)). \]
    \item Сторона $A$, получив сообщение $E_{K_B}(E_{K_A}(K))$, использует свой секретный ключ $K_A$ для расшифрования:
            \[ D_{K_A}(E_{K_B} (E_{K_A}(K))) = E_{K_B}(K). \]
        Сторона $A$ передаёт стороне $B$ сообщение:
        \[ A \rightarrow B: ~ E_{K_B}(K). \]
    \item Сторона $B$, получив сообщение $E_{K_B}(K)$, использует свой секретный ключ $K_B$ для расшифрования:
            \[ D_{K_B}(E_{K_B}(K)) = K. \]
        В результате стороны получают общий секретный ключ $K$.
\end{enumerate}

Приведём пример неудачного шифрования с использованием коммутативных функций.

\begin{enumerate}
    \item $A$ имеет функцию шифрования совершенной секретности $E_{K_A}(K) = K \oplus K_A$, где $K_A$ -- двоичная последовательность с равномерным распределением символов. $A$ посылает это сообщение стороне $B$:
            \[ A \rightarrow B: ~ E_{K_A}(K) = K \oplus K_A. \]
    \item $B$ использует такую же функцию шифрования совершенной секретности с ключом $K_B$ (двоичная последовательность с равномерным распределением символов). $B$ шифрует полученное сообщение и отправляет $A$:
            \[ A \leftarrow B: ~ E_{K_A}(K) \oplus K_B = K \oplus K_A \oplus K_B. \]
    \item Сторона $A$, получив сообщение $K \oplus K_A \oplus K_B$, выполняет расшифрование:
            \[ K \oplus K_A \oplus K_B \oplus K_A = K \oplus K_B. \]
        Сторона $A$ передаёт стороне $B$ сообщение:
            \[ A \rightarrow B: ~ K \oplus K_B. \]
    \item Сторона $B$, получив сообщение $K \oplus K_B$, выполняет расшифрование:
            \[ K \oplus K_B \oplus K_B = K. \]
        Обе стороны получают общий секретный ключ $K$.
\end{enumerate}

Предложенный выбор коммутативной функции шифрования совершенной секретности был назван неудачным, так как существуют ситуации, при которых криптоаналитик может определить ключ $K$. Предположим, что криптоаналитик перехватил все три сообщения:
    \[ K \oplus K_A, ~~ K \oplus K_A \oplus K_B, ~~ K \oplus K_B. \]
Сложение по модулю 2 всех трёх сообщений даёт ключ $K$. Поэтому такая система шифрования не применяется.

Теперь приведём протокол надёжной передачи секретного ключа, основанный на экспоненциальной (коммутативной) функции шифрования. Стойкость этого протокола связана с трудной задачей -- задачей вычисления дискретного логарифма: известны значения $y, g, p$, найти $x$ в уравнении $y = g^x \mod p$.

\textbf{Протокол Шамира распространения ключей}
Выбирают большое простое\index{число!простое} число $p\sim 2^{1024}$ и используют его как открытый ключ.

\begin{enumerate}
    \item Сторона $A$ задаёт общий секретный ключ $K <p$ и выбирает целое число $a$, взаимно простое с $p-1$. $A$ вычисляет и посылает сообщение стороне $B$:
            \[ A \rightarrow B: ~ K^a \mod p. \]
        Существует число $c$ такое, что $a c =1 \mod (p-1)$, то есть $a c = 1 + l (p-1)$, где $l$ -- целое число. Число $c$ будет использовано стороной $A$ на следующем этапе;
    \item Сторона $B$ выбирает целое число $b$, взаимно простое с $p-1$. Используя полученное сообщение, вычисляет и посылает сообщение стороне $A$:
            \[ A \leftarrow B: ~ (K^a)^b \mod p. \]
    \item Сторона $A$, получив сообщение, вычисляет
        \[ \left( K^{ab} \right)^c = K^{1 + l (p-1) b} = K^b \cdot K^{l (p-1) b} = K^b \mod p. \]
        Здесь применена малая теорема Ферма\index{теорема!Ферма малая}: $K^{p-1} = 1 \mod p$, поэтому $\left( K^{p-1} \right)^{lb} = 1 \mod p$.
        $A$ посылает $B$ сообщение:
            \[ A \rightarrow B: ~ K^b \mod p. \]
    \item Сторона $B$, получив сообщение $K^{b}\mod p$, вычисляет
        \[ (K^b \mod p)^d = K^{bd} \mod p = K. \]
\end{enumerate}

Теперь проверим криптостойкость этого протокола. Предположим, что криптоаналитик перехватил три сообщения:
\[ \begin{array}{l}
    y_1 = K^a \mod p, \\
    y_2 = K^{ab} \mod p, \\
    y_3 = K^b \mod p. \\
\end{array} \]
Чтобы найти ключ $K$, надо решить систему из этих трёх уравнений, что имеет очень большую вычислительную сложность, неприемлемую с практической точки зрения, если все три числа $a, b, ab$ достаточно велики. Предположим, что $a$ (или $b$) мало. Тогда, вычисляя последовательные степени $y_3$ (или $y_1$), можно найти  $a$ (или $b$), сравнивая результат с $y_2$. Зная $a$, легко найти $a^{-1}\mod(p-1)$ и $K=(y_1)^{a^{-1}}\mod p$.

Недостатком этого протокола является отсутствие аутентификации сторон. Следовательно, нужно дополнительно использовать цифровую подпись при передаче сообщения.


\section{Симметричные протоколы}

\subsection{Аутентификация и атаки воспроизведения}

Рассмотрим такую ситуацию: обе стороны $A$ и $B$ имеют общий долговременный ключ $K_{AB}$ и симметричную систему шифрования. Нужно выработать сеансовый секретный ключ $K$. Сторона $A$ создаёт ключ $K$ и желает его передать стороне $B$.

\begin{enumerate}
    \item Для этого сторона $A$ с помощью общего ключа $K_{AB}$ создаёт и передаёт $B$ шифрованное сообщение:
            \[ A \rightarrow B: ~ E_{K_{AB}}(K, B, A). \]
        В этом сообщении имеются так называемые поля -- $(B,A)$ -- информация для дополнительного подтверждения.
    \item Сторона $B$, используя общий ключ $K_{AB}$, расшифровывает полученное сообщение:
            \[ D_{K_{AB}}( E_{K_{AB}}( K, B, A)) = (K, B, A). \]
        В результате сторона $B$ получает сеансовый ключ $K$ и дополнительные данные $(B,A)$.
\end{enumerate}

Недостаток этого протокола состоит в том, что криптоаналитик может перехватывать сообщения и через некоторое время переслать их стороне $A$.

Рассмотрим другие варианты решения задачи о передаче сеансового ключа.
Задача остаётся прежней: обе стороны $A$ и $B$ имеют общий долговременный секретный ключ $K_{AB}$, сторона $A$ должна выработать сеансовый секретный ключ $K$ и доставить стороне $B$.

Протокол включает \textbf{метки времени} -- информацию о моменте $t_A$ отправки сообщения и моменте получения сообщения $t_B$.

\begin{enumerate}
    \item Сторона $A$ вырабатывает $K$ и с помощью долговременного ключа $K_{AB}$ создаёт шифрованное сообщение с меткой времени $t_A$ и передаёт его стороне $B$:
            \[ A \rightarrow B: ~ E_{K_{AB}}(K, t_A). \]
    \item Сторона $B$ получает сообщение и расшифровывает его с помощью общего ключа:
            \[ D_{K_{AB}}( E_{K_{AB}}( K, t_A)) = (K, t_A). \]
        В результате $B$ получает $(K, t_A)$, то есть секретный ключ и метку времени. $B$ измеряет время прихода $t_B$ и интервал запаздывания. Если $|t_B - t_A| \le \delta$, то $B$ аутентифицирует $A$.
\end{enumerate}
Метка времени является одноразовой меткой и защищает от атак воспроизведения ранее записанных сообщений.

Рассмотрим другой способ передачи ключа с дополнительной информацией в виде \textbf{одноразовых случайных меток} (nonce -- number used once) вместо меток времени. Протокол передачи состоит в следующем:

\begin{enumerate}
    \item Сторона $A$ вырабатывает случайное число $r_A$, шифрует сообщение, в котором  $(r_A, A)$ -- реквизиты $A$, и передаёт его стороне $B$:
            \[ A \rightarrow B: ~ E_{K_{AB}}(r_A, A). \]
    \item Сторона $B$ вырабатывает сеансовый ключ $K$, создаёт шифрованное сообщение и посылает его $A$:
            \[ A \leftarrow B: ~ E_{K_{AB}}(K, r_A, A). \]
    \item Сторона $A$ расшифровывает полученное сообщение:
            \[ D_{K_{AB}}( E_{K_{AB}}( K, r_A, A)) = (K, r_A, A). \]
        В результате $A$ получает сеансовый ключ и подтверждение своих реквизитов, что является дополнительной аутентификацией.
\end{enumerate}

Предположим, что сторона $B$ тоже желает убедиться, что имеет дело со стороной $A$. Тогда этот протокол следует дополнить передачей реквизитов $B$. По-прежнему считаем, что у $A$ и $B$ -- общая система шифрования с долговременным секретным ключом $K_{AB}$.

\begin{enumerate}
    \item Сторона $A$ вырабатывает случайное число $r_A$, шифрует и передаёт стороне $B$ сообщение, в котором  $(r_A, A)$ -- реквизиты $A$:
            \[ A \rightarrow B: ~ E_{K_{AB}}(r_A, A). \]
    \item Сторона $B$ вырабатывает случайное число $r_B$ и отправляет стороне $A$ зашифрованное сообщение:
            \[ A \leftarrow B: ~ E_{K_{AB}}(K_B, r_B, r_A, A), \]
        где $K_B$ -- ключ $B$.
     \item Сторона $A$ осуществляет расшифрование:
            \[ D_{K_{AB}}(E_{K_{AB}}(K_B, r_B, r_A, A)) = (K_B, r_B, r_A, A) \]
        и получает ключ $K_B$ и реквизиты $r_B, r_A, A$. Для аутентификации себя сторона $A$ создаёт свой ключ $K_A$ и отправляет стороне $B$ шифрованное сообщение:
            \[ A \rightarrow B: ~ E_{K_{AB}}(K_A, r_B, r_A, B). \]
     \item Сторона $B$ осуществляет расшифрование
            \[ D_{K_{AB}}(E_{K_{AB}}(K_A, r_B, r_A, B)) = (K_A, r_B, r_A, B), \]
        которое определяет ключ $K_A$ и аутентифицирует $A$.
\end{enumerate}

Таким образом, обе стороны имеют в своем распоряжении ключи $K_A, K_B$ в качестве сеансовых секретных ключей.

\subsection{Протокол с ключевым кодом аутентификации}

При использовании хэш-функции $K = h(K_{A} \| K_{B})$ происходит усиление секретности. Здесь $(K_{A} \| K_{B})$ -- конкатенация $K_{A} $ и $K_{B}$.

% Достоинства: предположим, $K_{A} ,K_{B} $ -- не обладают «хорошими» свойствами случайности (биты распределены неравномерно или зависимы друг от друга), т.~е., $P_{K_{A} ,K_{B} } (0)=\frac{1}{2} -\varepsilon $, где $\varepsilon $ - мало, но не 0. Тогда вероятность того, что этот бит в \textit{K }будет равным нулю, $P_{K} (0)=\frac{1}{2} -\varepsilon ',\varepsilon '<\varepsilon $- усиление секретности.

Вычисление хэш-значения, как правило, выполняется быстрее, чем расшифрование. Поэтому были разработаны протоколы, в которых вместо функции шифрования используется имитовставка\index{имитовставка} на основе хэш-функции ($\MAC_K$). Рассмотрим протокол такого рода:
\begin{enumerate}
    \item Сторона $A$ вырабатывает сеансовый ключ $K$, использует одноразовую метку $t_{A}$, создаёт и пересылает стороне $B$ сообщение:
            \[ A \rightarrow B: ~ t_A, ~ B, ~ K \oplus \MAC_{K_{AB}}( t_A, B), ~ \MAC_{K_{AB}}(K, t_A, B). \]
    \item Сторона $B$ вычисляет
            \[ \MAC_{K_{AB}}(t_A, B) \oplus K \oplus \MAC_{K_{AB}}(t_A, B) = K \]
        и получает сеансовый ключ $K$.
\end{enumerate}

Заметим, что криптоаналитик может добавить в поле случайную последовательность, тогда вместо $K$ получаем <<$K$ плюс помеха>>. Вмешательство криптоаналитика будет выявлено благодаря наличию четвертого поля в сообщении. Используя полученное значение $K$, вычисляют $\MAC_{K_{AB}}(K, t_A, B)$ и сравнивают с четвертым полем. Если совпадает, то вмешательства криптоаналитика не было.

\subsection{Протокол Нидхема~---~Шрёдера с доверенным центром}\index{протокол!Нидхема~---~Шрёдера}
\selectlanguage{russian}

Рассмотрим ситуацию, когда в сети имеется некоторый надёжный (доверенный) сервер (центр) $T$, которому доверяют все пользователи сети. Сервер для работы с абонентами сети использует некоторую систему шифрования $E_S(*)$, где ключ $S=K_{AT}$  известен только $A$ и $T$, но неизвестен остальным участникам сети, $S = K_{BT}$ известен только $B$ и  $T$. Предполагаем, что сервер имеет хороший генератор случайных чисел. Сеансовый ключ сервер вырабатывает по запросу. Стороны $A$ и $B$ могут выбирать разные одноразовые метки.

Приведём в качестве примера упрощенную версию известного \textbf{протокола Нидхема~---~Шрёдера} (Needham~---~Schroeder) с симметричным шифром:
\begin{enumerate}
    \item Сторона $A$ передаёт серверу $T$ реквизиты сторон $A$ и $B$  и некую одноразовую метку $N_A$, которая может быть, например, меткой времени или случайным (одноразовым) числом, что оговаривается заранее:
            \[ A \rightarrow T: ~ A, B, N_A. \]
    \item Сервер $T$ вырабатывает секретный сеансовый ключ $K$ для $A$ и $B$ и отправляет стороне $A$ шифрованное сообщение:
            \[ A \leftarrow T: ~ E_{K_{AT}}(N_A, B, K, E_{K_{BT}}(K, A)). \]
    \item Сторона $A$ расшифровывает сообщение
            \[ D_{K_{AT}}( E_{K_{AT}}(N_A, B, K, E_{K_{BT}}(K, A))) = (N_A, B, K, E_{K_{BT}}(K, A)) \]
        и, чтобы доставить ключ, передаёт стороне $B$ сообщение:
            \[ A \rightarrow B: ~ E_{K_{BT}}(K, A). \]
    \item Сторона $B$ расшифровывает полученное сообщение
            \[ D_{K_{BT}}( E_{K_{BT}}( K,A)) = (K,A) \]
        и получает ключ и реквизиты $A$, которые требуются для того, чтобы сторона $B$ знала, кому отвечать. Кроме того, сторона $B$ дополнительно желает идентифицировать сторону $A$. Для этого $B$ пересылает $A$ зашифрованную одноразовую метку:
            \[ A \leftarrow B: ~ E_{K}(N_B). \]
    \item Сторона $A$ расшифровывает
            \[ D_K( E_K( N_B)) = N_B \]
        и возвращает $B$ изменённую одноразовую метку
            \[ A \rightarrow B: ~ E_K(N_B + 1). \]
    \item Сторона $B$ расшифровывает
            \[ D_K( E_K( N_B + 1)) = N_B + 1, \]
        проверяет $N_B$ и убеждается, что имеет дело со стороной $A$.
    \item Если требуется двусторонняя аутентификация, то аналогично поступают со стороной $A$: на некотором этапе вносится одноразовая метка $N_A$.
\end{enumerate}


\section{Асимметричные протоколы}

Асимметричные протоколы, или же протоколы, основанные на криптосистемах с открытыми ключами, позволяют ослабить требования к предварительному этапу протоколов. Вместо общего секретного ключа, который должны иметь две стороны (либо обе стороны и доверенный центр), в рассматриваемых ниже протоколах стороны должны предварительно обменяться открытыми ключами (между собой, либо между собой и доверенным центром). Такой предварительный обмен может проходить по открытому каналу связи, в предположении, что криптоаналитик не может повлиять на содержимое канала связи на данном предварительном этапе.

\subsection{Простой протокол}

Рассмотрим протокол распространения ключей с помощью асимметричных шифров. Введём обозначения: $K_B$ -- открытый ключ стороны $B$, а $K_A$ -- открытый ключ стороны $A$. Протокол включает три сеанса обмена информацией:
\begin{enumerate}
    \item В первом сеансе сторона $A$ посылает стороне $B$ сообщение:
            \[ A \rightarrow B: ~ E_{K_B}(K_1, A), \]
        где $K_1$ -- ключ, выработанный стороной $A$.
    \item Сторона $B$ получает $(K_1, A)$ и передаёт стороне $A$ наряду с другой информацией свой ключ $K_2$ в сообщении, зашифрованном с помощью открытого ключа $K_A$:
            \[ A \leftarrow B: ~ E_{K_A}(K_2, K_1, B). \]
    \item Сторона $A$ получает и расшифровывает сообщение $(K_2, K_1, B)$. Во время третьего сеанса сторона $A$, чтобы подтвердить, что она знает ключ $K_2$, посылает стороне $B$ сообщение
            \[ A \rightarrow B: ~ E_{K_B}(K_2). \]
\end{enumerate}
Общий ключ формируется из двух ключей $K_1, K_2$.

\subsection{Протоколы с цифровыми подписями}

Существуют протоколы обмена, в которых перед началом обмена ключами генерируются подписи сторон $A$ и $B$, соответственно $S_A(m)$ и $S_B(m)$. В этих протоколах можно использовать различные одноразовые метки. Рассмотрим пример.
\begin{enumerate}
    \item Сторона $A$ выбирает ключ $K$ и вырабатывает сообщение
            \[ \left( K, ~ t_A, ~ S_A(K, t_A, B) \right), \]
        где $t_A$ -- метка времени. Зашифрованное сообщение передаёт стороне $B$:
        \[ A \rightarrow B: ~ E_{K_B}(K, ~ t_A, ~ S_A(K, t_A, B)). \]
    \item Сторона $B$ получает это сообщение, расшифровывает $\left( K, ~ t_A, ~ S_A(K, t_A, B) \right)$ и вырабатывает свою метку времени $t_B$. Проверка считается успешной, если $|t_B - t_A | < \delta $. Сторона $B$ знает свои реквизиты и может осуществлять проверку подписи.
\end{enumerate}

Имеется второй вариант протокола, в котором шифрование и подпись выполняются раздельно.
\begin{enumerate}
    \item Сторона $A$ вырабатывает ключ $K$, использует одноразовую метку (или метку времени) $t_{A}$ и передаёт стороне $B$ два различных зашифрованных сообщения:
            \[ \begin{array}{ll}
                A \rightarrow B: & ~ E_{K_B}(K, t_A), \\
                A \rightarrow B: & ~ S_A(K, t_A, B). \\
            \end{array} \]
    \item Сторона $B$ получает это сообщение, расшифровывает $K, t_A$ и, добавив свои реквизиты $B$, может проверить подпись $S_A(K, t_A, B)$.
\end{enumerate}

В третьем варианте протокола сначала производится шифрование, потом подпись.
\begin{enumerate}
    \item Сторона $A$ вырабатывает ключ $K$, использует одноразовую случайную метку или метку времени $t_A$ и передаёт стороне $B$ сообщение
        \[ A \rightarrow B: ~ t_A, ~ E_{K_B}(K, A), ~ S_A(t_A, ~ K, ~ E_{K_B}(K, A)). \]
    \item Сторона $B$ получает это сообщение, расшифровывает $\left( t_A, ~ K, ~ A, ~ E_{K_B}(K, A) \right)$ и проверяет подпись $S_A(t_A, ~ K, ~ E_{K_B}(K, A))$.
\end{enumerate}

\subsection{Протокол Диффи~---~Хеллмана}\index{протокол!Диффи~---~Хеллмана}
\selectlanguage{russian}

Алгоритм с открытым ключом впервые был предложен У.~Диффи (W.~Diffie) и М.~Хеллманом (M.E.~Hellman) в работе 1976 года <<Новые направления в криптографии>>(<<New directions in cryptography>>,~\cite{Diffie:Hellman:1976}).

Рассмотрим протокол Диффи~---~Хеллмана обмена информацией двух сторон $A$ и $B$. Задача состоит в том, чтобы создать общий сеансовый ключ.

Пусть $p$ -- большое простое число\index{число!простое}, $g$ -- примитивный элемент группы $\Z_p^*$, ~ $y = g^x \mod p$, причем $p,y,g$ -- известны заранее. Функцию $y=g^{x} \mod p$ считаем однонаправленной, т.~е. вычисление функции при известном значении аргумента является легкой задачей, а ее обращение (нахождение аргумента) при известном значении функции -- трудной.

Протокол обмена состоит из следующих действий.
\begin{enumerate}
    \item Сторона $A$ выбирает случайное число $x, ~ 2 \leq x \leq (p-1)$, вычисляет и передаёт стороне $B$ сообщение:
        \[ A \rightarrow B: ~ g^x \mod p. \]
    \item Сторона $B$ выбирает случайное число $y, ~ 2\leq y \leq (p-1)$, вычисляет и передаёт стороне $A$:
        \[ A \leftarrow B: ~ g^y \mod p. \]
    \item Сторона $A$, используя известные ей значения $x,g^{y} \mod p$, вычисляет ключ
        \[ K_{A} =(g^{y})^{x}\mod p=g^{xy} \mod p. \]
    \item Сторона $B$, используя известные ей значения $y,g^{x} \mod p$, вычисляет ключ
        \[ K_{B} =(g^{x})^{y}\mod p=g^{xy}\mod p. \]
        В результате получаем равенство  $K_A = K_B = K$.
\end{enumerate}

Таким способом создан общий секретный сеансовый ключ. В каждом новом сеансе используется этот же протокол для создания нового сеансового ключа.

Рассмотрим протокол Диффи~---~Хеллмана в ситуации, когда имеются три легальных пользователя $A,B,C$.

Каждая из сторон $A,B,C$ вырабатывает случайные числа $x,y,z$ соответственно и держит их в секрете.

\begin{enumerate}
    \item Первый этап обмена информацией аналогичен вышеописанному обмену информацией между двумя сторонами:
        \begin{enumerate}
            \item $A \rightarrow B: ~ g^x \mod p$.
            \item $B \rightarrow C: ~ g^y \mod p$.
            \item $C \rightarrow A: ~ g^z \mod p$.
        \end{enumerate}
    \item Второй этап состоит из передач сообщений:
        \begin{enumerate}
            \item $A \rightarrow B: ~ (g^z)^x = g^{zx} \mod p$.
            \item $B \rightarrow C: ~ (g^x)^y = g^{xy} \mod p$.
            \item $C \rightarrow A: ~ (g^y)^z = g^{yz} \mod p$.
        \end{enumerate}
    \item На завершающем третьем этапе стороны вычисляют:
        \begin{enumerate}
            \item $A: ~ K_A = (g^{yz})^x = g^{xyz} \mod p$.
            \item $B: ~ K_A = (g^{zx})^y = g^{xyz} \mod p$.
            \item $C: ~ K_A = (g^{xy})^z = g^{xyz} \mod p$.
        \end{enumerate}
\end{enumerate}

Как видно из произведенных действий, выработанные сторонами $A, B, C$ ключи совпадают: $K_A = K_B = K_C = K$. Следовательно, создан общий секретный сеансовый ключ $K$ для трёх участников.

Таким же образом можно построить протокол Диффи~---~Хеллмана для любого числа легальных пользователей.

Рассмотрим этот двусторонний протокол с точки зрения криптоаналитика, желающего узнать ключ $K$. Предположим, ему удалось перехватить сообщения $g^{x}\mod p$ и $g^{y}\mod p $. Используя заранее известные данные $g,p $ и эти сообщения, криптоаналитик старается найти хотя бы одно из чисел $(x,y)$, то есть решить задачу дискретного логарифма. В настоящее время эта задача считается вычислительно трудной при обычно выбираемых значениях $p\sim 2^{1024}$.

Существует атака активного криптоаналитика\index{криптоаналитик!активный}, названная <<человек посредине>> (man-in-the-middle)\index{атака!<<человек посередине>>}. Пусть имеются две легальные стороны $A$ и $B$ и нелегальная сторона $E$, активный криптоаналитик\index{криптоаналитик!активный}, который имеет возможность перехватывать и подменять сообщения как от $A$, так и от $B$:
    \[ A \leftrightsquigarrow E \leftrightsquigarrow B. \]
    %\[ A \leftrightarrow E \leftrightarrow B. \]

\begin{enumerate}
    \item Подмена ключей:
        \begin{enumerate}
            \item Сторона $A$ передаёт стороне $B$ сообщение:
                \[ A \overset{E}{\nrightarrow} B: ~ g^x \mod p. \]
            \item Сторона $E$ перехватывает сообщение $g^x \mod p$, сохраняет его и, зная $g$, передаёт стороне $B$ свое сообщение:
                \[ E \rightarrow B: ~ g^z \mod p. \]
            \item Сторона $B$ передаёт стороне $A$ сообщение:
                \[ A \overset{E}{\nleftarrow} B: ~ g^y \mod p. \]
            \item Сторона $E$ перехватывает сообщение $g^y \mod p$, сохраняет его и передаёт стороне $A$ свое сообщение:
                \[ A \leftarrow E: ~ g^z \mod p \]
                или какое-то другое.
            \item Таким образом между сторонами $A$ и $E$ образуется общий секретный ключ $K_{AE}$, между $B$ и $E$ -- ключ $K_{BE}$, причем $A$ и $B$ не знают, что у них ключи со стороной $E$, а не с друг другом
                \[ \begin{array} {l}
                    K_{AE} = g^{xz} \mod p, \\
                    K_{BE} = g^{yz} \mod p. \\
                \end{array} \]

        \end{enumerate}
    \item Подмена сообщений:
        \begin{enumerate}
            \item Сторона $A$ посылает $B$ сообщение $m$, зашифрованное на ключе $K_{AE}$:
                % \rightsquigarrow
                \[ A \overset{E}{\nrightarrow} B: ~ E_{K_{AE}}(m). \]
            \item Сторона $E$ перехватывает сообщение, расшифровывает с ключом $K_{AE}$, возможно, подменяет на $m'$, зашифровывает с ключом $K_{BE}$ и посылает $B$:
                \[ E \rightarrow B: ~ E_{K_{BE}}(m'). \]
            \item То же самое происходит при обратной передаче от $B$ к $A$.
        \end{enumerate}
\end{enumerate}

Криптоаналитик $E$ имеет возможность перехватывать и подменять все передаваемые сообщения. Если по тексту письма нельзя обнаружить участие криптоаналитика в обмене информацией, то атака <<человек посередине>>\index{атака!<<человек посередине>>} успешна.

Существует несколько протоколов для преодоления атаки этого типа.


%\section{Протоколы с аутентификацией}

\subsection{Односторонняя аутентификация}

\textbf{Протокол Эль-Гамаля}\index{протокол!Эль-Гамаля} относится к протоколам с аутентификацией одного из двух легальных пользователей.
\selectlanguage{russian}
\begin{enumerate}
    \item Для начала стороны выбирают общие параметры $p, g$, где $p$ -- большое простое число, где $g$ -- примитивный элемент поля $\Z_p^*$;
    \item Сторона $B$ создаёт свои закрытый и открытый ключи:
            \[ \SK_B = b, ~ \PK_B = g^b \mod p, \]
        $b$ -- случайное секретное число, $2 \leq b \leq p-1$.

        Открытый ключ $\PK_B$ находится в общем открытом доступе для всех сторон, поэтому криптоаналитик $E$ не может подменить его -- подмена будет заметна:
    \item Сторона $A$ вырабатывает свой секрет $x$, сеансовый ключ:
            \[ K_A = (\PK_B)^x = g^{bx} \mod p \]
        и отправляет $B$:
            \[ A \rightarrow B: ~ g^x \mod p. \]
    \item Сторона $B$, получив от $A$ число $g^x \mod p$, использует его и свой секрет $\SK_B = b$, чтобы создать свой ключ:
            \[ K_B = (g^x)^{\SK_B} = g^{bx} \mod p, \]
        то есть сеансовые ключи обеих сторон совпадают:
            \[ K_A = K_B = K. \]
\end{enumerate}

Достоинством этого протокола является следующее его свойство. Если ключи $K_A$ и $K_B$ совпали и стороны могут обмениваться информацией, то сторона $A$ аутентифицирует сторону $B$, так как для шифрования она использовала открытый ключ $B$, который не может быть незаметно подменён, и только сторона $B$ может расшифровывать сообщения.

Что касается криптоаналитика в качестве <<человека-посередине>>, то он может отправлять ложные сообщения, но не может узнать ключ $K$ и читать сообщения.

Есть протоколы, в которых стороны, осуществляющие обмен информацией, являются равноправными. Они называются протоколами взаимной аутентификации.


\input{mti}

\subsection{Взаимная аутентификация схемой ЭП}
\selectlanguage{russian}

\textbf{Протокол STS (Station-to-Station)}\index{протокол!Station-to-Station} предназначен для систем мобильной связи. Он использует идеи протокола Диффи~---~Хеллмана\index{протокол!Диффи~---~Хеллмана} и криптосистемы RSA\index{криптосистема!RSA}. Особенностью протокола является использование механизма электронной подписи\index{электронная подпись} для взаимной аутентификации сторон\index{аутентификация!взаимная}.

Здесь открытые общедоступные данные:
    \[ p, ~ g, ~ \PK_A, ~ \PK_B. \]

Каждая из сторон $A$ и $B$ обладаёт долговременной парой ключей: закрытым ключом для создания электронной подписи $\SK$ и открытым ключом для проверки подписи $\PK$.
\[ \begin{array}{ll}
    A: & ~ \SK_A, ~~ \PK_A, \\
    B: & ~ \SK_B, ~~ \PK_B. \\
\end{array} \]

Протокол состоит из трёх раундов обмена информацией между сторонами $A$ и $B$:
\begin{enumerate}
    \item Сторона $A$ создаёт секретное случайное число $2 \leq x \leq p-1$ и отправляет $B$:
            \[ A \rightarrow B: ~ g^x \mod p. \]
    \item Сторона $B$ создаёт секретное случайное число $2 \leq y \leq p-1$, вычисляет общий секретный ключ
            \[ K = (g^x)^y = g^{xy} \mod p, \]
        с помощью которого создаёт шифрованное сообщение $E_K(S_B(g^x, g^y))$ для аутентификации, и отправляет $A$:
            \[ A \leftarrow B: ~ \left( g^y \mod p, ~~ E_K( S_B( g^x, g^y)) \right). \]
    \item Сторона $A$ с помощью $x, g^y \mod p$ вычисляет общий секретный ключ
            \[ K = (g^y)^x \mod p = g^{xy} \mod p \]
        и расшифровывает сообщение
            \[ D_K( E_K( S_B( g^x, g^y))) = S_B( g^x, g^y). \]
            Затем аутентифицирует сторону $B$, проверяя подпись $S_B$ открытым ключом $\PK_B$. Вычисляет и пересылает стороне $B$ сообщение:
            \[ A \rightarrow B: ~ E_K( S_A( g^x, g^y)). \]
    \item Сторона $B$ расшифровывает принятое сообщение
            \[ D_K( E_K( S_A( g^x, g^y))) = S_A( g^x, g^y) \]
        и осуществляет аутентификацию, выполняя проверку подписи $S_A$ с помощью открытого ключа $\PK_A$.
\end{enumerate}


\input{girault_scheme}

\subsection{Схема Блома}\index{схема!Блома}
\selectlanguage{russian}

Рассмотрим распределение ключей по \emph{схеме Блома} (Rolf Blom,~\cite{Blom:1984, Blom:1985}), в котором каждые два пользователя из общего числа $N$ пользователей могут создать общий секретный ключ, причём секретные ключи каждой пары различны. Данная схема используется в протоколе HDCP\index{протокол!HDCP} (\langen{High-bandwidth Digital Content Protection}) для предотвращения копирования высококачественного видеосигнала.

На этапе инициализации доверенный центр выбирает симметричную матрицу $D_{m,m}$ над конечным полем $\GF p$. Для присоединения к сети распространения ключей новый участник либо самостоятельно, либо с помощью доверенного центра выбирает новый открытый ключ (идентификатор) $I$, представляющий собой вектор длины $k$ над $\GF p$. Доверенный центр вычисляет для нового участника закрытый ключ $K$:

\begin{equation}
	K = D_{m,m} I.
	\label{eq:blom_center_matrix}
\end{equation}

Симметричность матрицы $D_{m,m}$ доверенного центра позволяет любым двум участникам сети создать общий сеансовый ключ. Пусть Алиса и Боб -- легальные пользователи сети, то есть они обладают открытыми ключами $I_A$ и $I_B$ соответственно, а их закрытые ключи $K_A$ и $K_B$ были вычислены одним и тем же доверенным центром по формуле~\ref{eq:blom_center_matrix}. Тогда протокол выработки общего секретного ключа выглядит следующим образом.

\begin{enumerate}
	\item Алиса отправляет Бобу свой открытый ключ $I_A$.
	\item Боб отправляет Алисе свой открытый ключ $I_A$.
	\item Алиса вычисляет значение $s_{AB} = K^t_A I_B = I^t_A D_{m,m} I_B$;
	\item Боб вычисляет значение $s_{BA} = K^t_B I_A = I^t_B D_{m,m} I_A$.
\end{enumerate}

Из симметричности матрицы $D_{m,m}$ следует, что значения $s_{AB}$ и $s_{BA}$ совпадут, что и будет являться общим секретным ключом для Алисы и Боба. Этот секретный ключ будет свой для каждой пары легальных пользователей сети.

Присоединение новых участников к схеме строго контролируется доверенным центром, что позволяет защитить сеть от нелегальных пользователей. Надёжность данной схемы основывается на невозможности восстановить исходную матрицу. Однако для восстановления матрицы доверенного центра размера $m \times m$ необходимо и достаточно всего $m$ пар линейно независимых открытых и закрытых ключей. В 2010 году компания Intel, которая является <<доверенным центром>> для пользователей системы защиты HDCP, подтвердила, что криптоаналитикам удалось найти секретную матрицу (точнее, аналогичную ей), используемую для генерации ключей в упомянутой системе предотвращения копирования высококачественного видеосигнала.


В этом разделе были рассмотрены протоколы, в которых ключи вырабатываются в процессе обмена информацией.
%Существует и другой подход, который будет рассмотрен в следующих разделах.


\chapter{Разделение секрета}

\section{Пороговые схемы}

Идея \textbf{пороговой} $(n, N)$-схемы\index{разделение секрета!пороговое} разделения общего секрета среди $N$ пользователей состоит в следующем.
%описывается так:
Доверенная сторона хочет распределить некий секрет $K_0$ между $N$ пользователями таким образом, что:
%. Поставлены следующие условия:
\begin{itemize}
    \item любые $m, ~ n \le m \le N$ легальных пользователей могут получить секрет (или доступ к секрету), если предъявят свои секретные ключи;
    \item любые $m, ~ m < n$, легальных пользователей не могут получить секрет и не могут определить (вычислить) этот секрет, пытаясь решить трудную в вычислительном смысле задачу.
\end{itemize}

Далее рассмотрены два случая: $(n, N)$-схема Шамира и простая $(N,N)$-схема.

\subsection[Схема Шамира]{Схема распределения секрета Шамира}
\selectlanguage{russian}


\subsubsection{Необходимые сведения из линейной алгебры}

Рассмотрим матрицу Вандермонда $V$ размера $(n \times n)$, где
\[
    V = \left(\begin{array}{cccc}
        {1} & {1} & { \ldots } & {1} \\
        {x_{1} } & {x_{2} } & { \ldots } & {x_{n} } \\
        {\begin{array}{l} {} \\ {x_{1}^{2} } \end{array}} &
            {\begin{array}{l} {} \\ {x_{2}^{2} } \end{array}} &
            {\begin{array}{l} {} \\ { \ldots } \end{array}} &
            {\begin{array}{l} {} \\ {x_{n}^{2} } \end{array}} \\
        {\begin{array}{l} { \ldots } \\ {x_{1}^{n-1} } \end{array}} &
            {\begin{array}{l} { \ldots } \\ {x_{2}^{n-1} } \end{array}} &
            {\begin{array}{l} { \ldots } \\ { \ldots } \end{array}} &
            {\begin{array}{l} { \ldots } \\ {x_{n}^{n-1} } \end{array}}
    \end{array}\right),
\]
где $x_{i}$ -- элемент поля $\Z_{p}$, $x_{i} \ne x_{j}$, $p$ -- большое простое\index{число!простое} число.

Определитель матрицы Вандермонда равен
    \[ \det V = \prod_{1\le i < j\le n} \left(x_j - x_i \right) \mod p. \]
В частности, при $n=2$ определитель
    \[ \det V = x_2 - x_1, \]
при $n=3$ определитель равен
    \[ \det V = (x_3 - x_2) (x_3 - x_1) (x_2 - x_1). \]

Если все элементы $x_{i}$ имеют различные значения, то $\det V \ne 0$, и матрица Вандермонда является невырожденной. Это значит, что существует обратная матрица.

Определим вектор-строку
    \[ \left(K_{0}, K_{1},  \ldots,  K_{n-1}\right), ~ K_{i} \in \Z_{p} \]
как решение уравнения
%Ставим задачу решить уравнение
    \[ (K_{0} ,K_{1} ,  \ldots, K{}_{n-1} )V=(y_{1} ,  \ldots, y_{n} ) \]
или, что эквивалентно, решение системы уравнений
\[ \begin{array}{l}
    y_1 = K_0 + K_1 x_1 + K_2 x_1^2 + \dots + K_{n-1} x_1^{n-1}, \\
    y_2 = K_0 + K_1 x_2 + K_2 x_2^2 + \dots + K_{n-1} x_2^{n-1}, \\
    \dots \\
\end{array} \]

Используя методы линейной алгебры, получим решение в виде
    \[ (K_{0} ,K_{1} ,  \ldots,  K_{n-1} )=(y_{1} ,  \ldots,  y_{n} )V^{-1}, \]
где $V^{-1}$ -- обратная матрица.

Однако прямого вычисления обратной матрицы здесь можно избежать. Введём для этого многочлены
\[ \begin{array}{c}
    K(x)= K_{0} +K_{1} x+ \ldots +K_{n-1} x^{n-1}, \\
    y_{i} =K(x_{i}). \\
\end{array} \]
Теперь решение этой задачи можно задать интерполяционной формулой Лагранжа:
\[
    K(x) = y_1 \frac{(x - x_2)(x - x_3) \dots (x - x_n)} {(x_1-x_2)(x_1-x_3) \dots (x_1-x_n)} +
\] \[
    + y_2 \frac{(x-x_1)(x-x_3) \dots (x-x_n)} {(x_2-x_1)(x_2-x_3) \dots (x_2-x_n)} + \dots +
\] \[
    + y_n \frac{(x-x_1)(x-x_2) \dots (x-x_{n-1})} {(x_n-x_1)(x_n-x_2) \dots (x_n-x_{n-1})}.
\]

Первое слагаемое равно нулю в точках $x_2, x_3, \dots, x_n$,   равно $y_1$ в точке $x_1$. Знаменатель не обращается в нуль, так как все $x_1, \dots, x_n$ имеют различные значения. Второе слагаемое равно $y_2$ в точке $x_2$, а при всех других значениях $x_i$ обращается в нуль. Аналогично обстоят дела с остальными слагаемыми.

Из всех коэффициентов $K_0, K_1, \dots, K_{n-1}$ нас интересует только $K_0$.
Положив $x=0$, получаем выражение для $K_{0} $ в виде
\[
    K(x) = (-1)^{n-1} y_1 \frac{x_2 x_3 \dots x_n} {(x_1-x_2) \dots (x_1-x_n)} +
\] \[
    + (-1)^{n-1} y_2 \frac{x_1 x_3 \dots x_n} {(x_2-x_1) \dots (x_2-x_n)} + \dots +
\] \[
    + (-1)^{n-1} y_n \frac{x_1 x_2 \dots x_{n-1}} {(x_n-x_1) \dots (x_n-x_{n-1})}.
\]

Интерполяционный многочлен Лагранжа принимает заданные значения в заданных точках. В нашей задаче $K(x)=K_{0}$ при $x=0$.


\subsubsection{Описание схемы Шамира}

В пороговой \textbf{схеме Шамира}\index{распределение секрета!Шамира} распределения секретов доверенная сторона предварительно производит следующие действия:
\begin{itemize}
    \item Выбирает большое простое\index{число!простое} число $p: ~ p \sim 2^{512} \dots 2^{1024}$;
    \item Выбирает $N$ различных чисел $x_1, x_2, \dots, x_N$, каждое из которых меньше $p$;
    \item Выбирает прямоугольную матрицу Вандермонда:
        \[
            V_{n \times N} = \left( \begin{array}{cccc}
                {1} & {1} & { \ldots } & {1} \\
                {x_{1} } & {x_{2} } & { \ldots } & {x_{N} } \\
                { \ldots } & { \ldots } & { \ldots } & { \ldots } \\
                {x_{1}^{n-1} } & {x_{2}^{n-1} } & { \ldots } & {x_{N}^{n-1} }
            \end{array} \right) \mod p.
        \]
    \item Выбирает секрет $K_0$, а также выбирает случайные числа $K_1, K_2, \dots, K_{n-1}$;
    \item Вычисляет частичные секреты -- числа $y_1, y_2, \dots, y_N$:
        \[ (y_1, y_2, \dots, y_N) = (K_0, K_1, \dots, K_{n-1}) V. \]
\end{itemize}

Распределение секрета $K_0$ между $N$ сторонами состоит в том, что доверенная сторона выдаёт легальному пользователю $i$  открытый ключ $\PK_i$, который известен всем, и секретный ключ $\SK_i$ (секрет только $i$-го пользователя):
\[ \begin{array}{ll}
    \PK_1 = x_1, & \SK_1 = y_1, \\
    \PK_2 = x_2, & \SK_2 = y_2, \\
    \cdots & \\
    \PK_N = x_N, & \SK_N = y_N. \\
\end{array} \]

Покажем, что такое распределение удовлетворяет поставленным требованиям.

Пусть собрались любые $n$ из общего числа $N$ пользователей, имеющих значения $(x_i, y_i)$. Каждому из них можно поставить в соответствие один столбец матрицы Вандермонда: $y_i = K(x_i)$. Как показано в предыдущем параграфе, это позволяет найти значение $K_0$.

Предположим, что собралось $m$ ($m<n$) пользователей. Заметим, что число неизвестных $K_0, K_1, \dots, K_{n-1}$ в системе уравнений осталось неизменным и равным $n$, а число уравнений меньше, так как $m<n$. В этом случае решение существует, но не является единственным. Если коэффициенты многочленов взяты из поля $\Z_p$, то число решений является конечным.

Например, если $m = n - 1$, тогда
    \[ K_0 + K_1 x_j + K_2 x_j^2 + \dots + K_{n-1} x_j^{n-1} = y_j, \]
и $K_0$ может принимать $p$ значений. Найти все решения перебором -- вычислительно трудная задача.

Если $m = n - 2$, то число различных решений равно $p^2$. Это число экспоненциально возрастает по мере уменьшения числа собравшихся вместе получателей секрета.

Таким образом, схема Шамира распределения секрета удовлетворяет предъявленным требованиям.

\subsubsection{Пример схемы Шамира}

Метод Шамира, называемый также схемой интерполяционных полиномов Лагранжа\index{многочлен!интерполяционный Лагранжа}, основывается на том, что для восстановления многочлена $f(x)$ степени $k-1$ необходимо и достаточно знать значения многочлена в любых $k$ разных точках.

Для секрета $M$ формируется многочлен
    \[ f(x) = \sum\limits_{i=1}^{k-1} a_i x^i + M, \]
где коэффициенты $a_i$ выбираются случайно. Вычисляются значения $y_i = f(x_i)$ в $n$ различных точках. Пользователю $i$ выдаётся тень $(x_i, y_i)$.

Для восстановления секрета по любым $k$ точкам $(x_i, y_i)$ используется интерполяционный многочлен Лагранжа:
    \[ f(x) = \sum\limits_{i=0}^{k-1} y_i \cdot l_i(x), ~~ l_i(x) = \prod\limits_{j=0, j \neq i}^{k-1} \frac{x - x_j}{x_i - x_j}. \]
Общий секрет $M$ является свободным коэффициентом $f(x)$.
    \[ M = \sum\limits_{i=0}^{k-1} y_i \prod\limits_{j=0, j \neq i}^{k-1} \frac{x_j}{x_j - x_i}. \]

\example
Приведём схему Шамира в поле $\GF{p}$. Для разделения секрета $M$ в $(3,n)$ схеме используется
    \[ f(x) = a x^2 + b x + M \mod p, \]
где $p$ -- простое\index{число!простое} число. Пусть $p=23$. Восстановим секрет $M$ по \emph{теням}
    \[ (1,14), (4,21), (15,6). \]

Последовательно вычисляем

    \[ M = \sum\limits_{i=0}^{k-1} y_i \prod\limits_{j=0, j \neq i}^{k-1} \frac{x_j}{x_j - x_i} \mod p = \]
    \[= 14 \cdot \frac{4}{4-1} \cdot \frac{15}{15-1} + 21 \cdot \frac{1}{1-4} \cdot \frac{15}{15-4} + 6 \cdot \frac{1}{1-15} \cdot \frac{4}{4-15} \mod 23 = \]
    \[ =14 \cdot \frac{4}{3} \cdot \frac{15}{14} + 21 \cdot \frac{1}{-3} \cdot \frac{15}{11} + 6 \cdot \frac{1}{-14} \cdot \frac{4}{-11} \mod 23 = \]
    \[= 20 - 7 \cdot 15 \cdot 11^{-1} + 12 \cdot 7^{-1} \cdot 11^{-1} \mod 23 = \]
    \[ = 13 \mod 23.\]
\exampleend


\subsection[$(N, N)$-схема]{$(N, N)$-схема распределения секрета}
\selectlanguage{russian}

Рассмотрим пороговую схему распределения одного секрета между двумя легальными пользователями. Она обозначается $(2,2)$-схема -- это означает, что оба и только оба пользователя могут получить секрет. Предположим, что секрет $K_{0}$ -- это двоичная последовательность длины $M$, $K_{0} \in \Z_{M}$.

Распределение секрета $K_{0}$ состоит в следующем:
\begin{itemize}
    \item Первый пользователь в качестве секрета получает случайную двоичную последовательность $A_{1}$ длины $M$;
    \item Второй пользователь в качестве секрета получает случайную двоичную последовательность $A_{2} =K_{0} \oplus A_{1}$ длины $M$;
    \item Для получения секрета $K_{0}$ оба пользователя должны сложить по модулю 2 свои секретные ключи  (последовательности)  $K_{0} = A_{2} \oplus A_{1}$.
\end{itemize}

Теперь рассмотрим пороговую $(N,N)$-схему.

Имеется общий секрет $K_{0} \in \Z_{M}$ и $N$ легальных пользователей, которые могут получить секрет только в случае, если одновременно предъявят свои секретные ключи. Распределение секрета $K_{0}$ происходит следующим образом.

\begin{itemize}
    \item Первый пользователь в качестве секрета получает случайную двоичную последовательность $A_{1} \in \Z_{M}$;
    \item Второй пользователь в качестве секрета получает случайную двоичную последовательность $A_{2}\in \Z_{M}$ и т.~д.;
    \item $(N-1)$-й пользователь в качестве секрета получает случайную двоичную последовательность $A_{N-1}\in \Z_{M}$;
    \item $N$-й пользователь в качестве секрета получает двоичную последовательность
        \[ K_0 \oplus A_1 \oplus A_2 \oplus \dots \oplus A_{N-1}. \]
    \item Для получения секрета $K_0$ все пользователя должны сложить по модулю 2 свои последовательности:
        \[ A_1 \oplus A_2 \oplus \dots \oplus A_{N-1} \oplus (K_0 \oplus A_1 \oplus A_2 \dots \oplus A_{N-1}) = K_0. \]
\end{itemize}

Предположим, что собравшихся вместе пользователей меньше общего числа $N$, например всего $N-1$ первых пользователей. Тогда суммирование $N-1$ последовательностей не определяет секрета, а перебор невозможен, так как данная схема разделения секрета аналогична криптосистеме Вернама и обладаёт совершенной криптостойкостью.


\section[Распределение секрета по коалициям]{Распределение секрета по \protect\\ коалициям}

\subsection{Схема для нескольких коалиций}

Предположим, что имеется $N$ легальных пользователей
    \[ \{ U_1, U_2, \dots, U_N \}, \]
которым нужно сообщить (открыть, получить доступ к) общий секрет $K$.

Секрет может быть доступен только определённым коалициям\index{распределение секрета!по коалициям}, например
\[ \begin{array}{l}
    C_1 = \{ U_1, U_2 \}, \\
    C_2 = \{ U_1, U_3, U_4 \}, \\
    C_3 = \{ U_2, U_3 \}, \\
    \dots
\end{array} \]
При этом ни одна из коалиций $C_i, ~ i = 1, 2, \dots$ не должна быть подмножеством другой коалиции.


\subsubsection{Пример 1}

Имеется 4 участника
    \[ \{ U_1, U_2, U_3, U_4 \}, \]
которые образуют 3 коалиции
\[ \begin{array}{l}
    C_1 = \{ U_1, U_2 \}, \\
    C_2 = \{ U_1, U_3 \}, \\
    C_3 = \{ U_2, U_3, U_4 \}. \\
\end{array} \]
Распределение частичных секретов между ними представлено в виде табл.~\ref{tab:secret-share-coalition-1}, в которой введены следующие обозначения: $a_1, b_1, c_2, c_3$ -- случайные числа из кольца $\Z_M$. В строках таблицы содержатся частичные секреты каждого из пользователей, в столбцах таблицы показаны частичные секреты, соответствующие каждой из коалиций.

\begin{table}[!ht]
    \centering
    \caption{Распределение секрета по определённым коалициям\label{tab:secret-share-coalition-1}}
    \begin{tabular}{|c||c|c|c|}
        \hline
              & $C_1 = \{ U_1, U_2 \}$ & $C_2 = \{U_1, U_3 \}$ & $C_3 = \{ U_2, U_3, U_4 \}$ \\
        \hline \hline
        $U_1$ & $a_1$     & $b_1$     & -- \\
        $U_2$ & $K - a_1$ & --        & $c_2$ \\
        $U_3$ & --        & $K - b_1$ & $c_3$  \\
        $U_4$ & --        & --        & $K - c_2 - c_3$ \\
        \hline
    \end{tabular}
\end{table}

Как видно из приведенных данных, суммирование по модулю $M$ чисел, приведенных в каждом из столбцов таблицы, открывает секрет $K$.


\subsubsection{Пример 2}

%\section{Схема разделения секрета на монотонных булевых функциях}
%\example
В системе распределения секрета доверенный
%с использованием монотонных булевых функций
центр использует кольцо $\Z_m$ целых чисел по модулю $m$. Требуется разделить секрет $K$ между $5$ пользователями
    \[ \{ U_1, U_2, U_3, U_4, U_5 \} \]
так, чтобы восстановить секрет могли только коалиции
\[ \begin{array}{lll}
    C_1 = \{ U_1, U_2 \},      & & C_2 = \{ U_1, U_3 \}, \\
    C_3 = \{ U_2, U_3, U_4 \}, & & C_4 = \{ U_2, U_3, U_5 \}, \\
    C_5 = \{ U_3, U_4, U_5 \}, & & C_6 = \{ U_1, U_2, U_3 \}. \\
\end{array} \]

Заданное множество коалиций с доступом не является минимальным, так как одни коалиции входят в другие:
    \[ C_1 \subset C_6, ~ C_2 \subset C_6. \]
Исключая $C_6$, получим минимальное множество коалиций с доступом к секрету -- ни одна из оставшихся коалиций не входит в другую $C_i \nsubseteq C_j$ для $i \neq j$. Пользователям выдаются тени по минимальному множеству коалиций с доступом. В строках таблицы~\ref{tab:secret-share-coalition-2} содержатся частичные секреты каждого из пользователей, в столбцах таблицы показаны частичные секреты, соответствующие каждой из коалиций.

\begin{table}[!ht]
    \centering
    \caption{Распределение секрета по определённым коалициям\label{tab:secret-share-coalition-2}}
    \begin{tabular}{|c||c|c|c|c|c|}
        \hline
              & $C_1$     & $C_2$     & $C_3$           & $C_4$           & $C_5$  \\
        \hline \hline
        $U_1$ & $a_1$     & $b_1$     & --              & --              & -- \\
        $U_2$ & $K - a_1$ & --        & $c_2$           & $d_2$           & --\\
        $U_3$ & --        & $K - b_1$ & $c_3$           & $d_3$           & $e_3$ \\
        $U_4$ & --        & --        & $K - c_2 - c_3$ & --              & $e_4$ \\
        $U_5$ & --        & --        & --              & $K - d_2 - d_3$ & $K - e_3 - e_4$ \\
        \hline
    \end{tabular}
\end{table}

Тени выбираются случайно из кольца $\mathbb{\Z}_m$. В результате у пользователей будут тени:
%\exampleend

\input{brickells_scheme}

\chapter{Примеры систем защиты}

\section{Система Kerberos для локальной сети}
\selectlanguage{russian}

Система аутентификации и распределения ключей Kerberos основана на протоколе Нидхема~---~Шрёдера. Самые известные реализации протокола Kerberos включают Microsoft Active Directory и ПО Kerberos с открытым кодом для Unix.

Протокол предназначен для решения задачи аутентификации и распределения ключей в рамках локальной сети, в которой есть группа пользователей, имеющих доступ к набору сервисов, и требуется обеспечить единую аутентификацию для всех сервисов. Протокол Kerberos использует только симметричное шифрование. Секретный ключ используется для взаимной аутентификации.

Естественно, что в нелокальной сети Интернет невозможно секретно создать и распределить пары секретных ключей, поэтому Kerberos построен для (виртуальной) локальной сети.

В протоколе используется 4 типа субъектов:

\begin{itemize}
    \item пользователи системы $C_i$;
    \item сервисы $S_i$, доступ к которым имеют пользователи;
    \item сервер аутентификации AS (Authentication Server), который производит аутентификацию пользователей по паролям и/или смарт-картам только один раз и выдаёт секретные сеансовые ключи для дальнейшей аутентификации;
    \item сервер выдачи мандатов TGS (Ticket Granting Server) для аутентификации доступа к запрашиваемым сервисам, аутентификация выполняется по сеансовым ключам\index{ключ!сеансовый}, выданным сервером AS.
\end{itemize}

Для работы протокола требуется заранее распределить следующие секретные симметричные ключи для взаимной аутентификации.
\begin{itemize}
    \item Ключи $K_{C_i}$ между пользователем $i$ и сервером AS. Как правило, ключом служит обычный пароль\index{пароль}, точнее результат хэширования пароля. Может быть использована и смарт-карта.
    \item Ключ $K_{TGS}$ между серверами AS и TGS.
    \item Ключи $K_{S_i}$ между сервисами $S_i$ и сервером TGS.
\end{itemize}

\begin{figure}[!ht]
	\centering
	\includegraphics[width=\textwidth]{pic/kerberos}
	\caption{Схема аутентификации и распределения ключей Kerberos\label{fig:kerberos}}
\end{figure}

На рис.~\ref{fig:kerberos} представлена схема протокола, состоящая из 6 шагов.

Введём обозначения для протокола между пользователем $C$ с ключом $K_C$ и сервисом $S$ с ключом $K_S$:
\begin{itemize}
    \item $ID_C, ID_{TGS}, ID_S$ -- идентификаторы пользователя, сервера TGS и сервиса $S$ соответственно;
    \item $t_i, \tilde{t}_i$ -- запрашиваемые и выданные границы времени действия сеансовых ключей аутентификации;
    \item $ts_i$ -- метка текущего времени (timestamp);
    \item $N_i$ -- одноразовая метка (nonce)\index{одноразовая метка}, псевдослучайное число для защиты от атак воспроизведения сообщений;
    \item $K_{C,TGS}, K_{C,S}$ -- выданные сеансовые ключи аутентификации пользователя и сервера TGS, пользователя и сервиса $S$ соответственно;
    \item $T_{TGS} = E_{K_{TGS}}(K_{C,TGS} ~\|~ ID_C ~\|~ \tilde{t}_1)$ -- мандат (ticket) для TGS, который пользователь расшифровать не может;
    \item $T_{S} = E_{K_S}(K_{C,S} ~\|~ ID_C ~\|~ \tilde{t}_2)$ -- мандат для сервиса $S$, который пользователь расшифровать не может;
    \item $K_1, K_2$ -- обмен информацией для генерирования общего секретного симметричного ключа дальнейшей коммуникации, например, по протоколу Диффи~---~Хеллмана\index{протокол!Диффи~---~Хеллмана}.
\end{itemize}

Схема протокола следующая:
\begin{enumerate}
    \item Первичная аутентификация пользователя по паролю, получение сеансового ключа $K_{C,TGS}$ для дальнейшей аутентификации. Это действие выполняется один раз для каждого пользователя, чтобы уменьшить риск компрометации пароля.
        \begin{enumerate}
            \item $C \rightarrow AS: ~~ ID_C ~\|~ ID_{TGS} ~\|~ t_1 ~\|~ N_1$.
            \item $C \leftarrow AS: ~~ ID_C ~\|~ T_{TGS} ~\|~ E_{K_C}( K_{C,TGS} ~\|~ \tilde{t}_1 ~\|~ N_1 ~\|~ ID_{TGS})$.
        \end{enumerate}
    \item Аутентификация сеансовым ключом $K_{C,TGS}$ на сервере TGS для запроса доступа к сервису выполняется один раз для каждого сервиса. Получение другого сеансового ключа аутентификации $K_{C,S}$.
        \begin{enumerate}
            \item $C \rightarrow TGS: ~~ ID_S ~\|~ t_2 ~\|~ N_2 ~\|~ T_{TGS} ~\|~ E_{K_{C,TGS}}(ID_C ~\|~ ts_1)$.
            \item $C \leftarrow TGS: ~~ ID_C ~\|~ T_{S} ~\|~ E_{K_{C,TGS}}( K_{C,S} ~\|~ \tilde{t}_2 ~\|~ N_2 ~\|~ ID_S)$.
        \end{enumerate}
    \item Аутентификация сеансовым ключом $K_{C,S}$ на сервисе $S$ -- создание общего сеансового ключа дальнейшего взаимодействия.
        \begin{enumerate}
            \item $C \rightarrow S: ~~ T_{S} ~\|~ E_{K_{C,S}}(ID_C ~\|~ ts_2 ~\|~ K_1)$.
            \item $C \leftarrow S: ~~ E_{K_{C,S}}( ts_2 ~\|~ K_2)$.
        \end{enumerate}
\end{enumerate}

Аутентификация и проверка целостности достигается сравнением идентификаторов, одноразовых меток и меток времени внутри зашифрованных сообщений после расшифрования с их действительными значениями.

Некоторым недостатком схемы является необходимость синхронизации часов между субъектами сети.


\section{Шифрование файлов и почтовых сообщений в PGP}
\selectlanguage{russian}

В качестве примера передачи файлов по сети с обеспечением аутентификации, конфиденциальности и целостности рассмотрим систему PGP (Pretty Good Privacy), разработанную Филом Циммерманном (Phil Zimmermann) в 1991 г. Изначально система предлагалась к использованию для защищённой передачи электронной почты. Стандартом PGP является OpenPGP. Примерами реализации стандарта OpenPGP являются GNU Privacy Guard (GPG) и netpgp, разработанные в рамках проектов GNU и NetBSD соответственно.

Каждый пользователь обладает одним или несколькими закрытыми ключами для криптосистемы с открытым ключом, которые используются для аутентификации посредством ЭП. Пользователь хранит также открытые ключи других пользователей, которые он использует для шифрования секретного сеансового ключа блокового шифрования. Передаваемое сообщение подписывается секретным ключом отправителя, затем сообщение шифруется блоковой криптосистемой на случайно выбранном сеансовом ключе. Сам сеансовый ключ шифруется криптосистемой с открытым ключом на открытом ключе получателя.

Свои закрытые ключи отправитель хранит в зашифрованном виде. Набор ключей называется связкой закрытых ключей. Шифрование закрытых ключей в связке производится симметричным шифром\index{шифр!симметричный}, ключом которого является функция от пароля, вводимого пользователем. Шифрование закрытых ключей, хранимых на компьютере, является стандартной практикой для защиты от утечки, например, в случае взлома ОС, утери ПК и т.~д.

Набор открытых ключей других пользователей называется связкой открытых ключей.

\begin{figure}[!ht]
	\centering
	\includegraphics[width=0.9\textwidth]{pic/pgp}
	\caption{Схема обработки сообщения в PGP\label{fig:pgp}}
\end{figure}

На рис.~\ref{fig:pgp} представлена схема обработки сообщения в PGP для передачи от $A$ к $B$. Использование аутентификации, сжатия и блокового шифрования является опциональным. Обозначения на рисунке следующие:
\begin{itemize}
    \item Пароль -- пароль, вводимый отправителем для расшифрования связки своих закрытых ключей;
    \item $D$ -- расшифрование блоковой криптосистемы для извлечения секретного ключа ЭП отправителя;
    \item $SK_A$ -- закрытый ключ ЭП отправителя;
    \item $ID_{SKa}$ -- идентификатор ключа ЭП отправителя, по которому получатель определяет, какой ключ из связки открытых ключей использовать для проверки подписи;
    \item $m$ -- сообщение (файл) для передачи;
    \item $h(m)$ -- криптографическая хэш-функция;
    \item $E_{SKa}$ -- схема ЭП на секретном ключе $SK_A$;
    \item $\|$ -- конкатенация битовых строк;
    \item $Z$ -- сжатие сообщения алгоритмом компрессии;
    \item $RND$ -- криптографический генератор псевдослучайной последовательности;
    \item $K_s$ -- сгенерированный псевдослучайный сеансовый ключ;
    \item $E_{Ks}$ -- блоковое шифрование на секретном сеансовом ключе $K_s$;
    \item $PK_B$ -- открытый ключ получателя;
    \item $ID_{PKb}$ -- идентификатор открытого ключа получателя, по которому получатель определяет, какой ключ из связки закрытых ключей использовать для расшифрования сеансового ключа;
    \item $E_{PKb}$ -- шифрование сеансового ключа криптосистемой с открытым ключом на открытом ключе $B$;
    \item $c$ -- зашифрованное подписанное сообщение.
\end{itemize}


\section{Протокол SSL/TLS}\index{протокол!SSL/TLS}
\selectlanguage{russian}

Протокол SSL (Secure Sockets Layer) был разработан компанией Netscape. Начиная с версии 3, протокол развивается как открытый стандарт TLS (Transport Layer Security). Протокол SSL/TLS обеспечивает защищённое соединение по незащищённому каналу связи на прикладном уровне модели TCP/IP. Протокол встраивается между прикладным и транспортным уровнями стека протоколов TCP/IP. Для обозначения <<новых>> протоколов, полученных с помощью инкапсуляции прикладного уровня (HTTP\index{протокол!HTTP}, FTP\index{протокол!FTP}, SMTP\index{протокол!SMTP}, POP3\index{протокол!POP3}, IMAP\index{протокол!IMAP} и т.~д.) в SSL/TLS, к обозначению добавляют суффикс <<S>> (<<Secure>>): HTTPS\index{протокол!HTTPS}, FTPS\index{протокол!FTPS}, POP3S\index{протокол!POP3S}, IMAPS\index{протокол!IMAPS} и т.~д.

Протокол обеспечивает следующее:
\begin{itemize}
    \item Одностороннюю или взаимную аутентификацию клиента и сервера по открытым ключам сертификата X.509. В Интернете, как правило, делается \textbf{односторонняя} аутентификация веб-сервера браузеру клиента, то есть только веб-сервер предъявляет сертификат (открытый ключ и ЭП к нему от вышележащего УЦ);
    \item Создание сеансовых симметричных ключей для шифрования и кода аутентификации сообщения для передачи данных в обе стороны;
    \item Конфиденциальность -- блоковое или потоковое шифрование передаваемых данных в обе стороны;
    \item Целостность -- аутентификацию отправляемых сообщений в обе стороны имитовставкой\index{имитовставка} $\HMAC(K,M)$, описанной ранее.
\end{itemize}

Рассмотрим протокол TLS последней версии 1.2.


\subsection{Протокол <<рукопожатия>>}

Протокол <<рукопожатия>> (Handshake Protocol) производит аутентификацию и создание сеансовых ключей между клиентом $C$ и сервером $S$.

\begin{enumerate}
    \item $C \rightarrow S$:
        \begin{enumerate}
            \item ClientHello: ~ 1) URI сервера, ~ 2) одноразовая метка $N_C$\index{одноразовая метка}, 3) ~ поддерживаемые алгоритмы шифрования, кода аутентификации сообщений, хэширования, ЭП и сжатия.
        \end{enumerate}

    \item $C \leftarrow S$:
        \begin{enumerate}
            \item ServerHello: одноразовая метка $N_S$, поддерживаемые алгоритмы сервера;

            После обмена набором желательных алгоритмов сервер и клиент по единому правилу выбирают общий набор алгоритмов.
            \item Server Certificate: сертификат X.509v3 сервера с запрошенным URI (URI нужен в случае нескольких виртуальных веб-серверов с разными URI на одном узле c одним IP адресом);
            \item Server Key Exchange Message: информация для создания предварительного общего секрета $premaster$ длиной 48 байт в виде: ~ 1) либо обмена по протоколу Диффи~---~Хеллмана\index{протокол!Диффи~---~Хеллмана} с клиентом (сервер отсылает $(g, g^a)$), ~ 2) либо по другому алгоритму с открытым ключом, ~ 3) либо разрешения клиенту выбрать ключ;
            \item Электронная подпись к Server Key Exchange Message на ключе сертификата сервера для аутентификации сервера клиенту;
            \item Certificate Request: опциональный запрос сервером сертификата клиента;
            \item Server Hello Done: идентификатор конца транзакции.
        \end{enumerate}

    \item $C \rightarrow S$:
        \begin{enumerate}
            \item Client Certificate: сертификат X.509v3 клиента, если он был запрошен сервером;
            \item Client Key Exchange Message: информация для создания предварительного общего секрета $premaster$ длиной 48 байт в виде: ~ 1) либо обмена по протоколу Диффи~---~Хеллмана\index{протокол!Диффи~---~Хеллмана} с сервером (клиент отсылает $g^b$, в результате обе стороны вычисляют ключ $premaster = g^{ab}$), ~ 2) либо по другому алгоритму, ~ 3) либо ключа, выбранного клиентом и зашифрованного на открытом ключе из сертификата сервера;
            \item Электронная подпись к Client Key Exchange Message на ключе сертификата клиента для аутентификации клиента серверу (если клиент использует сертификат);
            \item Certificate Verify: результат проверки сертификата сервера;
            \item Change Cipher Spec: уведомление о смене сеансовых ключей;
            \item Finished: идентификатор конца транзакции.
        \end{enumerate}

    \item $C \leftarrow S$:
        \begin{enumerate}
            \item Change Cipher Spec: уведомление о смене сеансовых ключей;
            \item Finished: идентификатор конца транзакции.
        \end{enumerate}
\end{enumerate}

%      http://tools.ietf.org/html/rfc5246#page-37

%      struct {
%          ProtocolVersion client_version;
%          Random random;
%          SessionID session_id;
%          CipherSuite cipher_suites<2..2^16-2>;
%          CompressionMethod compression_methods<1..2^8-1>;
%          select (extensions_present) {
%              case false:
%                  struct {};
%              case true:
%                  Extension extensions<0..2^16-1>;
%          };
%      } ClientHello;

%      struct {
%          ProtocolVersion server_version;
%          Random random;
%          SessionID session_id;
%          CipherSuite cipher_suite;
%          CompressionMethod compression_method;
%          select (extensions_present) {
%              case false:
%                  struct {};
%              case true:
%                  Extension extensions<0..2^16-1>;
%          };
%      } ServerHello;

%      struct {
%          ASN.1Cert certificate_list<0..2^24-1>;
%      } Certificate;

%      struct {
%          select (KeyExchangeAlgorithm) {
%              case dh_anon:
%                  ServerDHParams params;
%              case dhe_dss:
%              case dhe_rsa:
%                  ServerDHParams params;
%                  digitally-signed struct {
%                      opaque client_random[32];
%                      opaque server_random[32];
%                      ServerDHParams params;
%                  } signed_params;
%              case rsa:
%              case dh_dss:
%              case dh_rsa:
%                  struct {} ;
%                 /* message is omitted for rsa, dh_dss, and dh_rsa */
%              /* may be extended, e.g., for ECDH -- see [TLSECC] */
%          };
%      } ServerKeyExchange;

%      struct {
%          ClientCertificateType certificate_types<1..2^8-1>;
%          SignatureAndHashAlgorithm
%            supported_signature_algorithms<2^16-1>;
%          DistinguishedName certificate_authorities<0..2^16-1>;
%      } CertificateRequest;

%      struct {
%          select (KeyExchangeAlgorithm) {
%              case rsa:
%                  EncryptedPreMasterSecret;
%              case dhe_dss:
%              case dhe_rsa:
%              case dh_dss:
%              case dh_rsa:
%              case dh_anon:
%                  ClientDiffieHellmanPublic;
%          } exchange_keys;
%      } ClientKeyExchange;

%      struct {
%           digitally-signed struct {
%               opaque handshake_messages[handshake_messages_length];
%           }
%      } CertificateVerify;

%      struct {
%          opaque verify_data[verify_data_length];
%      } Finished;

Одноразовая метка $N_C$ состоит из 32 байт. Первые 4 байта содержат текущее время (gmt_unix_time), оставшиеся байты -- псевдослучайную последовательность, которую формирует криптографически стойкий генератор псевдослучайных чисел.

Предварительный общий секрет $premaster$ длиной 48 байт вместе с одноразовыми метками используется как инициализирующее значение генератора $PRF$ для получения общего секрета $master$ тоже длиной 48 байт:
    \[ master = PRF(premaster, ~\text{текст ''master secret''}, ~ N_C + N_S) .\]

И наконец, уже из секрета $master$ таким же способом генерируется 6 окончательных сеансовых ключей, следующих друг за другом в битовой строке:
    \[ \{ (K_{E,1} ~\|~ K_{E,2}) ~\|~ (K_{\MAC,1} ~\|~ K_{\MAC,2}) ~\|~ (IV_1 ~\|~ IV_2) \} = \]
        \[ = PRF(master, ~\text{текст ''key expansion''}, ~ N_C + N_S), \]
где $K_{E,1}, ~ K_{E,2}$ -- два ключа симметричного шифрования, ~ $K_{\MAC,1}, ~ K_{\MAC,2}$ -- два ключа имитовставки\index{имитовставка}, ~ $IV_1, ~IV_2$ -- два инициализирующих вектора режима сцепления блоков\index{вектор инициализации}. Ключи с индексом 1 используются для коммуникации от клиента к серверу, с индексом 2 -- от сервера к клиенту.


\subsection{Протокол записи}

Протокол записи (Record Protocol) определяет формат TLS-пакетов для вложения в TCP-пакеты.

\begin{enumerate}
    \item Исходными сообщениями $M$ для шифрования являются пакеты протокола следующего уровня в модели OSI: HTTP\index{протокол!HTTP}, FTP\index{протокол!FTP}, IMAP\index{протокол!IMAP} и т.~д.;
    \item Сообщение $M$ разбивается на блоки $m_i$ не более 16 кБ;
    \item Блоки $m_i$ сжимаются алгоритмом компрессии в блоки $z_i$;
    \item Вычисляется имитовставка\index{имитовставка} для каждого блока $z_i$ и добавляется в конец блоков: $a_i = z_i ~\|~ \HMAC(K_{\MAC}, z_i)$;
    \item Блоки $a_i$ шифруются симметричным алгоритмом с ключом $K_E$ в некотором режиме сцепления блоков с инициализирующим вектором $IV$ в полное сжатое аутентифицированное зашифрованное сообщение $C$;
    \item К шифротексту $C$ добавляется заголовок протокола записи TLS и в результате получается TLS-пакет для вложения в TCP-пакет.
\end{enumerate}


\section{Защита IPsec на сетевом уровне}
\selectlanguage{russian}

Набор протоколов IPsec (Internet Protocol Security)\index{протокол!IPsec}~\cite{rfc4301} является неотъемлемой частью IPv6\index{протокол!IPv6} и дополнительным необязательным расширением IPv4. IPsec обеспечивает защиту данных на сетевом уровне IP-пакетов.

IPsec определяет:
\begin{itemize}
    \item первичную аутентификацию сторон и управление сеансовыми ключами (протокол IKE, Internet Key Exchange),\index{протокол!IKE}
    \item шифрование с аутентификацией (протокол ESP, Encapsulating Security Payload),\index{протокол!ESP}
    \item только аутентификацию сообщений (протокол AH, Authentication Header)\index{протокол!AH}.
\end{itemize}
Основное (современное) применение этих протоколов состоит в построении VPN\index{протокол!VPN} (Virtual Private Network --- виртуальная частная сеть) при использовании IPsec в так называемом туннельном режиме.

Аутентификация в режимах ESP и AH определяется по-разному. Аутентификация в ESP гарантирует целостность только зашифрованных полезных данных (пакетов следующего уровня после IP). Аутентификация AH гарантирует целостность всего IP-пакета (за исключением полей, изменяемых в процессе передачи пакета по сети).

\subsection{Протокол создания ключей IKE}

%http://www.ietf.org/rfc/rfc4306.txt

Протокол IKE версии 2 (Internet Key Exchange)\index{протокол!IKE}~\cite{rfc4306}, по существу, можно описать следующим образом. Пусть $I$ -- инициатор соединения, $R$ -- отвечающая сторона.

Протокол состоит из двух фаз. Первая фаза очень похожа на установление соединения в SSL/TLS: она включает возможный обмен сертификатами $C_I, C_R$ стандарта X.509 для аутентификации (или альтернативную аутентификацию по общему заранее созданному секретному ключу) и создание общих предварительных сеансовых ключей протокола IKE по протоколу Диффи~---~Хеллмана\index{протокол!Диффи~---~Хеллмана}. Сеансовые ключи протокола IKE служат для шифрования и аутентификации сообщений второй фазы. Вторая фаза создаёт сеансовые ключи для протоколов ESP, AH, то есть ключи для шифрования конечных данных. Сообщения второй фазы также используются для смены ранее созданных сеансовых ключей, и в этом случае протокол сразу начинается со второй фазы с применением ранее созданных сеансовых ключей протокола IKE.

\begin{enumerate}
    \item Создание предварительной защищённой связи для протокола IKE и аутентификация сторон.
        \begin{enumerate}
            \item $I \rightarrow R$: ~ $\left(g^{x_I}\right.$, одноразовая метка $N_I$, идентификаторы поддерживаемых криптографических алгоритмов$\left.\right)$.
                % HDR, SAi1, KEi, Ni   -->

            \item $I \leftarrow R$: ~$\left(g^{x_R}\right.$, одноразовая метка $N_R$, идентификаторы выбранных алгоритмов, запрос сертификата $C_I\left.\right)$.
                % <--    HDR, SAr1, KEr, Nr, [CERTREQ]

                Протокол Диффи~---~Хеллмана\index{протокол!Диффи~---~Хеллмана} оперирует с генератором $g=2$ в группе $\Z_p^*$ для одного из двух фиксированных $p$ длиной 768 или 1024 бита. После обмена элементами $g^{x_I}$ и $g^{x_R}$ обе стороны обладают общим секретом $g^{x_I x_R}$.

                Одноразовые метки $N_I, N_R$ созданы криптографическим генератором псевдослучайных чисел $PRF$.

                После данного сообщения стороны договорились об используемых алгоритмах и создали общие сеансовые ключи:
                    \[ seed = PRF(N_i ~\|~ N_r, ~g^{x_I x_R}), \]
                    \[ \{ K_d \| Ka_I \| Ka_R \| Ke_I \| Ke_R
                        % \| Kp_I \| Kp_R
                        \} = PRF(seed, ~ N_i ~\|~ N_r), \]
                где $Ka_I, Ka_R$ -- ключи кода аутентификации для связи в оба направления, ~ $Ke_I, Ke_R$ -- ключи шифрования сообщений для двух направлений, ~ $K_d$ -- инициирующее значение генератора $PRF$ для создания сеансовых ключей окончательной защищённой связи, ~ функцией $PRF(x)$ обозначается выход генератора с инициализирующим значением $x$.
                %$Kp_I, Kp_R$ --  which areused when generating an AUTH payload.

                Дальнейший обмен данными зашифрован алгоритмом AES в режиме CBC со случайно выбранным инициализирующим вектором $IV$ на сеансовых ключах $Ke$  и аутентифицирован имитовставкой\index{имитовставка} на ключах $Ka$. Введём обозначения для шифрования сообщения $m$ со сцеплением блоков $E_{Ke_X}(m)$ и совместного шифрования и добавления кода аутентификации сообщений $\langle m \rangle_X$ для исходящих данных от стороны $X$:
                    \[  E_{Ke_X}(m) = IV ~\|~ E_{Ke_X}(IV ~\|~ m), \]
                    \[  \langle m \rangle_X = E_{Ke_X}(m) ~\|~ \HMAC(Ka_X, ~ E_{Ke_X}(m)). \]

            \item $I \rightarrow R$: ~ $\langle ID_I, ~ C_I, ~\text{запрос сертификата}~ C_R, ~ ID_R, ~ A_I \rangle_I$.
                % HDR, SK {IDi, [CERT,] [CERTREQ,] [IDr,] AUTH, SAi2, TSi, TSr}     -->

                По значениям идентификаторов $ID_I, ID_R$ сторона $R$ проверяет знание стороной $I$ ключей $Ke, Ka$.

                Поле $A_I$ обеспечивает аутентификацию стороны $I$ стороне $R$ по одному из двух способов. Если используются сертификаты, то $I$ показывает, что обладает закрытым ключом, парным открытому ключу сертификата $C_I$, подписывая сообщение $data$:
                    \[ A_I = \textrm{ЭП}(data). \]
                Сторона $R$ также проверяет сертификат $C_I$ по цепочке до доверенного сертификата верхнего уровня.

                Второй вариант аутентификации  -- по общему секретному симметричному ключу аутентификации $K_{IR}$, который заранее был создан $I$ и $R$, как в Kerberos. Сторона $I$ показывает, что знает общий секрет, вычисляя
                    \[ A_I = PRF( PRF(K_{IR}, ~ \text{текст ''Key Pad for IKEv2''}), ~ data). \]
                Сторона $R$ сравнивает присланное значение $A_I$ с вычисленным и убеждается, что $I$ знает общий секрет.

                Сообщение $data$ -- это открытое сообщение данной транзакции, за исключением нескольких полей.

            \item $I \leftarrow R$: ~ $\langle ID_R, ~ C_R, ~ A_R \rangle_R$.
                % HDR, SK {IDr, [CERT,] AUTH, SAr2, TSi, TSr}

                Производится аутентификация стороны $R$ стороне $I$ аналогичным образом.
        \end{enumerate}

    \item Создание защищённой связи для протоколов ESP, AH, то есть ключей шифрования и кодов аутентификации конечных полезных данных. Фаза повторяет первые две транзакции первой фазы с созданием ключей по одноразовой метке $N'$ и протоколу Диффи~---~Хеллмана\index{протокол!Диффи~---~Хеллмана} с секретными ключами $x'$.
        \begin{enumerate}
            \item $I \rightarrow R$: ~ $\langle g^{x'_I}$, одноразовая метка $N'_I$, поддерживаемые алгоритмы для ESP, AH$\rangle_I$.
                % HDR, SK {[N], SA, Ni, [KEi], [TSi, TSr]}             -->
            \item $I \rightarrow R$: ~ $\langle g^{x'_R}$, одноразовая метка $N'_R$, выбранные алгоритмы для ESP, AH$\rangle_R$.
                % <--    HDR, SK {SA, Nr, [KEr], [TSi, TSr]}
        \end{enumerate}
        По окончании второй фазы обе стороны имеют общие секретные ключи $Ke, Ka$ для шифрования и кодов аутентификации в двух направлениях, от стороны $I$ и от стороны $R$:
            \[ \{Ka'_I ~\|~ Ka'_R ~\|~ Ke'_I ~\|~ Ke'_R \} = PRF(K_d, ~ g^{x'_I x'_R} ~\|~ N'_I ~\|~ N'_r). \]
\end{enumerate}

Итогом протокола IKE является набор сеансовых ключей для шифрования $Ke'_I, ~ Ke'_R$ и кодов аутентификации $Ka'_I, ~ Ka'_R$ в протоколах ESP и AH.


\subsection{Таблица защищённых связей}

\textbf{Защищённая связь} (Security Association, SA) является \emph{однонаправленной} от отправителя к получателю и характеризуется тремя основными параметрами:
\begin{itemize}
    \item индексом параметров защиты -- уникальное 32-битовое число, входит в заголовок ESP- и AH- пакетов;
    \item IP-адресом стороны-отправителя;
    \item идентификатором применения ESP- или AH-протокола.
\end{itemize}

Защищённые связи хранятся в таблице защищённых связей со следующими полями.
\begin{itemize}
    \item Счётчик порядкового номера, входит в заголовок ESP- и AH-пакетов.
    \item Окно защиты от воспроизведения -- скользящий буфер порядковых номеров пакетов для защиты от пропуска и повтора пакетов.
    \item Информация протокола ESP и AH -- алгоритмы, ключи, время действия ключей.
    \item Режим протокола: транспортный или туннельный.
\end{itemize}

По индексу параметров защиты, находящемся в заголовке ESP- и AH-пакетов, получатель из таблицы защищённых связей извлекает параметры (названия алгоритмов, ключи и т.~д.), производит проверки счётчиков, аутентифицирует и расшифровывает вложенные данные для принятого IP-пакета.
%Или создаёт зашифрованный аутентифицированный IP пакет, который включает индекс параметров защиты, чтобы получатель мог из своей таблицы защищённых связей извлечь ключи для аутентификации и расшифрования.

Протоколы ESP и AH можно применять к IP-пакету в трёх вариантах:
\begin{itemize}
    \item только ESP-протокол,
     \item только AH-протокол,
    \item последовательное применение ESP и AH протоколов.
\end{itemize}
Подчеркнем, что только AH-протокол гарантирует целостность всего IP-пакета, поэтому для организации виртуальной сети VPN\index{сеть!виртуальная частная}, как правило, применяется третий вариант (последовательно ESP и AH протоколы).


\subsection{Транспортный и туннельный режимы}

Протоколы ESP, AH могут применяться в транспортном режиме, когда исходный IP-пакет расширяется заголовками и концевиками протоколов ESP, AH, или в туннельном режиме, когда весь IP-пакет вкладывается в новый IP-пакет, который включает заголовки и концевики ESP, AH.

Новый IP-пакет в туннельном режиме может иметь другие IP-адреса, отличные от оригинальных. Именно это свойство используется для построения виртуальных частных сетей (Virtual Private Network, VPN)\index{сеть!виртуальная частная}. IP-адресом нового пакета является IP-адрес IPsec шлюза виртуальной сети. IP-адрес вложенного пакета является локальным адресом виртуальной сети. IPsec шлюз производит преобразование IPsec пакетов в обычные IP-пакеты виртуальной сети и наоборот.

Схемы транспортного и туннельного режимов показаны ниже отдельно для ESP- и AH-протоколов.


\subsection{Протокол шифрования и аутентификации ESP}

Протокол ESP\index{протокол!ESP} определяет шифрование и аутентификацию вложенных в IP-пакет сообщений в формате, показанном на рис.~\ref{fig:ipsec-esp}.

\begin{figure}[!ht]
	\centering
	\includegraphics[width=0.9\textwidth]{pic/ipsec-esp}
	\caption{Формат ESP пакета\label{fig:ipsec-esp}}
\end{figure}

Шифрование вложенных данных производится в режиме CBC алгоритмом AES на ключе $Ke'$ с псевдослучайным вектором инициализации IV, вставленном перед зашифрованными данными.

Аутентификатор сообщения определяется как усеченное до 96 бит значение $\HMAC(Ka', m)$, вычисленное стандартным способом.

\begin{figure}[!ht]
	\centering
	\includegraphics[width=0.9\textwidth]{pic/ipsec-esp-modes}
	\caption{Применение ESP протокола к пакету IPv6\label{fig:ipsec-esp-modes}}
\end{figure}

На рис.~\ref{fig:ipsec-esp-modes} показано применение протокола в транспортном и туннельном режимах.


\subsection{Протокол аутентификации AH}

Протокол AH определяет аутентификацию всего IP пакета в формате, показанном на рис.~\ref{fig:ipsec-ah}.

\begin{figure}[!ht]
	\centering
	\includegraphics[width=0.7\textwidth]{pic/ipsec-ah}
	\caption{Заголовок AH пакета\label{fig:ipsec-ah}}
\end{figure}

Аутентификатор сообщения определяется так же, как и в протоколе ESP -- усеченное до 96 бит значение $\HMAC(Ka', m)$, вычисленное стандартным способом.

\begin{figure}[!ht]
	\centering
	\includegraphics[width=0.9\textwidth]{pic/ipsec-ah-modes}
	\caption{Применение протокола AH к пакету IPv6\label{fig:ipsec-ah-modes}}
\end{figure}

На рис.~\ref{fig:ipsec-ah-modes} показано применение протокола в транспортном и туннельном режимах.


\section[Защита персональных данных в мобильной связи]{Защита персональных данных в \protect\\ мобильной связи}

\input{gsm2}

\subsection{UMTS (3GSM)}
\selectlanguage{russian}

В третьем поколении сети GSM, называемом UMTS, защищённость немного улучшена. Общая схема аутентификации (рис.~\ref{fig:gsm3}) осталась примерно такой же, как и в GSM2. Жирным шрифтом на рисунке выделены новые добавленные элементы по сравнению с GSM2.
\begin{enumerate}
    \item Производится взаимная аутентификация SIM-карты и центра аутентификации по токенам $\textrm{RES}$ и $\MAC$.
    \item Добавлены проверка целостности и аутентификация данных (имитовставка\index{имитовставка}).
    \item Используются новые алгоритмы создания ключей, шифрования и имитовставки\index{имитовставка}.
    \item Добавлены счётчики на SIM-карте $\textrm{SQN}_\textrm{T}$ и в Центре аутентификации $\textrm{SQN}_\textrm{Ц}$ для защиты от атак воспроизведения. Значения увеличиваются при каждой попытке аутентификации и должны примерно совпадать.
    \item Увеличена длина ключа шифрования до 128 бит.
\end{enumerate}

\begin{figure}[!ht]
	\centering
	\includegraphics[width=\textwidth]{pic/gsm3}
	\caption{Взаимная аутентификация и шифрование в UMTS (3GSM)\label{fig:gsm3}}
\end{figure}

Обозначения на рис.~\ref{fig:gsm3} следующие:
\begin{itemize}
    \item $K$ -- общий секретный 128-битовый ключ SIM-карты и центра аутентификации.
    \item $\textrm{RAND}$ -- 128-битовое псевдослучайное число, создаваемое Центром аутентификации.
    \item $\textrm{SQN}_\textrm{T}, \textrm{SQN}_\textrm{Ц}$ -- 48-битовые счётчики для защиты от атак воспроизведения.
    \item $\textrm{AMF}$ -- 16-битовое значение окна для проверки синхронизации счётчиков.
    \item $CK, IK, AK$ -- 128-битовые ключи шифрования данных $CK$, кода аутентификации данных $IK$, гаммы значения счётчика $AK$.
    \item $\MAC, \textrm{XMAC}$ -- 128-битовые аутентификаторы Центра SIM-карте.
    \item $\textrm{RES}, \textrm{XRES}$ -- 128-битовые аутентификаторы SIM-карты Центру.
    \item $\textrm{AUTN}$ -- вектор аутентификации.
\end{itemize}

Алгоритмы $fi$ не фиксированы стандартом и выбираются при реализациях.

Из оставшихся недостатков защиты персональных данных можно перечислить:
\begin{enumerate}
    \item Уникальный идентификатор SIM-карты IMSI по-прежнему передаётся в открытом виде, что позволяет идентифицировать абонентов по началу сеанса регистрации SIM-карты в сети;
    \item Шифрование и аутентификация производится только между телефоном и базовой станцией, а не между двумя телефонами. Это является необходимым условием для подключения СОРМ (Система технических средств для обеспечения функций оперативно-розыскных мероприятий) по закону <<О связи>>. С другой стороны, это повышает риск нарушения конфиденциальности персональных данных;
    \item Алгоритм шифрования данных A5/3 (KASUMI) на 128-битовом ключе теоретически взламывается атакой на основе известного открытого текста для 64 MB данных с использованием 1 GiB памяти $2^{32}$ операциями (2 часа на обычном ПК).
\end{enumerate}


%\section{Беспроводная сеть Wi-Fi}
%\subsection{WPA-PSK2, 802.11n, Radix?}
%\subsection{Wimax 802.16(?)}

\chapter{Аутентификация пользователя}


\section{Многофакторная аутентификация}

Для защищённых приложений применяется \textbf{многофакторная} аутентификация одновременно по факторам различной природы:
\begin{enumerate}
    \item Свойство, которым обладает субъект. Например, биометрия, природные уникальные отличия: лицо, радужная оболочка глаз, папиллярные узоры, последовательность ДНК;
    \item Знание -- информация, которую знает субъект. Например, пароль, PIN-код;
    \item Владение -- вещь, которой обладает субъект. Например, электронная или магнитная карта, флеш-память.
%    \item Факторы присвоения. Например, номер машины, RFID-метка.
\end{enumerate}

В обычных массовых приложениях из-за удобства использования применяется аутентификация только по \textbf{паролю}\index{пароль}, который является общим секретом пользователя и информационной системы. Биометрическая аутентификация по отпечаткам пальцев применяется существенно реже. Как правило, аутентификация по отпечаткам пальцев является дополнительным, а не вторым обязательным фактором (тоже из-за удобства ее использования).

%Так же явно или неявно используется аутентификация по факторам:
%\begin{enumerate}
%    \item Социальная сеть. Доверие к индивидууму в личном общении или интернет на основании общих связей.
%    \item Географическое положение. Например, для проверки оплаты товаров по кредитной карте.
%    \item Время. Доступ к сервисам или местам только в определённое время.
%    \item И др.
%\end{enumerate}


\section[Энтропия и криптостойкость паролей]{Энтропия и криптостойкость \protect\\ паролей}

Стандартный набор символов паролей, которые можно набрать на клавиатуре, используя английские буквы и небуквенные символы, состоит из $D=94$ символов. При длине пароля $L$ символов и предположении равновероятного использования символов энтропия паролей равна
    \[ H = L \log_2 D. \]

Клод Шеннон, исследуя энтропию символов английского текста, изучал вероятность успешного предсказания людьми следующего символа по первым нескольким символам слов или текста. В результате Шеннон получил оценку энтропии первого символа $s_1$ текста порядка $H(s_1) \approx 4{,}6$--$4{,}7$ бит/символ и оценки энтропий последующих символов, постепенно уменьшающиеся до $H(s_9) \approx 1{,}5$ бит/символ для 9-го символа. Энтропия для длинных текстов литературных произведений получила оценку $H(s_\infty) \approx 0{,}4$ бит/символ.

Статистические исследования баз паролей показывают, что наиболее часто используются буквы <<a>>, <<e>>, <<o>>, <<r>> и цифра <<1>>.

NIST использует следующие рекомендации для оценки энтропии паролей\index{энтропия!пароля}, создаваемых людьми.
\begin{enumerate}
    \item Энтропия первого символа $H(s_1) = 4$ бит/символ.
    \item Энтропия со 2-го по 8-й символы $H(s_{2 \leq i \leq 8}) = 2$ бит/символ.
    \item Энтропия с 9-го по 20-й символы $H(s_{9 \leq i \leq 20}) = 1{,}5$ бит/символ.
    \item Энтропия с 21-го символа $H(s_{i \geq 21}) = 1$ бит/символ.
    \item Проверка композиции на использование символов разных регистров и небуквенных символов добавляет до 6 бит энтропии пароля.
    \item Словарная проверка на слова и часто используемые пароли добавляет до 6 бит энтропии для коротких паролей. Для 20-символьных и более длинных паролей прибавка к энтропии 0 бит.
\end{enumerate}

Для оценки энтропии пароля нужно сложить энтропии символов $H(s_i)$ и сделать дополнительные надбавки, если пароль удовлетворяет тестам на композицию и отсутствие в словаре.

\begin{table}[!ht]
    \centering
    \caption{Оценка NIST предполагаемой энтропии паролей\label{tab:password-entropy}}
    \resizebox{\textwidth}{!}{ \begin{tabular}{|c||c|c|c||c|}
        \hline
        \multirow{2}{*}{\parbox{1.5cm}{Длина пароля, символы}} & \multicolumn{3}{|c||}{\parbox{6cm}{Энтропия паролей пользователей по критериям NIST}} & \multirow{2}{*}{\parbox{2.5cm}{Энтропия случайных равновероятных паролей}} \\
        \cline{2-4}
        & \parbox{1.5cm}{Без проверок} & \parbox{2cm}{Словарная проверка} & \parbox{2.5cm}{Словарная и композиционная проверка} & \\
        \hline
        4  & 10 & 14 & 16 & 26.3 \\
        6  & 14 & 20 & 23 & 39.5 \\
        8  & 18 & 24 & 30 & 52.7 \\
        10 & 21 & 26 & 32 & 65.9 \\
        12 & 24 & 28 & 34 & 79.0 \\
        16 & 30 & 32 & 38 & 105.4 \\
        20 & 36 & 36 & 42 & 131.7 \\
        24 & 40 & 40 & 46 & 158.0 \\
        30 & 46 & 46 & 52 & 197.2 \\
        40 & 56 & 56 & 62 & 263.4 \\
        \hline
    \end{tabular} }
\end{table}

В табл.~\ref{tab:password-entropy} приведена оценка NIST на величину энтропии пользовательских паролей в зависимости от их длины и сравнение с энтропией случайных паролей с равномерным распределением символов из набора в $D=94$ символов клавиатуры. Вероятное число попыток для подбора пароля составляет $O(2^H)$. Из таблицы видно, что по критериям NIST энтропия реальных паролей в 2--4 раза меньше энтропии случайных паролей с равномерным распределением символов.

\example
Оценим общее количество существующих паролей. Население Земли -- 7 млрд. Предположим, что все население использует компьютеры, Интернет, и у каждого человека по 10 паролей. Общее количество существующих паролей -- $7 \cdot 10^{10} \approx 2^{36}$.
%Следовательно, \emph{реальная энтропия паролей не превышает 36 бит}.

Имея доступ к наиболее массовым интернет-сервисам с количеством пользователей десятки и сотни миллионов, в которых пароли часто хранятся в открытом виде из-за необходимости обновления ПО и, в частности, выполнения аутентификации, мы 1) имеем базу паролей, покрывающую существенную часть пользователей, 2) можем статистически построить правила генерирования паролей. Даже если пароль хранится в защищённом виде, то при вводе пароль, как правило, в открытом виде пересылается по Интернету, и все преобразования пароля для аутентификации осуществляет интернет-сервис, а не веб-браузер. Следовательно, интернет-сервис имеет доступ к исходному паролю.
\exampleend

В 2002 г. был подобран ключ для 64-битового блокового шифра RC5 сетью \texttt{distributed.net} персональных компьютеров, выполнявших вычисления в фоновом режиме. Суммарное время вычислений всех компьютеров -- 1757 дней, было проверено 83\% пространства всех ключей. Это означает, что пароли с оценочной энтропией менее 64 бит, то есть \emph{все пароли} до 40 символов по критериям NIST, могут быть подобраны в настоящее время. Конечно, с оговорками на то, что 1) нет ограничений на количество и скорость попыток аутентификаций, 2) алгоритм генерирования вероятных паролей эффективен.

Строго говоря, использование даже 40-символьного пароля для аутентификации или в качестве ключа блокового шифрования является небезопасным.


\subsubsection{Число паролей}

Приведём различные оценки числа паролей, создаваемых людьми.

Пароли, создаваемые людьми, основаны на словах или закономерностях естественного языка. В английском языке всего около $1\ 000\ 000 \approx 2^{20}$ слов, включая термины.

%http://www.springerlink.com/content/bh216312577r6w64/fulltext.pdf
%http://www.antimoon.com/forum/2004/4797.htm

Используемые слоги английского языка имеют вид V, CV, VC, CVV, VCC, CVC, CCV, CVCC, CVCCC, CCVCC, CCCVCC, где C -- согласная (consonant), V -- гласная (vowel). 70\% слогов имеют структуру VC или CVC. Общее число слогов $S = 8000 - 12000$. Средняя длина слога -- 3 буквы.

Предполагая равновероятное распределение всех слогов английского языка, для числа паролей из $r$ слогов получим верхнюю оценку
    \[ N_1 = S^r = 2^{13 r} \approx 2^{4.3 L_1}. \]
Средняя длина паролей составит
    \[ L_1 \approx 3 r. \]

Теперь предположим, что пароли могут состоять только из 2--3 буквенных слогов вида CV, VC, CVV, VCC, CVC, CCV с равновероятным распределением символов. Подсчитаем число паролей $N_2$, которые могут быть построены из $r$ таких слогов. В английском алфавите $n_v = 10, n_c = 16, n = n_v + n_c = 26$. Верхняя оценка числа $r$-слоговых паролей:
    \[ N_2 = (n_c n_v + n_v n_c + n_c n_v n_v + n_v n_c n_c + n_c n_v n_c + n_c n_c n_v)^r \approx \]
        \[ \approx \left( n_c n_v(3 n_c + n_v) \right)^r, \]
    \[ N_2 \approx \left( \frac{n^3}{2} \right)^r \approx 2^{13 r} \approx 2^{4.3 L_2}. \]
Средняя длина паролей:
    \[ L_2 = \frac{n_c n_v(2 + 2 + 3 n_v + 3 n_c + 3 n_c + 3 n_c)}{n_c n_v (1 + 1 + n_v + n_c + n_c + n_c)} \cdot r \approx 3 r. \]

Как видно, получились одинаковые оценки числа и длины паролей.

Подсчитаем верхние оценки числа паролей из $L$ символов, предполагая равномерное распределение символов из алфавита в $D$ символов: a) $D_1 = 26$ строчных буквы, б) все $D_2 = 94$ печатных символа клавиатуры (латиница и небуквенные символы):
    \[ N_3 = D_1^L \approx 2^{4.7 L}, \]
    \[ N_4 = D_2^L \approx 2^{6.6 L}. \]

\begin{table}[!ht]
    \centering
    \caption{Различные верхние оценки числа паролей\label{tab:password-number}}
    \resizebox{\textwidth}{!}{ \begin{tabular}{|c||c|c|c|}
        \hline
        \multirow{2}{*}{\parbox{1.5cm}{Длина пароля}} & \multicolumn{3}{|c|}{Число паролей} \\
        \cline{2-4}
            & \parbox{3cm}{На основе слоговой композиции} &
            \parbox{3cm}{Алфавит $D=26$ символов} &
            \parbox{3cm}{Алфавит $D=94$ символа} \\
        \hline \hline
        6  & $2^{26}$ & $2^{28}$ & $2^{39}$ \\
        9  & $2^{39}$ & $2^{42}$ & $2^{59}$ \\
        12 & $2^{52}$ & $2^{56}$ & $2^{79}$ \\
        15 & $2^{65}$ & $2^{71}$ & $2^{98}$ \\
        \hline
        21 & $2^{91}$ & $2^{99}$ & $2^{137}$ \\
        \hline
        39 & $2^{169}$ & $2^{183}$ & $2^{256}$ \\
        \hline
    \end{tabular} }
\end{table}

Из таблицы~\ref{tab:password-number} видно, что при доступном объёме вычислений в $2^{60 \ldots 70}$ операций, пароли вплоть до 15 символов, построенные на словах, слогах, изменениях слов, вставках цифр, небольшом изменении регистров и других простейших модификациях, могут быть найдены перебором на кластере (или ПК) в настоящее время.

Для достижения криптостойкости паролей, сравнимой со 128- или 256-битовым секретным ключом, требуется выбирать пароль из 20 и 40 символов соответственно, что, как правило, не реализуется из-за сложности запоминания и ввода без ошибок.


%Подсчитаем число паролей $N_1$, которые могут могут построены из $r$ ~ 2-3 буквенных слогов: $cv, vc, ccv, cvc, vcc$, где $c$ -- согласная, $v$ -- гласная. В английском алфавите $n_v = 10, n_c = 16, n = n_v + n_c = 26$. Число паролей
%    \[ N_1 = \left( n_v n_c (1 + 1 + n_c + n_c + n_c) \right)^r \approx 3^r n_v^r n_c^{2r}. \]
%Средняя длина паролей
%    \[ L = r \left( \frac{2 + 2 + 3 n_c + 3 n_c + 3 n_c}{1 + 1 + n_c + n_c + n_c} \right) \approx 3r. \]
%
%%Учтем, что $b \leq r$ символов могут быть заглавными: $N_1 \rightarrow N_2 < N_1 \binom{L}{b} \left( \frac{n}{n_v} \right)^b$. Вставим $d$ цифр в случайные места: $N_2 \rightarrow N_3 = N_2 (10 (1 + L))^d \approx N_2 (10 L)^d$.
%%
%%Общее число паролей
%%    \[ N = N_3 = 3^r 10^r 16^{2r} \binom{3r}{b} 2.6^b \left(10 \cdot 3 r \right)^d. \]
%%
%%\begin{table}[!ht]
%%    \centering
%%    \small
%%    \begin{tabular}{|c|c|c|c|c||cr|}
%%        \hline
%%        \parbox{1.3cm}{Слогов, $r$} & \parbox{1.8cm}{Заглавных букв, $b$} & \parbox{1.5cm}{Вставок цифр, $d$} & \parbox{2.8cm}{Средняя длина пароля, $L+d$} & \parbox{3cm}{Верхняя оценка числа паролей $N$} & \multicolumn{2}{|c|}{\parbox{3.2cm}{Число всех паролей}} \\
%%        \hline
%%        $2$ & $0$ & $0$ & $6$ & $2^{26}$ & $2^{36}$ & a-z \\
%%        $2$ & $2$ & $0$ & $6$ & $2^{32}$ & $2^{48}$ & A-Z, a-z \\
%%        $2$ & $2$ & $2$ & $8$ & $2^{45}$ & $2^{48}$ & A-Z, a-z, 0-9 \\
%%        \hline
%%        $3$ & $0$ & $0$ & $9$ & $2^{39}$ & $2^{54}$ & a-z \\
%%        $3$ & $3$ & $0$ & $9$ & $2^{49}$ & $2^{54}$ & A-Z, a-z \\
%%        $3$ & $3$ & $2$ & $11$ & $2^{63}$ & $2^{65}$ & A-Z, a-z, 0-9 \\
%%        \hline
%%        $4$ & $0$ & $0$ & $12$ & $2^{52}$ & $2^{93}$ & a-z \\
%%        $4$ & $3$ & $0$ & $12$ & $2^{64}$ & $2^{186}$ & A-Z, a-z \\
%%        $4$ & $3$ & $2$ & $14$ & $2^{78}$ & $2^{222}$ & A-Z, a-z, 0-9 \\
%%        \hline
%%    \end{tabular}
%%    \caption{Сравнение верхней оценки числа паролей, построенных на слогах, со всем доступным множеством паролей.}
%%    \label{tab:password-number}
%%\end{table}
%
%Учтем, что $b$ символов в пароле могут быть взяты не из 26-символьного алфавита строчных букв, а из всего алфавита в $D=94$ печатных символа клавиатуры (латиница и небуквенные символы):
%\[
%    \begin{array}{ll}
%    b=1 & N_1 \rightarrow N_2 = \frac{n_v}{n_v+n_c} 3^r n_v^{r-1} n_c^{2r} \cdot L. \]
%
%    \[ N_1 \rightarrow N_2 < N_1 \binom{L}{b} \left( \frac{D}{n_v} \right)^b. \]
%
%
%
%Общее число паролей
%    \[ N < 3^r n_v^r n_c^{2r} \binom{L}{b} \left( \frac{D}{n_v} \right)^b = 3^r 10^r 16^{2r} \binom{3r}{b} \left( \frac{94}{10} \right)^b. \]
%
%\begin{table}[!ht]
%    \centering
%    \small
%    \begin{tabular}{|c|c|c|c||cr|}
%        \hline
%        \parbox{1.5cm}{Слогов, $r$} & \parbox{3cm}{Средняя длина пароля, $L$} & \parbox{3cm}{Символов из всего алфавита, $b$} & \parbox{3cm}{Верхняя оценка числа паролей $N$} & \multicolumn{2}{|c|}{\parbox{3.2cm}{Число всех паролей, $D^L$}} \\
%        \hline
%        \multirow{3}{*}{2} & \multirow{3}{*}{6} & $0$ & $2^{26}$ & $2^{28}$ & a-z \\
%        & & $1$ & $2^{32}$ & $2^{34}$ & A-Z, a-z \\
%        & & $3$ & $2^{40}$ & $2^{39}$ & Весь алфавит \\
%        \hline
%        \multirow{3}{*}{3} & \multirow{3}{*}{9} & $0$ & $2^{39}$ & $2^{42}$ & a-z \\
%        & & $2$ & $2^{50}$ & $2^{51}$ & A-Z, a-z \\
%        & & $4$ & $2^{59}$ & $2^{59}$ & Весь алфавит \\
%        \hline
%        \multirow{3}{*}{4} & \multirow{3}{*}{12} & $0$ & $2^{52}$ & $2^{56}$ & a-z \\
%        & & $3$ & $2^{69}$ & $2^{68}$ & A-Z, a-z \\
%        & & $6$ & $2^{81}$ & $2^{77}$ & Весь алфавит \\
%        \hline
%    \end{tabular}
%    \caption{Сравнение верхней оценки числа паролей, построенных на слогах, со всем доступным множеством паролей в алфавите из $D$ символов.}
%    \label{tab:password-number}
%\end{table}
%
%Из таблицы~\ref{tab:password-number} видно, что при доступном объёме вычислений в $2^{60 \ldots 70}$ операций, пароли вплоть до 12 символов, построенные на словах, слогах, изменениях слов, вставках цифр, небольшого изменения регистров и другой простейшей обфускации, могут быть найдены перебором на кластере (или ПК) в настоящее время.


\subsubsection{Атака для подбора паролей и ключей шифрования}

В схемах аутентификации по паролю иногда используется хэширование и хранение хэша пароля на сервере. В таких случаях применима словарная атака или атака с применением заранее вычисленных таблиц для ускорения поиска.

Для нахождения пароля, прообраза хэш-функции, или для нахождения ключа блокового шифрования по атаке с выбранным шифротекстом (для одного и того же известного открытого текста и соответствующего шифротекста) может быть применён метод перебора с балансом между памятью и временем вычислений. Самый быстрый метод радужных таблиц (rainbow tables)\index{радужные таблицы}, 2003 г., заранее вычисляет следующие цепочки и хранит результат в памяти.

Для нахождения пароля, прообраза хэш-функции $H$, цепочка строится как
    \[ M_0 \xrightarrow{H(M_0)} h_0 \xrightarrow{R_0(h_0)} M_1 \ldots M_t \xrightarrow{H(M_t)} h_t \xrightarrow{R_t(h_t)} M_{t+1}, \]
где $R_i(h)$ -- функция редуцирования, преобразования хэша в пароль для следующего хэширования.

Для нахождения ключа блокового шифрования для одного и того же известного открытого текста $M$ таблица строится как
    \[ K_0 \xrightarrow{E_{K_0}(M)} c_0 \xrightarrow{R_0(c_0)} K_1 \ldots K_t \xrightarrow{E_{K_t}(M)} c_t \xrightarrow{R_t(c_t)} K_{t+1}, \]
где $R_i(c)$ -- функция редуцирования, преобразования шифротекста в новый ключ.

Функция редуцирования $R_i$ зависит от номера итерации, чтобы избежать дублирующиеся подцепочки, которые возникают в случае коллизий между значениями в разных цепочках в разных итерациях, если $R$ постоянна. Rainbow-таблица размера $(m \times 2)$ состоит из строк $(M_{0,j}, M_{t+1,j})$ или $(K_{0,j}, K_{t+1,j})$, вычисленных для разных значений стартовых паролей $M_{0,j}$ или $K_{0,j}$ соответственно.

Опишем атаку на примере нахождения прообраза $\overline{M}$ хэша $\overline{h} = H(\overline{M})$. На первой итерации исходный хэш $\overline{h}$ редуцируется в сообщение $\overline{h} \xrightarrow{R_t(\overline{h})} \overline{M}_{t+1} $ и сравнивается со всеми значениями последнего столбца $M_{t+1,j}$ таблицы. Если нет совпадения, переходим ко второй итерации. Хэш $\overline{h}$ дважды редуцируется в сообщение $\overline{h} \xrightarrow{R_{t-1}(\overline{h})} \overline{M}_t \xrightarrow{H(\overline{M}_t)} \overline{h}_t \xrightarrow{R_t(\overline{h}_t)} \overline{M}_{t+1}$ и сравнивается со всеми значениями последнего столбца $M_{t+1,j}$ таблицы. Если не совпало, то переходим к третьей итерации и т.~д. Если для $r$-кратного редуцирования сообщение $\overline{M}_{t+1}$ содержится в таблице во втором столбце, то из совпавшей строки берётся $M_{0,j}$, и вся цепочка пробегается заново для поиска искомого сообщения $\overline{M}: ~ \overline{h} = H(\overline{M})$.

Найдем вероятность нахождения пароля в таблице. Пусть мощность множества всех паролей $N$. Изначально в столбце $M_{0,j}$ содержится $m_0 = m$ различных паролей. Предполагая случайное отображение с пересечениями паролей $M_{0,j} \rightarrow M_{1,j}$, в $M_{1,j}$ будет $m_1$ различных паролей. Согласно задаче о размещении,
\[
    m_{i+1} = N \left( 1 - \left( 1 - \frac{1}{N} \right)^{m_i} \right) \approx N \left( 1 - e^{-\frac{m_i}{N}} \right).
\]
Вероятность нахождения пароля
\[
    P = 1 - \prod \limits_{i=1}^t \left( 1 - \frac{m_i}{N} \right).
\]

Чем больше таблица из $m$ строк, тем больше шансов найти пароль или ключ, выполнив в наихудшем случае   $O \left( m \frac{t(t+1)}{2} \right)$ операций.

Примеры применения атаки на хэш-функциях $\textrm{MD5}$\index{хэш-функция!MD5}, $\textrm{LM} \sim \textrm{DES}_{\textrm{Password}} (\textrm{const})$ приведены в табл.~\ref{tab:rainbow-tables}.

\begin{table}[!ht]
    \centering
    \caption{Атаки на радужных таблицах на \emph{одном} ПК\label{tab:rainbow-tables}}
    \resizebox{\textwidth}{!}{ \begin{tabular}{|c|c|c|c|c|c|c|}
        \hline
        \multirow{2}{*}{\parbox{1.0cm}{Длина, биты}} & \multicolumn{3}{|c|}{Пароль или ключ} &
            \multicolumn{3}{|c|}{Радужная таблица} \\
        \cline{2-7}
        & \parbox{1.2cm}{Длина, симв.} & \parbox{1cm}{Множе- ство} & \parbox{1cm}{Мощн- ость} &
            объём & \parbox{1.5cm}{Время вычисления таблиц} & \parbox{1.3cm}{Время поиска} \\
        \hline \hline
        \multicolumn{7}{|c|}{Хэш LM} \\
        \hline
        \multirow{3}{*}{$2 \times 56$} & \multirow{3}{*}{14} &
            A--Z & $2^{33}$ & 610 MB &  & 6 с \\
        & & A--Z, 0-9 & $2^{36}$ & 3 GB &  & 15 с \\
        & & все & $2^{43}$ & 64 GB & \parbox{1.5cm}{несколько лет} & 7 мин \\
        \hline \hline
        \multicolumn{7}{|c|}{Хэш MD5} \\
        \hline
        128 & 8 & a-z, 0-9 & $2^{41}$ & 36 GiB & - & 4 мин \\
        \hline
    \end{tabular} }
\end{table}


\section{Аутентификация по паролю}

Из-за малой энтропии пользовательских паролей во всех системах регистрации и аутентификации пользователей применяется специальная политика безопасности. Типичные минимальные требования:
\begin{enumerate}
    \item Длина пароля от 8 символов. Использование разных регистров и небуквенных символов в паролях. Запрет паролей из словаря слов или часто используемых паролей. Запрет паролей в виде дат, номеров машин и других номеров.
    \item Ограниченное время действия пароля. Обязательная смена пароля по истечении срока действия.
    \item Блокирование возможности аутентификации после нескольких неудачных попыток. Ограниченное число актов аутентификаций в единицу времени. Временная задержка перед выдачей результата аутентификации.
\end{enumerate}

Дополнительные рекомендации (требования) пользователям:
\begin{enumerate}
    \item Не использовать одинаковые или похожие пароли для разных систем, таких как электронная почта, вход в ОС, электронная платежная система, форумы, социальные сети. Пароль часто передается в открытом виде по сети. Пароль доступен администратору системы, возможны утечки конфиденциальной информации с серверов. Стараться выбирать случайные стойкие пароли.
    \item Не записывать пароли. Никому не сообщать пароль, даже администратору. Не передавать пароли по почте, телефону, Интернету и т.~д.
    \item Не использовать одну и ту же учётную запись для разных пользователей даже в виде исключения.
    \item Всегда блокировать компьютер, когда пользователь отлучается от него даже на короткое время.
\end{enumerate}

\section[Хранение паролей и аутентификация в ОС]{Хранение паролей и \protect\\ аутентификация в ОС}
\selectlanguage{russian}

Для усложнения подбора пароля и избежания словарной атаки используется добавление перед хэшированием к паролю <<соли>> -- случайной битовой строки. \textbf{Солью} (salt)\index{соль} называется (псевдо)случайная битовая строка $s$, добавляемая к аргументу $m$ (паролю) функции хэширования $h(m)$ для рандомизации хэширования одинаковых сообщений.

<<Соль>> применяется для избежания словарных атак. \textbf{Словарная} атака заключается в том, что злоумышленник один раз заранее вычисляет таблицы хэшей от наиболее \emph{вероятных} сообщений, т.~е. составляет словарь пароль-хэш, и далее производит поиск по вычисленной таблице для взламывания исходного сообщения. Ранее словарные атаки использовались для взлома паролей $m$, которые хранились в виде обычных хэшей $h(m)$. Усовершенствованной словарной атакой является метод радужных таблиц, позволяющий практически взламывать хэши длиной до 64--128 бит. Использование <<соли>> делает невозможной словарную атаку, так как значение функции вычисляется уже не от оригинального пароля, а от конкатенации <<соли>> и пароля.

<<Соль>> может храниться как отдельное значение, единственное и уникальное для системы целиком, так и быть уникальной для каждого сохранённого пароля и храниться со значением функции хэширования:
\begin{itemize}
	\item $s ~\|~ h(s ~\|~ m)$
	\item $s ~\|~ h(m ~\|~ s)$
	\item $s_1 ~\|~ h(m ~\|~ s_1 ~\|~ s_2)$
\end{itemize}

В первом случае функция хэширования вычисляется от конкатенации <<соли>> и пароля пользователя. Во втором случае в строке сначала идёт пароль, а потом -- <<соль>>. Это позволяет немного усложнить задачу злоумышленнику при переборе паролей (он не сможет сократить время вычисления значения функции хэширования за счёт одинакового префикса у всех аргументов функции хэширования). В третьем случае используется сразу две <<соли>> -- одна хранится вместе с паролем, а вторая выступает внешним параметром, хранящимся отдельно от базы данных паролей.

В рассмотренной ранее модели построения паролей в виде слогов с элементами небольшой модификации мы получили количество паролей около $2^{70}$ для 12-символьных паролей. Данный объём вычислений уже почти достижим. Следовательно, даже <<соль>> не защищает пароли от взлома, если у злоумышленника есть доступ к файлу с паролями или возможность неограниченных попыток аутентификации. Поэтому файлы с паролями дополнительно защищаются, а в системы аутентификации по паролю вводят ограничения на попытки неуспешной аутентификации.

\subsection[Unix]{Хранение паролей в Unix}

В ОС Unix пароль $m$ пользователя хранится в файле \texttt{/etc/shadow} в виде хэша (SHA, MD5 и~т.~д.) или результата шифрования (DES, Blowfish и~т.~д.), вычисленного с солью $s$ длиной от 2 (для функции crypt в оригинальной ОС UNIX) до 16 (для Blowfish в OpenBSD) ASCII-символов. То, как используется соль, зависит от используемого алгоритма. Например, в традиционном алгоритме, используемом в оригинальном UNIX, соль модифицирует S-блоки и P-блоки в протоколе DES.

Файл \texttt{/etc/shadow} доступен только привилегированным процессам, что вносит дополнительную защиту.


\subsection[Windows]{Хранение паролей и аутентификация в \protect\\ Windows}

%[MS-NLMP]: NT LAN Manager (NTLM) Authentication Protocol Specification -- 09/25/2009, Rev. 11.0
%http://blogs.technet.com/authentication/archive/2006/04/07/ntlm-s-time-has-passed.aspx
%http://technet.microsoft.com/en-us/library/cc755284(WS.10).aspx -- Windows Authentication, Updated: February 7, 2008
%http://207.46.16.252/en-us/magazine/2006.08.securitywatch.aspx - The Most Misunderstood Windows Security Setting of All Time, Jesper Johansson
%http://en.wikipedia.org/wiki/NTLM
%http://www.windowsnetworking.com/nt/atips/atips92.shtml

ОС Windows, начиная с Vista, Server 2008, Windows 7, сохраняет пароли в виде NT-хэша, который вычисляется как 128-битовый хэш MD4 от пароля в Unicode кодировке. NT-хэш не использует соль, поэтому применима словарная атака. На словарной атаке основаны программы поиска (взлома) паролей для Windows. Файл паролей называется SAM (Security Account Manager) в случае локальной аутентификации. Если пароли хранятся на сетевом сервере, то они хранятся в специальном файле, доступ к которому ограничен.

В последнем протоколе аутентификации NTLMv2\index{протокол!NTLM}\index{протокол!NTLMv2}~\cite{MS-NLMP} пользователь для входа в свой компьютер аутентифицируется либо локально на компьютере, либо удалённым сервером, если учётная запись пользователя хранится на сервере. Пользователь с именем $user$ вводит пароль в программу-\emph{клиент}, которая, взаимодействуя с программой-\emph{сервером} (локальной или удаленной на сервере домена $domain$), аутентифицирует пользователя для входа в систему.
\begin{enumerate}
    \item Клиент $\rightarrow$ Сервер: запрос аутентификации.
    \item Клиент $\leftarrow$ Сервер: 64-битовая псевдослучайная одноразовая метка $n_s$.
    \item Вводимый пользователем пароль хэшируется в $\textrm{NThash}$ без соли. Клиент генерирует 64-битовую псевдослучайную одноразовую метку $n_c$, создаёт метку времени $ts$. Далее вычисляются 128-битовые имитовставки\index{имитовставка} $\HMAC$ на хэш-функции MD5 с ключами $\textrm{NT-hash}$ и $\textrm{NTOWF}$:
        \[ \textrm{NThash} = \text{MD4}(\text{Unicode}(\text{пароль})), \]
        \[ \textrm{NTOWF} = \textrm{HMAC-MD5}_{\textrm{NThash}}(user, domain), \]
        %\[ \text{LMv2-response} = \text{HMAC-MD5}_{\text{NTLMv2-hash}}(n_c, n_s), \]
        \[ \textrm{NTLMv2-response} = \textrm{HMAC-MD5}_{\textrm{NTOWF}}(n_c, n_s, ts, domain). \]
    \item Клиент $\rightarrow$ Сервер: $(n_c, \textrm{NTLMv2-response})$. %LMv2-response,
    \item Сервер для указанных имен пользователя и домена извлекает из базы паролей требуемый NT-hash, производит аналогичные вычисления и сравнивает значения имитовставок. Если они совпадают, аутентификация успешна.
\end{enumerate}

В случае аутентификации на локальном компьютере сравниваются значения $\textrm{NTOWF}$: вычисленное от пароля пользователя и извлеченное из файла паролей SAM.

Как видно, протокол аутентификации NTLMv2 обеспечивает одностороннюю аутентификацию пользователя серверу (или своему ПК).

При удаленной аутентификации по сети последние версии Windows используют протокол Kerberos, который обеспечивает взаимную аутентификацию, и только если аутентификация по Kerberos не поддерживается клиентом или сервером, она происходит по NTLMv2.


\input{http_auth}

\chapter{Программные уязвимости}

\section[Контроль доступа в информационных системах]{Контроль доступа в \protect\\ информационных системах}
\selectlanguage{russian}

%http://www.acsac.org/2005/papers/Bell.pdf
%http://www.dranger.com/iwsec06_co.pdf
%http://csrc.nist.gov/groups/SNS/rbac/documents/design_implementation/Intro_role_based_access.htm
%http://en.wikipedia.org/wiki/Access_control#Computer_security
%http://en.wikipedia.org/wiki/Discretionary_access_control
%http://en.wikipedia.org/wiki/Mandatory_access_control
%http://en.wikipedia.org/wiki/Role-Based_Access_Control

В информационных системах контроль доступа вводится на \emph{действия} \emph{субъектов} над \emph{объектами}. В операционных системах под субъектами почти всегда понимаются процессы, под объектами -- процессы, разделяемая память, объекты файловой системы, порты ввода-вывода и т.~д., под действием -- чтение (файла или содержимого директории), запись (создание, добавление, изменение, удаление, переименование файла или директории) и исполнение (файла-программы). Система контроля доступа в информационной системе (операционной системе, базе данных и т.~д.) определяет множество субъектов, объектов и действий.

Применение контроля доступа создаётся:

\begin{enumerate}
	\item \emph{аутентификацией} субъектов и объектов,
	\item \emph{авторизацией} допустимости действия,
	\item \emph{аудитом} (проверкой и хранением) ранее совершенных действий.
\end{enumerate}

Различают три основные модели контроля доступа -- дискреционная\index{управление доступом!дискреционное} (discretionary access control, DAC), мандатная\index{управление доступом!мандатное} (mandatory access control, MAC) и ролевая\index{управление доступом!ролевое} (role-based access control, RBAC) модели. Современные операционные системы используют \emph{комбинации} двух или трёх моделей доступа, причем решения о доступе принимаются в порядке убывания приоритета: ролевая, мандатная, дискреционная модели.

Системы контроля доступа и защиты информации в операционных системах используются не только для защиты от злоумышленника, но и для повышения устойчивости системы в целом. Однако появление новых механизмов в новых версиях ОС может привести к проблемам совместимости с уже существующим программным обеспечением.

\subsection{Дискреционная модель}

Классическое определение из так называемой Оранжевой книги (Trusted Computer System Evaluation Criteria, устаревший стандарт министерства обороны США 5200.28-STD, 1985 г.~\cite{DOD-5200.28-STD}) следующее: дискреционная модель\index{контроль доступа!дискреционный} -- средства ограничения доступа к объектам, основанные на сущности (identity) субъекта и/или группы, к которой они принадлежат. Субъект, имеющий определённый доступ к объекту, имеет возможность полностью или частично передать право доступа другому субъекту.

На практике дискреционная модель доступа предполагает, что для каждого объекта в системе определен субъект-владелец. Этот субъект может самостоятельно устанавливать необходимые, по его мнению, права доступа к любому из своих объектов для остальных субъектов, в том числе и для себя самого. Логически владелец объекта является владельцем информации, находящейся в этом объекте. При доступе некоторого субъекта к какому-либо объекту система контроля доступа лишь считывает установленные для объекта права доступа и сравнивает их с правами доступа субъекта. Кроме того, предполагается наличие в ОС некоторого выделенного субъекта -- администратора дискреционного управления доступом, который имеет привилегию устанавливать дискреционные права доступа для любых, даже ему не принадлежащих, объектов в системе.

Дискреционную модель реализуют почти все популярные ОС, в частности, Windows и Unix. У каждого субъекта (процесса пользователя или системы) и объекта (файла, другого процесса и т.~д.) есть владелец, который может делегировать доступ другим субъектам, изменяя атрибуты на чтение, запись файлов для других пользователей и групп пользователей. Администратор системы имеет возможность назначить нового владельца и другие права доступа к объектам.


\subsection{Мандатная модель}

Классическое определение мандатной модели\index{контроль доступа!мандатный} из Оранжевой книги -- средства ограничения доступа к объектам, основанные на важности (секретности) информации, содержащейся в объектах, и обязательная авторизация действий субъектов для доступа к информации с присвоенным уровнем важности. Важность информации определяется уровнем доступа, приписываемым всем объектам и субъектам. Исторически мандатная модель определяла важность информации в виде иерархии, например, совершенно секретно (СС), секретно (С), конфиденциально (К) и рассекречено (Р). При этом верно следующее: СС $>$ C $>$ K $>$ P, то есть каждый уровень включает сам себя и все уровни, находящиеся ниже в иерархии.

Современное определение мандатной модели -- применение явно указанных правил доступа субъектов к объектам, определяемых только администратором системы. Сами субъекты (пользователи) не имеют возможности для изменения прав доступа. Правила доступа описаны матрицей, в которой столбцы соответствуют субъектам, строки -- объектам, а в ячейках содержатся допустимые действия субъекта над объектом. Матрица покрывает все пространство субъектов и объектов. Также определены правила наследования доступа для новых создаваемых объектов. В мандатной модели матрица может быть изменена только администратором системы.

Модель Белла --- Ла Падулы\index{модель!Белла --- Ла Падулы} (Bell --- LaPadula,~\cite{Bell:LaPadula:1973, Bell:LaPadula:1976}) использует два мандатных и одно дискреционное правила политики безопасности.
\begin{enumerate}
    \item Субъект с определённым уровнем секретности не может иметь доступ на \emph{чтение} объектов с более \emph{высоким} уровнем секретности (no read-up);
    \item Субъект с определённым уровнем секретности не может иметь доступ на \emph{запись} объектов с более \emph{низким} уровнем секретности (no write-down);
    \item Использование матрицы доступа субъектов к объектам для описания дискреционного доступа.
\end{enumerate}

\subsection{Ролевая модель}

Ролевая модель доступа основана на определении ролей в системе\index{контроль доступа!ролевой}. Понятие <<роль>> в этой модели -- это совокупность действий и обязанностей, связанных с определённым видом деятельности. Таким образом, достаточно указать тип доступа к объектам для определённой роли и определить группу субъектов, для которых она действует.
Одна и та же роль может использоваться несколькими различными субъектами (пользователями). В некоторых системах пользователю разрешается выполнять несколько ролей одновременно, в других есть ограничение на одну или несколько не противоречащих друг другу ролей в каждый момент времени.

Ролевая модель, в отличие от дискреционной и мандатной, позволяет реализовать разграничение полномочий пользователей, в частности, на системного администратора и офицера безопасности, что повышает защиту от человеческого фактора.


\section{Контроль доступа в ОС}
\selectlanguage{russian}

\subsection{Windows}
%http://www.gentlesecurity.com/blog/andr/cracking_windows_access_control.pdf
%http://msdn.microsoft.com/en-us/library/bb250462(VS.85).aspx#upm_ovwim
%http://msdn.microsoft.com/en-us/library/bb625963.aspx
%http://msdn.microsoft.com/en-us/library/bb625964.aspx

Операционные системы Windows до Windows Vista использовали только дискреционную модель безопасности. Владелец файла имел возможность изменить права доступа или разрешить доступ другому пользователю.

Начиная с Windows Vista, в дополнение к стандартной дискреционной модели субъекты и объекты стали обладать мандатным уровнем доступа, устанавливаемым администратором (или по умолчанию системой для новых созданных объектов) и имеющим приоритет над стандартным дискреционным доступом, который может менять владелец.

В Vista мандатный уровень доступа предназначен в большей степени для обеспечения \emph{целостности} и устойчивости системы, чем для обеспечения секретности.

Уровень доступа объекта (integrity level в терминологии Windows) помечается шестнадцатеричным числом в диапазоне от \texttt{0} до \texttt{0x4000}, большее число означает более высокий уровень доступа. В Vista определены 5 базовых уровней:
\begin{itemize}
    \item ненадёжный (Untrusted, \texttt{0x0000});
    \item низкий (Low Integrity, \texttt{0x1000});
    \item средний (Medium Integrity, \texttt{0x2000});
    \item высокий (High Integrity, \texttt{0x3000}) и
    \item системный (System Integrity, \texttt{0x4000}).
\end{itemize}

Дополнительно объекты имеют три атрибута, которые, если они установлены, запрещают доступ субъектов с более низким уровнем доступа к ним: cубъекты с более низким уровнем доступа не могут
\begin{itemize}
    \item читать (no read-up),
    \item изменять (no write-up),
    \item исполнять (no execute-up)
\end{itemize}
объекты с более высоким уровнем доступа. Для всех объектов по умолчанию установлен атрибут запрета записи объектов с более высоким уровнем доступа, чем имеет субъект (no write-up).

Субъекты имеют два атрибута:
\begin{itemize}
    \item запрет записи объектов с более высоким уровнем доступа, чем у субъекта (no write-up, эквивалентно аналогичному атрибуту объекта),
    \item установка уровня доступа созданного процесса-потомка как минимума от уровня доступа родительского процесса (субъекта) и исполняемого файла (объекта файловой системы).
\end{itemize}
Оба атрибута установлены по умолчанию.

Все пользовательские данные и процессы по умолчанию имеют средний уровень доступа, а системные файлы -- системный. Например, если в Internet Explorer, который в защищённом (protected) режиме запускается с низким уровнем доступа, обнаружится уязвимость, злоумышленник не будет иметь возможности изменить системные данные на диске, даже если браузер запущен администратором.

Уровень доступа процесса соответствует уровню доступа пользователя (процесса), который запустил процесс. Например, пользователи LocalSystem, LocalService, NetworkService получают системный уровень, администраторы -- высокий, обычные пользователи системы -- средний, остальные (everyone) -- низкий.

По каким-то причинам, вероятно, для целей совместимости с ранее разработанными программами и/или для упрощения разработки и настройки новых сторонних программ других производителей, субъекты с системным, высоким и средним уровнями доступа создают объекты или владеют объектами со \emph{средним} уровнем доступа. И только субъекты с низким уровнем доступа создают объекты с низким уровнем доступа. Это означает, что системный процесс может владеть файлом или создать файл со средним уровнем доступа, и другой процесс с более низким уровнем доступа, например средним, может получить доступ к файлу, в т.ч. и на запись. Это нарушает принцип запрета записи в объекты, созданные субъектами с более высоким уровнем доступа.


\subsection{Linux}

Стандартная ОС Unix обеспечивает дискреционную модель контроля доступа на следующей основе.
\begin{itemize}
    \item Каждый субъект (процесс) и объект (файл) имеют владельца пользователя и группу, которые могут изменять доступ к данному объекту для себя и других пользователей и групп.
    \item Каждый объект (файл) имеет атрибуты доступа на чтение (r), запись (w) и исполнение (x) для трёх типов пользователей: владельца-пользователя (u), владельца-группы (g), остальных пользователей (o) -- (u:rwx, g:rwx, o:rwx).
    \item Субъект может входить в несколько групп.
\end{itemize}

В 2000 г. Агентство Национальной Безопасности США (NSA) выпустило набор изменений SELinux с открытым исходным кодом к ядру ОС Linux версии 2.4. Начиная с версии ядра 2.6, SELinux входит как часть стандартного ядра. SELinux реализует комбинацию ролевой, мандатной и дискреционной моделей контроля доступа, которые могут быть изменены только администратором системы (и/или администратором безопасности). По сути, SELinux каждому субъекту приписывает одну или несколько ролей, и для каждой роли указано, к объектам с какими атрибутами они могут иметь доступ и какого вида.

Основная проблема ролевых систем контроля доступа -- очень большой список описания ролей и атрибутов объектов, что увеличивает сложность системы и приводит к регулярным ошибкам в таблицах описания контроля доступа.


\section{Виды программных уязвимостей}

\textbf{Вирусом} называется самовоспроизводящаяся часть кода (подпрограмма)\index{вирус}, которая встраивается в носители (другие программы) для своего исполнения и распространения. Вирус не может исполняться и передаваться без своего носителя.

\textbf{Червем} называется самовоспроизводящаяся отдельная (под)программа\index{червь}, которая может исполняться и распространяться самостоятельно, не используя программу-носитель.

Первой вехой в изучении компьютерных вирусов можно назвать 1949 год, когда Джон фон Нейман прочёл курс лекций в Университете Иллинойса под названием <<Теория самовоспроизводящихся машин>> (изданы в 1966~\cite{Neumann:1966}, переведены на русский язык издательством <<Мир>> в 1971 году~\cite{Neumann:1971}), в котором ввёл понятие самовоспроизводящихся механических машин. Первым сетевым вирусом считается вирус Creeper 1971 г., распространявшийся в сети ARPANET, предшественнике Интернета. Для его уничтожения был создан первый антивирус, Reaper, который находил и уничтожал Creeper.

Первый червь для Интернета, червь Морриса 1988 г., уже использовал \emph{смешанные} атаки\index{атака!смешанная} для заражения UNIX машин~\cite{EichinRochlis:1988, Spafford:1989}. Сначала программа получала доступ к удалённому запуску команд, эксплуатируя уязвимости в сервисах \texttt{sendmail}, \texttt{finger} (с использованием атаки переполнением буфера) или \texttt{rsh}. Далее с помощью механизма подбора паролей червь получал доступ к локальным аккаунтам пользователей:
\begin{itemize}
    \item получение доступа к учётным записям с очевидными паролями:
		\begin{itemize}
			\item без пароля вообще;
			\item имя аккаунта в качестве пароля;
			\item имя аккаунта в качестве пароля, повторенное дважды;
			\item использование <<ника>> (\langen{nickname});
			\item фамилия (\langen{last name, family name});
			\item фамилия, записанная задом наперёд;
		\end{itemize}
		\item перебор паролей на основе встроенного словаря из 432 слов;
		\item перебор паролей на основе системного словаря \texttt{/usr/dict/words}.
\end{itemize}

\textbf{Программной уязвимостью}\index{программная уязвимость} называется свойство программы, позволяющее нарушить ее работу. Программные уязвимости могут приводить к отказу в обслуживании (Denial of Service, DoS-атака)\index{атака!отказ в обслуживании}, утечке и изменению данных, появлению и распространению вирусов и червей.

Одной из распространенных атак для заражения персональных компьютеров является переполнение буфера в стеке. В интернет-сервисах наиболее распространенной программной уязвимостью в настоящее время является межсайтовый скриптинг (Cross-Site Scripting, XSS-атака)\index{атака!XSS}.

Наиболее распространенные программные уязвимости можно разделить на классы:
\begin{enumerate}
    \item Переполнение буфера -- копирование в буфер данных большего размера, чем длина выделенного буфера. Буфером может быть контейнер текстовой строки, массив, динамически выделяемая память и т.~д. Переполнение становится возможным вследствие либо отсутствия контроля над длиной копируемых данных, либо из-за ошибок в коде. Типичная ошибка -- разница в 1 байт между размерами буфера и данных при сравнении.
    \item Некорректная обработка (парсинг) данных, введенных пользователем, является причиной большинства программных уязвимостей в веб-приложениях. Под обработкой понимаются:
        \begin{enumerate}
            \item проверка на допустимые значения и тип (числовые поля не должны содержать строки и т.~д.);
            \item фильтрация и экранирование специальных символов, имеющих значения в скриптовых языках или для декодирования из одной текстовой кодировки в другую. Примеры символов: \texttt{\textbackslash},  \texttt{\%}, \texttt{<}, \texttt{>}, \texttt{"},  \texttt{'};
            \item фильтрация ключевых слов языков разметки и скриптов. Примеры: \texttt{script}, \texttt{JavaScript};
            \item декодирование различными кодировками при парсинге. Распространенный способ обхода системы контроля парсинга данных состоит в однократном или множественном последовательном кодировании текстовых данных в шестнадцатеричные кодировки \texttt{\%NN} ASCII и UTF-8. Например, браузер или веб-приложения производят $n$ -- кратные последовательные декодирования, в то время как система контроля делает $k$-кратное декодирование, $0 \leq k < n$, и, следовательно, пропускает закодированные запрещенные символы и слова.
        \end{enumerate}
    \item Некорректное использование синтаксиса функций. Например, \texttt{printf(s)} может привести к уязвимости записи в указанный адрес памяти. Если злоумышленник вместо обычной текстовой строки введёт в качестве \texttt{s = "текст некоторой длины\%n"}, то функция \texttt{printf()}, ожидающая первым аргументом строку формата \texttt{printf(fmt, \dots)}, обнаружив \texttt{\%n}, возьмет значения из ячеек памяти, следующих перед текстовой строкой (устройство стека функции описано далее), и запишет в адрес памяти, равный считанному значению, количество выведенных символов на печать функцией \texttt{printf()}.
\end{enumerate}


\section{Переполнение буфера в стеке}
\selectlanguage{russian}

В качестве примера переполнения буфера опишем самую распространенную атаку, направленную на исполнение кода злоумышленника.

В 64-битовой x86\_64 архитектуре основное пространство виртуальной памяти процесса из 16 эксабайтов ($2^{64}$ байт) свободно, и только малая часть занята (выделена). Виртуальная память выделяется процессу операционной системой блоками по 4 Кб, называемыми страницами памяти. Выделенные страницы соответствуют страницам физической оперативной памяти или страницам файлов.

Пример выделенной виртуальной памяти процесса представлен в табл.~\ref{tab:virtual-memory}. Локальные переменные функций хранятся в области памяти, называемой стеком.
\begin{table}[!ht]
    \centering
    \caption{Пример структуры виртуальной памяти процесса\label{tab:virtual-memory}}
    \resizebox{\textwidth}{!}{ \begin{tabular}{r|c|}
        \multicolumn{2}{c}{Адрес ~~~~~~~~~~~~~~ Использование} \\
        \cline{2-2}
        \texttt{0x00000000 00000000} & \\
        & \\
        \cdashline{2-2}
        \texttt{0x00000000 0040063F} & \multirow{2}{*}{\parbox{6cm}{Исполняемый код, динамические библиотеки}} \\
        & \\
        \cdashline{2-2}
        & \\
        & \\
        & \\
        \cdashline{2-2}
        \texttt{0x00000000 0143E010} & \multirow{2}{*}{Динамическая память} \\
        & \\
        \cdashline{2-2}
        & \\
        & \\
        & \\
        \cdashline{2-2}
        \texttt{0x00007FFF A425DF26} & \multirow{2}{*}{Переменные среды} \\
        & \\
        \cdashline{2-2}
        & \\
        & \\
        & \\
        \cdashline{2-2}
        \texttt{0x00007FFF FFFFEB60} & \multirow{2}{*}{Стек функций} \\
        & \\
        \cdashline{2-2}
        & \\
        & \\
        \texttt{0xFFFFFFFF FFFFFFFF} & \\
        \cline{2-2}
    \end{tabular} }
\end{table}

Приведём пример переполнения буфера в стеке\index{стек}, которое даёт возможность исполнить код, для 64-разрядной ОС Linux. Ниже приводится листинг исходной программы, которая печатает расстояние Хэмминга между векторами $b1 = \text{\texttt{0x01234567}}$ и $b2 = \text{\texttt{0x89ABCDEF}}$:

\begin{verbatim}
#include <stdio.h>
#include <string.h>

int hamming_distance(unsigned a1, unsigned a2, char *text,
                     size_t textsize) {
  char buf[32];
  unsigned distance = 0;
  unsigned diff = a1 ^ a2;
  while (diff) {
    if (diff & 1) distance++;
    diff >>= 1;
  }
  memcpy(buf, text, textsize);
  printf("%s: %i\n", buf, distance);
  return distance;
}

int main() {
  char text[68] = "Hamming";
  unsigned b1 = 0x01234567;
  unsigned b2 = 0x89ABCDEF;
  return hamming_distance(b1, b2, text, 8);
}
\end{verbatim}

Вывод программы при запуске:
\begin{verbatim}
$ ./hamming
Hamming: 8
\end{verbatim}

При вызове вложенных функций вызывающая функция выделяет стековый кадр для вызываемой функции в сторону уменьшения адресов. Стековый кадр в порядке уменьшения адресов состоит из следующих частей.
\begin{enumerate}
    \item Аргументы вызова функции, расположенные в порядке уменьшения адреса (за исключением тех, которые передаются в регистрах процессора).
    \item Сохраненный регистр процессора \texttt{rip} внешней функции, также называемый адресом возврата. Регистр процессора \texttt{rip} содержит адрес следующей инструкции для исполнения. При входе во вложенную функцию адрес инструкции текущей функции запоминается в стеке, в регистре записывается новое значение адреса первой инструкции из вложенной функции, а по завершении функции регистр восстанавливается из стека, и, таким образом, исполнение возвращается назад.
    \item Сохраненный регистр процессора \texttt{rbp} внешней функции. Регистр процессора \texttt{rbp} содержит адрес сохраненного регистра \texttt{rbp} в стековом кадре вызывающей функции. Процессор обращается к локальным переменным функций по смещению относительно регистра \texttt{rbp}. При вызове вложенной функции регистр сохраняется в стеке, в регистр записывается текущее значение адреса стека (\texttt{rsp}), а по завершении функции регистр восстанавливается.
    \item Локальные переменные, как правило расположенные в порядке уменьшения адреса при объявлении новой переменной (порядок может быть изменён в результате оптимизаций и использования механизмов защиты, таких как Stack Smashing Protection в компиляторе GCC).
\end{enumerate}

Адрес начала стека, а также, возможно, адреса локальных массивов и переменных выровнены на границу параграфа в 16 байт, из-за чего в стеке могут образоваться неиспользуемые байты.

Если в программе есть ошибка, которая может привести к переполнению выделенного буфера в стеке при копировании, есть возможность записать вместо сохраненного значения регистра \texttt{rip} новое. В результате по завершении данной функции исполнение начнется с указанного адреса. Если есть возможность записать в переполняемый буфер исполняемый код, а затем на место сохраненного регистра \texttt{rip} адрес на этот код, то получим исполнение заданного кода в стеке функции.

На рис.~\ref{fig:stack-overflow} приведены исходный стек и стек с переполненным буфером, из-за которого записалось новое сохраненное значение \texttt{rip}.

\begin{figure}[!ht]
	\centering
	\includegraphics[width=0.95\textwidth]{pic/stack-overflow}
	\caption{Исходный стек и стек с переполнением буфера\label{fig:stack-overflow}}
\end{figure}


Изменим программу для демонстрации, поместив в копируемую строку исполняемый код для вызова \texttt{/bin/sh}.
{ \small
\begin{verbatim}
...
int main() {
  char text[68] =
    // 28 байт исполняемого кода
    "\x90" "\x90" "\x90"                // nop; nop; nop
    "\x48\x31" "\xD2"                   // xor %rdx, %rdx
    "\x48\x31" "\xF6"                   // xor %rsi, %rsi
    "\x48\xBF" "\xDC\xEA\xFF\xFF"
    "\xFF\x7F\x00\x00"                  // mov $0x7fffffffeadc,
                                        //   %rdi
    "\x48\xC7\xC0" "\x3B\x00\x00\x00"   // mov $0x3b, %rax
    "\x0F\x05"                          // syscall
    // 8 байт строки /bin/sh
    "\x2F\x62\x69\x6E\x2F\x73\x68\x00"  // "/bin/sh\0"
    // 12 байт заполнения и 16 байт новых
    // значений сохраненных регистров
    "\x00\x00\x00\x00"                  // незанятые байты
    "\x00\x00\x00\x00"                  // unsigned distance
    "\x00\x00\x00\x00"                  // unsigned diff
    "\x50\xEB\xFF\xFF"                  // регистр
    "\xFF\x7F\x00\x00"                  //   rbp=0x7fffffffeb50
    "\xC0\xEA\xFF\xFF"                  // регистр
    "\xFF\x7F\x00\x00"                  //   rip=0x7fffffffeac0
    ;
  ...
  return hamming_distance(b1, b2, text, 68);
  ...
}
\end{verbatim} }

Код эквивалентен вызову функции \texttt{execve(``/bin/sh'', 0 0)} через системный вызов функции ядра Linux для запуска оболочки среды \texttt{/bin/sh}. При системном вызове нужно записать в регистр \texttt{rax} номер системной функции, а в другие регистры процессора -- аргументы. Данный системный вызов с номером \texttt{0x3b} требует в качестве аргументов регистры \texttt{rdi} с адресом строки исполняемой программы, \texttt{rsi} и \texttt{rdx} с адресами строк параметров запускаемой программы и переменных среды. В примере в \texttt{rdi} записывается адрес \texttt{0x7fffffffeadc}, который указывает на строку \texttt{``/bin/sh''} в стеке после копирования. Регистры \texttt{rdx} и \texttt{rsi} обнуляются.

На рис.~\ref{fig:stack-overflow} приведён стек с переполненным буфером, в результате которого записалось новое сохраненное значение \texttt{rip}, указывающее на заданный код в стеке.

Начальные инструкции \texttt{nop} с кодом \texttt{0x90} означают пустые операции. Часто точные значения адреса и структуры стека неизвестны, поэтому злоумышленник угадывает предполагаемый адрес стека. В начале исполняемого кода создаётся массив из операций \texttt{nop} с надеждой, что предполагаемое значение стека, то есть требуемый адрес rip, попадет на эти операции, повысив шансы угадывания. Стандартная атака на переполнение буфера с исполнением кода также подразумевает последовательный перебор предполагаемых адресов для нахождения правильного адреса для \texttt{rip}.

В результате переполнения буфера в примере по завершении функции \texttt{hamming\_distance()} начнет исполняться инструкция с адреса строки \texttt{buf}, то есть заданный код.


\subsection{Защита}

Самый лучший способ защиты от атак переполнения буфера -- создание программного кода со слежением за размером данных и длиной буфера. Однако ошибки все равно происходят. Существует несколько стандартных способов защиты от исполнения кода в стеке в архитектуре x86 (x86-64).

\begin{enumerate}
	\item Современные 64-разрядные x86-64 процессоры включают поддержку флаги доступа к страницам памяти. В таблице виртуальной памяти, выделенной процессу, каждая страница имеет набор флагов, отвечающих за защиту страниц от некорректных действий программы.
	\begin{itemize}
		\item Флаг разрешения доступа из пользовательского режима. Если флаг не установлен, то доступ к данной области памяти возможен только из режима ядра.
		\item Флаг запрета записи. Если флаг установлен, то попытка выполнить запись в данную область памяти приведёт к возникновению исключения.
		\item Флаг запрета исполнения\index{бит запрета исполнения} (NX-Bit, No eXecute Bit в терминологии AMD; XD-Bit, Execute Disable Bit в терминологии Intel; DEP, Data Executuion Prevention -- соответствующая опция защиты в операционных системах). Если флаг установлен, при попытке передачи управления на данную область памяти возникнет исключение. Для совместимости со старым программным обеспечением есть возможность отключить использование данного флага на уровне операционной системы целиком или для отдельных программ.
	\end{itemize}
	Попытка выполнить операции, которые запрещены соответствующими настройками виртуальной памяти, вызывает ошибку сегментации (segmentation fault, segfault).

    \item Второй стандартный способ -- вставка проверочных символов (называемых canaries, guards) после массивов и в конце стека и их проверка перед выходом из функции. Если произошло переполнение буфера, программа аварийно завершится. Данный способ защиты реализован с помощью модификации конечного кода программы во время компиляции\footnote{см. опции \texttt{-fstack-protector} для GCC, \texttt{/GS} для компиляторов от Microsoft и другие}, его нельзя включить или отключить без перекомпиляции программного обеспечения.

    \item Третий способ -- рандомизация адресного пространства (address space layout randomization, ASLR), то есть случайное расположение стека, кода и т.~д. В настоящее время используется в большинстве современных операционных систем (Android, iOS, Linux, OpenBSD, OS X, Windows). Это приводит к маловероятному угадыванию адресов и значительно усложняет использование уязвимости.
\end{enumerate}


\subsection{Другие атаки с переполнением буфера}

Почти любую возможность для переполнения буфера в стеке или динамической памяти можно использовать для получения критической ошибки в программе из-за обращения к адресам виртуальной памяти, страницы которых не были выделены процессу. Следовательно, можно проводить атаки отказа в обслуживании (Denial of Service (DoS) атаки).

Переполнение буфера в динамической памяти в случае хранения в ней адресов для вызова функций может привести к подмене адресов и исполнению другого кода.

В описанных DoS-атаках NX-бит не защищает систему.


\input{xss}

\input{sql-injections}

%\chapter{Послесловие}
%Это должно быть что-то в виде заключения, объяснения, почему именно эти темы выбраны, насколько актуален материал с теоретической и практической точки зрения.


\appendix

\chapter{Математическое приложение}\label{chap:discrete-math}

\section{Общие определения}

Выражением $\mod n$ обозначается вычисление остатка от деления произвольного целого числа на целое число $n$. В полиномиальной арифметике эта операция означает вычисление остатка от деления многочленов.
%далее будем обозначать целые числа или операции с целыми числами, взятыми \textbf{по модулю}\index{модуль} целого числа $n$ (остаток от целочисленного деления). Примеры выражений:
    \[ a\mod n, \]
    \[ (a + b) c\mod n. \]
Равенство
    \[ a = b \mod n \]
означает, что выражения $a$ и $b$ равны (говорят также <<сравнимы>>, <<эквивалентны>>) по модулю $n$.

Множество
    \[ \{ 0, 1, 2, 3,  \dots,  n-1 \mod n\} \]
состоит из $n$ элементов, где каждый элемент $i$ представляет все целые числа, сравнимые с $i$ по модулю $n$.

Наибольший общий делитель (НОД) двух чисел $a,b$ обозначается $\gcd(a,b)$ (greatest common divisor).

Два числа $a,b$ называются взаимно простыми, если они не имеют общих делителей, кроме 1, т.е. $\gcd(a,b) = 1$.

Выражение $a \mid b$ означает, что $a$ делит $b$.

\input{birthdays_paradox}

\section{Группы}\label{section-groups}
\selectlanguage{russian}

\subsection{Свойства групп}

\textbf{Группой}\index{группа} называется множество $\Gr$, на котором задана бинарная операция <<$\cdot$>>, удовлетворяющая следующим аксиомам:
\begin{enumerate}
    \item замкнутость
        \[ \forall a,b \in \Gr: a \cdot b = c \in \Gr; \]
    \item ассоциативность
        \[ \forall a,b,c \in \Gr: (a \cdot b) \cdot c = a \cdot (b \cdot c); \]
    \item существование единичного элемента
        \[ \exists ~ e \in \Gr: e\cdot a = a \cdot e = a; \]
    \item существование обратного элемента
        \[ \forall a \in \Gr ~ \exists ~ b \in \Gr: a \cdot b = b \cdot a = e. \]
\end{enumerate}
Если
    \[ \forall a,b \in \Gr: a \cdot b = b \cdot a, \]
то группа коммутативная.

Если операция в группе задана как умножение $\cdot$, то группа называется \textbf{мультипликативной}, $e = 1$, обратный элемент -- $a^{-1}$, возведение в степень $k$ -- $a^k$.

Если операция задана как сложение $+$, то группа называется \textbf{аддитивной}, $e = 0$, обратный элемент $-a$, сложение $k$ раз -- $ka$.

Подмножество группы, удовлетворяющее аксиомам группы, называется \textbf{подгруппой}\index{подгруппа}.

\textbf{Порядком} $|\Gr|$ \textbf{группы}\index{порядок группы} $\Gr$ называется число элементов в группе. Пусть группа мультипликативная. Для любого элемента $a \in \Gr$ выполняется $a^{|\Gr|} = 1$.

\textbf{Порядком элемента} $a$ называется минимальное натуральное число
    \[ ord(a): a^{ord(a)} = 1. \]
 Порядок элемента делит порядок группы:
    \[ ord(a) \mid \left|\Gr\right|. \]


\subsection{Циклические группы}

\textbf{Генератором} $g \in \Gr$ называется элемент, \emph{порождающий} всю группу\index{генератор группы}
    \[ \Gr = \{g, g^2, g^3,  \ldots,  g^{|\Gr|} = 1\}. \]
Группа, в которой существует генератор, называется \textbf{циклической}\index{группа!циклическая}.

Если конечная группа не циклическая, то в ней существуют циклические подгруппы, порожденные всеми элементами. Любой элемент $a$ группы порождает либо циклическую \emph{подгруппу}
    \[ \{ a, a^2, a^3,  \dots,  a^{ord(a)} = 1 \} \]
порядка $ord(a)$, если порядок элемента $ord(a) < |\Gr|$, либо \emph{всю} группу
    \[ \Gr = \{ a, a^2, a^3,  \dots,  a^{|\Gr|} = 1 \}, \]
если порядок элемента равен порядку группы $ord(a) = |\Gr|$. Порядок любой подгруппы, как и порядок элемента, делит порядок всей группы.

Представим циклическую группу через генератор $g$ как
    \[ \Gr = \{g, g^2,  \ldots,  g^{|\Gr|} = 1\} \]
и каждый элемент $g^i$  возведём в степени $1, 2,  \ldots,  |\Gr|$. Тогда
\begin{itemize}
    \item элементы $g^i$, для которых число $i$ взаимно просто с $|\Gr|$, породят снова всю группу
            \[ \Gr = \{ g^i, g^{2i}, g^{3i},  \dots,  g^{|\Gr| i} = 1 \}, \]
        так как степени $\{i, 2i, 3i, \dots, |\Gr| i \}$ по модулю $|\Gr|$ образуют перестановку чисел $\{1, 2, 3, \dots, |\Gr|\}$; следовательно, $g^i$ -- тоже генератор, число таких чисел $i$ по определению функции Эйлера $\varphi(|\Gr|)$ ($\varphi(n)$ -- количество взаимно простых с $n$ целых чисел в диапазоне $[1,n-1]$);
    \item элементы $g^i$, для которых $i$ имеют общие делители
            \[ d_i = \gcd(i, |\Gr|) \neq 1 \]
        c $|\Gr|$, породят подгруппы
            \[ \{ g^i, g^{2i}, g^{3i},  \dots,  g^{\frac{i}{d_i} |\Gr|} = 1\}, \]
        так как степень последнего элемента кратна $|\Gr|$; следовательно, такие $g^i$ образуют циклические подгруппы порядка $d_i$.
\end{itemize}
%TODO Гашков, Болотов, Часовских "Эллиптическая криптография" или "Методы элл. кри-ии"

Из предыдущего утверждения следует, что число генераторов в циклической группе равно
    \[ \varphi(|\Gr|). \]

Для проверки, является ли элемент генератором всей группы, требуется знать разложение порядка группы $|\Gr|$ на множители. Далее элемент возводится в степени, равные всем делителям порядка группы, и сравнивается с единичным элементом $e$. Если ни одна из степеней не равна $e$, то этот элемент является примитивным элементом или генератором группы. В противном случае элемент будет генератором какой-либо подгруппы.

Задача разложения числа на множители является трудной для вычисления. На сложности ее решения, например, основана криптосистема RSA\index{криптосистема!RSA}. Поэтому при создании больших групп желательно заранее знать разложение порядка группы на множители для возможности выбора генератора.


\subsection{Группа $\Z_p^*$}\label{section-group-multiplicative}

\textbf{Группой $\Z_p^*$} называется группа\index{группа!$\Z_p^*$}
    \[ \Z_p^* = \{1, 2,  \dots,  p-1 \mod p\}, \]
где $p$ -- простое\index{число!простое} число, операция в группе -- умножение $\ast$ по $\mod p$.

Группа $\Z_p^*$ -- \textbf{циклическая}, порядок
    \[ |\Z_p^*| = \varphi(p) = p - 1. \]
Число генераторов в группе --
    \[ \varphi(|\Z_p^*|) = \varphi(p-1). \]

Из того, что $\Z_p^*$ -- группа, для простого\index{число!простое} $p$ и любого $a \in [2, p-1] \mod p$ следует \textbf{малая теорема Ферма}\index{теорема!Ферма малая}:
    \[ a^{p-1} = 1 \mod p. \]
На малой теореме Ферма основаны многие тесты проверки числа на простоту.

\example 1
Рассмотрим группу $\Z_{19}^*$. Порядок группы -- 18. Делители: 2, 3, 6, 9. Является ли 12 генератором?
\[ \begin{array}{l}
    12^2 = -8 \mod 19, \\
    12^3 = -1 \mod 19, \\
    12^6 = 1 \mod 19, \\
\end{array} \]
12 -- генератор подгруппы 6 порядка. Является ли 13 генератором?
\[ \begin{array}{l}
    13^2 = -2 \mod 19, \\
    13^3 = -7 \mod 19, \\
    13^6 = -8 \mod 19, \\
    13^9 = -1 \mod 19, \\
    13^{18} = 1 \mod 19, \\
\end{array} \]
13 -- генератор всей группы.
\exampleend

\example 2
В таб.~\ref{tab:Zp-sample} приведён пример группы $\Z_{13}^*$. Число генераторов -- $\varphi(12) = 4$. Подгруппы --
    \[ \Gr^{(1)}, \Gr^{(2)}, \Gr^{(3)}, \Gr^{(4)}, \Gr^{(6)}, \]
верхний индекс обозначает порядок подгруппы.

\begin{table}[!ht]
    \centering
    \caption {Генераторы и циклические подгруппы группы $\Gr=\Z_{13}^*$\label{tab:Zp-sample}}
    \resizebox{\textwidth}{!}{ \begin{tabular}{|c|p{0.66\textwidth}|c|}
        \hline
        Элемент & Порождаемая группа или подгруппа & Порядок \\
        \hline
        1 & $\Gr^{(1)} = \{ 1 \}$ & 1 \\
        2 & $\Gr = \{ 2, 4,  8 = -5, -10 = 3, 6, 12 = -1, -2, -4, -8 = 5, 10 = -3, -6, -12 = 1 \}$ & 12, ген. \\
        3 & $\Gr^{(3)} = \{ 3, 9 = -4, -12 = 1 \}$ & 3 \\
        4 & $\Gr^{(6)} = \{ 4, 16 = 3, 12 = -1, -4, -3, -12 = 1 \}$ & 6 \\
        5 & $\Gr^{(4)} = \{ 5, 25 = -1, -5, 1 \}$ & 4 \\
        6 & $\Gr = \{6, 36 = -3, -5, -4, 2, -1, -6, 3, 5, 4, -2, -12 = 1 \}$ & 12, ген. \\
        7 = -6 & $\Gr = \{ -6, 36 = -3, 5, -4, -2, -1, 6, 3, -5, 4, 2, -12 = 1 \}$ & 12, ген. \\
        8 = -5 & $\Gr^{(4)} = \{ -5, 25 = -1, 5, 1 \}$ & 4 \\
        9 = -4 & $\Gr^{(3)} = \{ -4, 16 = 3, -12 = 1 \}$ & 3 \\
        10 = -3 & $\Gr^{(6)} = \{ -3, 9 = -4, 12 = -1, 3, -9 = 4, -12 = 1 \}$ & 6 \\
        11 = -2 & $\Gr = \{ -2, 4, 5, 3, -6, -1, 2, -4, -5, -3, 6, -12 = 1 \}$ & 12, ген. \\
        12 = -1 & $\Gr^{(2)} = \{ -1, 1 \}$ & 2 \\
        \hline
    \end{tabular} }
\end{table}
\exampleend


\subsection{Группа $\Z_n^*$}

\textbf{Функция Эйлера}\index{функция!Эйлера} $\varphi(n)$ определяется как количество чисел, взаимно простых с $n$ , в интервале от 1 до $n-1$.

Если $n=p$ -- простое\index{число!простое} число, то
    \[ \varphi(p) = p - 1, \]
    \[ \varphi(p^k) = p^k - p^{k-1} = p^{k-1}(p - 1). \]
Если $n$ -- составное число и
    \[ n = \prod \limits_{i} p_i^{k_i} \]
разложено на простые множители $p_i$, то
    \[ \varphi(n) = \prod \limits_{i} \varphi(p_i^{k_i}) =  \prod \limits_{i} p_i^{k_i - 1}(p_i - 1). \]

\textbf{Группой $\Z_n^*$} называется группа\index{группа!$\Z_n^*$}
    \[ \Z_n^* = \left\{ \forall a \in \left\{ 1, 2,  \dots,  n-1 \mod n \right\} : \gcd(a,n) = 1 \right\}, \]
с операцией умножения $\ast$ по $\mod n$.

Порядок группы --
    \[ |\Z_n^*| = \varphi(n). \]
Группа $\Z_p^*$ -- частный случай группы $\Z_n^*$.

Если $n$ \emph{составное}\index{число!составное} (не простое) число, то группа $\Z_n^*$ \textbf{нециклическая}.

Из того, что $\Z_n^*$ -- группа, для любых $a \neq 0, n > 1: \gcd(a,n) = 1$ следует \textbf{теорема Эйлера}\index{теорема!Эйлера}:
    \[ a^{\varphi(n)} = 1 \mod n. \]

При возведении в степень, если $\gcd(a,n) = 1$, выполняется
    \[ a^b = a^{b \mod \varphi(n)} \mod n. \]

\example
В табл.~\ref{tab:Zn-sample} приведена нециклическая группа $\Z_{21}^*$ и ее циклические подгруппы
    \[ \Gr^{(1)}, \Gr_1^{(2)}, \Gr_2^{(2)}, \Gr_3^{(2)}, \Gr_1^{(3)}, \Gr_1^{(6)}, \Gr_2^{(6)}, \Gr_3^{(6)}, \]
верхний индекс обозначает порядок подгруппы, нижний индекс нумерует различные подгруппы одного порядка.

\begin{table}[!ht]
    \centering
    \caption{Циклические подгруппы нециклической группы $\Z_{21}^*$\label{tab:Zn-sample}}
    \begin{tabular}{|c|l|c|}
        \hline
        Элемент & Порождаемая циклическая подгруппа & Порядок \\
        \hline
        1  & $\Gr^{(1)} = \{ 1 \}$ & 1 \\
        2  & $\Gr_1^{(6)} = \{ 2, 4, 8, 16, 11, 1 \}$ & 6 \\
        4  & $\Gr_1^{(3)} = \{ 4, 16, 1 \}$ & 3 \\
        5  & $\Gr_2^{(6)} = \{ 5, 4, 20, 16, 17, 1 \}$ & 6 \\
        8  & $\Gr_1^{(2)} = \{ 8, 1 \}$ & 2 \\
        10 & $\Gr_3^{(6)} = \{ 10, 16, 13, 4, 19, 1 \}$ & 6 \\
        11 & $\Gr_1^{(6)} = \{ 11, 16, 8, 4, 2, 1 \}$ & 6 \\
        13 & $\Gr_2^{(2)} = \{ 13, 1 \}$ & 2 \\
        16 & $\Gr_1^{(3)} = \{ 16, 4, 1 \}$ & 3 \\
        17 & $\Gr_2^{(6)} = \{ 17, 16, 20, 4, 5, 1 \}$ & 6 \\
        19 & $\Gr_3^{(6)} = \{ 19, 4, 13, 16, 10, 1 \}$ & 6 \\
        20 & $\Gr_3^{(2)} = \{ 20, 1 \}$ & 2 \\
        \hline
    \end{tabular}
\end{table}
\exampleend

\subsection{Конечные поля}

\textbf{Полем} называется множество $\F$, для которого\index{поле}:
\begin{itemize}
    \item заданы две бинарные операции, условно называемые операциями умножения <<$\cdot$>> и сложения <<$+$>>;
    \item выполняются аксиомы группы для операции <<сложения>>: \\
        1. замкнутость:
		\[\forall a, b \in \F: a + b \in \F;\]
        2. ассоциативность:
		\[\forall a, b, c \in \F: (a+b)+c = a+(b+c);\]
        3. существование нейтрального элемента по сложению (часто обозначаемого как <<0>>):
		\[\exists 0 \in \F: \forall a \in \F: a + 0 = 0 + a = a; \]
        4. существование обратного элемента:
		\[\forall a \in \F: \exists -a: a + (-a) = 0; \]
    \item выполняются аксиомы группы для операции <<умножения>>, за одним исключением: \\
        1. замкнутость:
		\[\forall a, b \in \F: a \cdot b \in \F; \]
        2. ассоциативность:
		\[\forall a, b, c \in \F: (a \cdot b) \cdot c = a \cdot (b \cdot c);\]
        3. существование нейтрального элемента по умножению (часто обозначаемого как <<1>>):
		\[\exists 1 \in \F: \forall a \in \F: a \cdot 1 = 1 \cdot a = a;\]
        4. существование обратного элемента по умножению для всех элементов множества, кроме нейтрального элемента по сложению:
		\[\forall a \in {\F \backslash 0}: \exists a^{-1}: a \cdot a^{-1} = a^{-1} \cdot a = 1;\]
    \item операции <<сложения>> и <<умножения>> коммутативны
        \[ \begin{array}{l}
            \forall a, b \in \F: a + b = b + a, \\
            \forall a, b \in \F: a \cdot b = b \cdot a; \\
        \end{array} \]
    \item выполняется свойство дистрибутивности
        \[ \forall a, b, c \in \F: a \cdot (b + c) = (a \cdot b) + (a \cdot c). \]
\end{itemize}

Примеры \emph{бесконечных} полей (с бесконечным числом элементов) -- поле рациональных чисел $\group{Q}$, поле вещественных чисел $\group{R}$, поле комплексных чисел $\group{C}$ с обычными операциями сложения и умножения.

В криптографии рассматриваются \emph{конечные} поля (с конечным числом элементов), называемые также \textbf{полями Галуа}.

Число элементов в любом конечном поле равно $p^n$, где $p$ -- простое\index{число!простое} число и $n$ -- натуральное число. Обозначения поля Галуа: $\GF{p}, \GF{p^n}, \F_p, \F_{p^n}$ (аббревиатура от Galois field). Все поля Галуа $\GF{p^n}$ изоморфны друг другу (существует взаимно однозначное отображение между полями, сохраняющее действие всех операций). Другими словами, существует только одно поле Галуа $\GF{p^n}$ для фиксированных $p, n$.

Приведём примеры конечных полей.

Двоичное поле $\GF{2}$ состоит из двух элементов. Однако задать его можно разными способами:
\begin{itemize}
	\item Как множество из двух чисел <<0>> и <<1>> с определёнными на нём операциями <<сложение>> и <<умножение>> как сложение и умножение чисел по модулю 2. Нейтральным элементом по сложению будет <<0>>, по умножению -- <<1>>:
\[\begin{array}{ll}
	0 + 0 = 0,	& 	0 \cdot 0 = 0, \\
	0 + 1 = 1,	& 	0 \cdot 1 = 0, \\
	1 + 0 = 1,	& 	1 \cdot 0 = 0, \\
	1 + 1 = 0,	& 	1 \cdot 1 = 1. \\
\end{array}\]
	\item Как множество из двух логических объектов <<ЛОЖЬ>> ($F$) и <<ИСТИНА>> ($T$) с определёнными на нём операциями <<сложение>> и <<умножение>> как булевые операции <<исключающее или>> и <<и>> соответственно. Нейтральным элементом по сложению будет <<ЛОЖЬ>>, по умножению -- <<ИСТИНА>>:
\[\begin{array}{ll}
	F + F = F,	& 	F \cdot F = F, \\
	F + T = T,	& 	F \cdot T = F, \\
	T + F = T,	& 	T \cdot F = F, \\
	T + T = F,	& 	T \cdot T = T. \\
\end{array}\]
	\item Как множество из двух логических объектов <<ЛОЖЬ>> ($F$) и <<ИСТИНА>> ($T$) с определёнными на нём операциями <<сложение>> и <<умножение>> как булевые операции <<эквивалентность>> и <<или>> соответственно. Нейтральным элементом по сложению будет <<ИСТИНА>>, по умножению -- <<ЛОЖЬ>>:
\[\begin{array}{ll}
	F + F = T,	& 	F \cdot F = F, \\
	F + T = F,	& 	F \cdot T = T, \\
	T + F = F,	& 	T \cdot F = T, \\
	T + T = T,	& 	T \cdot T = T. \\
\end{array}\]
	\item Как множество из двух чисел <<0>> и <<1>> с определёнными на нём операциями <<сложение>> и <<умножение>>, заданными в табличном представлении. Нейтральным элементом по сложению будет <<1>>, по умножению -- <<0>>:
\[\begin{array}{ll}
	0 + 0 = 1,	& 	0 \cdot 0 = 0, \\
	0 + 1 = 0,	& 	0 \cdot 1 = 1, \\
	1 + 0 = 0,	& 	1 \cdot 0 = 1, \\
	1 + 1 = 1,	& 	1 \cdot 1 = 1. \\
\end{array}\]
\end{itemize}

Все перечисленные выше варианты множеств изоморфны друг другу. Поэтому в дальнейшем под конечным полем $\GF{p}$, где $p$ -- простое\index{число!простое} число, будем понимать поле, заданное как множество целых чисел от $0$ до $p-1$ включительно, на котором операции <<сложение>> и <<умножение>> заданы как операции сложения и умножения чисел по модулю числа $p$. Например, поле $\GF{7}$ будем считать состоящим из 7-и чисел $\{0, 1, 2, 3, 4, 5, 6\}$ с операциями умножения $(\cdot \mod 7)$ и сложения $(+ \mod 7)$ по модулю.

Конечное поле $\GF{p^n}, n > 1$ строится \textbf{расширением} \emph{базового} поля $\GF{p}$. Элемент поля представляется как многочлен степени $n-1$ (или меньше) с коэффициентами из базового поля $\GF{p}$:
    \[ \alpha = \sum\limits_{i=0}^{n-1} a_i x^i, ~ a_i \in \GF{p}. \]

Операция сложения элементов в таком поле традиционно задаётся как операция сложения коэффициентов при одинаковых степенях в базовом поле $\GF{p}$. Операция умножения -- как умножение многочленов со сложением и умножением коэффициентов в базовом поле $\GF{p}$ и дальнейшим приведением результата по модулю некоторого заданного (для поля) неприводимого\footnote{Многочлен называется \textbf{неприводимым}\index{многочлен!неприводимый}, если он не раскладывается на множители, и \textbf{приводимым}\index{многочлен!приводимый}, если раскладывается.} многочлена $m(x)$. Количество элементов в поле равно $p^n$.

Многочлен $g(x)$ называется \textbf{примитивным элементом}\index{многочлен!примитивный} (генератором) поля, если его степени порождают все ненулевые элементы, т.~е. $\GF{p^n} \setminus \{0\}$, заданное неприводимым многочленом $m(x)$, за исключением нуля:
    \[ \GF{p^n} \setminus \{0\} = \{ g(x), g^2(x), g^3(x), \dots, g^{p^n-1}(x) = 1 \mod m(x) \}. \]

\example
В табл.~\ref{tab:irreducible-gf2} приведены примеры многочленов \emph{над полем} $\GF{2}$.
\begin{table}[!ht]
    \centering
    \caption{Пример многочленов над полем $\GF{2}$\label{tab:irreducible-gf2}}
    \begin{tabular}{|c|c|c|}
        \hline
        Многочлен & \parbox{2.5cm}{Упрощенная запись} & Разложение \\
        \hline
        $'1' x + '0'$ & $x$ & неприводимый \\
        $'1' x + '1'$ & $x+1$ & неприводимый \\
        $'1' x^2 + '0' x + '0'$ & $x^2$ & $x \cdot x$ \\
        $'1' x^2 + '0'x + '1'$ & $x^2 + 1$ & $(x+1) \cdot (x+1)$ \\
        $'1' x^2 + '1' x + '0'$ & $x^2 + x$ & $x \cdot (x+1)$ \\
        $'1' x^2 + '1' x + '1'$ & $x^2 + x + 1$ & неприводимый \\
        $'1' x^3 + '0' x^2 + '0' x + '1'$ & $x^3 + 1$ & $(x+1) \cdot (x^2+x+1)$ \\
        \hline
    \end{tabular}
\end{table}
\exampleend


\input{aes_math}

\input{modular_ariphmetics}

\input{pseudo-primes}

\section{Группа точек эллиптической кривой над полем}
\selectlanguage{russian}

\subsection{Группы точек на эллиптических кривых}

Эллиптическая кривая $E$ над полем вещественных чисел записывается в виде уравнения, связывающего координаты $x$ и $y$ точек кривой:

\begin{equation}
    E: ~ y^{2} = x^{3} + ax + b,
    \label{Wer}
\end{equation}

где $a,b \in \R$ -- вещественные числа. Эта форма представления эллиптической кривой называется формой Вейерштрасса.

На кривой определен инвариант

\begin{equation}
    J(E)=1728\frac{4a^{3} }{4a^{3} +27b^{2} }
    %\label{Inv}
\end{equation}

Пусть $x_{1} ,x_{2} ,x_{3} $ -- корни уравнения $x^3 + a x + b = 0$. Определим дискриминант $D$ в виде
    \[ D =(x_1 - x_2)^2 (x_1 - x_3)^2 (x_2 - x_3)^2 = - 16(4 a^3 + 27 b^2) \].

Рассмотрим различные значения дискриминанта $D$ и соответствующие им кривые, которые представлены на рисунках~\ref{fig:elliptic-curve-1},~\ref{fig:elliptic-curve-2},~\ref{fig:elliptic-curve-3}.

\begin{enumerate}
    \item При $D>0$ график эллиптической кривой состоит из двух частей (см. рис.~\ref{fig:elliptic-curve-1}). Прямая, проходящая через точки $P(x_1, y_1)$ и $Q(x_2, y_2)$, обязательно пересечет вторую часть кривой в точке с координатами $(x_3, \widetilde{y}_3)$, отображением которой является точка $R(x_3, y_3)$, где $y_3 = - \widetilde{y}_3$. Любые точки на кривой при $D>0$ являются элементами группы по сложению.
        \begin{figure}[!ht]
        	\centering
        	\includegraphics[width=0.5\textwidth]{pic/elliptic-curve-1}
            \caption{Эллиптическая кривая с дискриминантом $D>0$\label{fig:elliptic-curve-1}}
        \end{figure}
    \item Если $D=0$, то левая и правая части касаются в одной точке (см. рис.~\ref{fig:elliptic-curve-2}). Эти кривые называются сингулярными и не рассматриваются.
        \begin{figure}[!ht]
        	\centering
        	\includegraphics[width=0.5\textwidth]{pic/elliptic-curve-2}
            \caption{Эллиптическая кривая с дискриминантом $D = 0$\label{fig:elliptic-curve-2}}
        \end{figure}
    \item Если $D<0$, то записанное выше уравнение~\ref{Wer} описывает одну кривую, представленную на рис.~\ref{fig:elliptic-curve-3}.
        \begin{figure}[!ht]
        	\centering
        	\includegraphics[width=0.5\textwidth]{pic/elliptic-curve-3}
            \caption{Эллиптическая кривая с дискриминантом $D < 0$\label{fig:elliptic-curve-3}}
        \end{figure}
\end{enumerate}

Рассмотрим операцию сложения точек на эллиптической кривой при $D>0$ (другие кривые не рассматриваются).

Пусть точки $P(x_1, y_1)$ и $Q(x_2, y_2)$ принадлежат эллиптической кривой (рис.~\ref{fig:elliptic-curve-1}). Определим операцию сложения точек
    \[ P + Q = R. \]

\begin{enumerate}
    \item Eсли $P \neq Q$, то точка $R$ определяется как отображение (инвертированная $y$-координата) точки, полученной пересечением эллиптической кривой и прямой $PQ$. Совместно решая уравнения кривой и прямой, можно найти координаты точки пересечения. Точка $R = (x_3, y_3)$ равна:
        \[ x_3 = \lambda^2 - x_1 - x_2, \]
        \[ y_3 = - y_1 + \lambda (x_1 - x_3), \]
        где
        \[ \lambda = \frac{y_2 - y_1}{x_2 - x_1} \]
        есть тангенс угла наклона между прямой, проходящей через точки $P$ и $Q$, и осью $x$.

        Теперь рассмотрим специальные случаи.
    \item Пусть точки совпадают: $P = Q$. Прямая $PQ$ превращается в касательную к кривой в точке $P$. Находим пересечение касательной с кривой, инвертируем $y$-координату полученной точки, это будет точка $P + P = R$. Тогда $\lambda$ -- тангенс угла между касательной, проведённой к эллиптической кривой в точке $P$, и осью $x$. Запишем уравнение касательной к эллиптической кривой в точке $(x,y)$ в виде
            \[ 2 y y' = 3 x^2 + a. \]
        Производная равна
            \[ y' = \frac{3 x^2 + a}{2 y} \]
        и
            \[ \lambda = \frac{3 x_1^2 + a}{2 y_1}. \]
        Координаты $R$ имеют прежний вид:
            \[ x_3 = \lambda^2 - x_1 - x_2, \]
            \[ y_3 = - y_1 + \lambda (x_1 - x_3), \]
    \item Пусть $P$ и $Q$ -- противоположные точки, то есть $P=(x,y)$ и $Q=(x, -y)$. Введём еще одну точку на бесконечности и обозначим ее $O$ (точка $O$ или точка 0 <<ноль>>, или альтернативное обозначение $\infty$). Результатом сложения двух противоположных точек определим точку $O$. Точка $Q$ в данном случае обозначается как $-P$:
        \[ P = (x,y), ~ -P = (x, -y), ~ P + (-P) = O. \]
    \item Пусть $P = (x, 0)$ лежит на оси $x$, тогда
        \[ -P = P, ~ P + P = O. \]
\end{enumerate}

Все точки эллиптической кривой, а также точка $O$, образуют коммутативную группу $\E(\R)$ относительно введенной операции сложения, то есть выполняются законы коммутативной группы\index{группа!точек эллиптической кривой}:
\begin{itemize}
    \item сумма точек $P + Q$ лежит на эллиптической кривой;
    \item существует нулевой элемент -- это точка $O$ на бесконечности:
        \[ \forall P \in \E(\R): ~ O + P = P; \]
    \item для любой точки $P$ существует единственный обратный элемент $-P$:
        \[ P + (-P) = O; \]
    \item выполняется ассоциативный закон:
        \[ (P + Q) + F = P + (Q + F) = P + Q + F; \]
    \item выполняется коммутативный закон:
        \[ P + Q = Q + P. \]
\end{itemize}

Сложение точки с самой собой $d$ раз обозначим как умножение точки на число $d$:
    \[ \underbrace{P + P + \ldots + P}_{d \text{ раз}} = d P. \]


\subsection{Эллиптические кривые над конечным полем}

Эллиптические кривые можно строить не только над полем рациональных чисел, но и над другими полями. То есть координатами точек могут выступать не только числа, принадлежащие полю рациональных чисел $\R$, но и элементы поля комплексных чисел $\mathbb{C}$ или конечного поля $\F$. В криптографии нашли своё применение эллиптические кривые именно над конечными полями.

Далее будем рассматривать эллиптические кривые над конечным полем, являющимся кольцом вычетов по модулю нечётного простого\index{число!простое} числа $p$ (дискриминант не равен 0):
    \[ E: ~ y^2 = x^3 + a x + b, \]
    \[ a, b, x, y \in \Z_p, \]
    \[ \Z_p = \{0, 1, 2,  \ldots,  p-1\}.\]

Возможна также более компактная запись:

    \[ E: ~ y^2 = x^3 + a x + b \mod p\]

Точкой эллиптической кривой является пара чисел
    \[ (x,y): x, y \in \Z_p, \]
удовлетворяющая уравнению эллиптической кривой, определённой над конечным полем $\Z_p$.

Операцию сложения двух точек $P = (x_1, y_1)$ и $Q = (x_2, y_2)$ определим точно так же, как и в случае кривой над полем вещественных чисел, описанным выше.

\begin{enumerate}
    \item Две точки $P = (x_1, y_1)$ и $Q = (x_2, y_2)$ эллиптической кривой, определённой над конечным полем $\Z_p$, складываются по правилу:
        \[
            P + Q = R \equiv (x_3, y_3),
        \] \[
            \left\{ \begin{array}{l}
                x_3 = \lambda^2 - x_1 - x_2 \mod p,\\
                y_3 = - y_1 + \lambda (x_1 - x_3) \mod p,\\
            \end{array} \right.
        \]
        где
        \[
            \lambda = \left\{ \begin{array}{l}
                \frac{y_2 - y_1}{x_2 - x_1} \mod p, ~ \text{ если } P \ne Q, \\
                \\
                \frac{3 x_1^2 + a}{2 y_1} \mod p, ~ \text{ если } P = Q. \\
            \end{array} \right.
        \]
    \item Сложение точки $P=(x,y)$ c противоположной точкой \\
        $(-P) = (x,-y)$ даёт точку в бесконечности $O$:
        \[ P + (-P) = O, \]
        \[ (x_1, y_1) + (x_1, -y_1) = O, \]
        \[ (x_1, 0) + (x_1, 0) = O. \]
\end{enumerate}

Мы рассматриваем эллиптические кривые над конечным полем $\Z_p$, где $p > 3$ -- простое\index{число!простое} число, элементы $\Z_p$ -- целые числа $\{0, 1, 2,  \ldots, p-1\}$, т.~е. исследуем следующее уравнение двух переменных $x, y \in \Z_p$:
    \[ y^2 = x^3 + a x + b \mod p, \]
где $a, b \in \Z_p$ -- некоторые константы.

Как и в случае выше, множество точек над конечным полем $\Z_p$, удовлетворяющих уравнению эллиптической кривой, вместе с точкой в бесконечности $O$ образуют конечную группу $\E(\Z_p)$ относительно описанного закона сложения:\index{группа!точек эллиптической кривой}
    \[ \E(\Z_p) ~ \equiv~  O ~ \bigcup ~
        \left\{ (x, y) \in \Z_p \times \Z_p ~\Big|~ y^2 = x^3 + a x + b \mod p \right\}. \]

По теореме Хассе\index{теорема!Хассе} порядок группы точек $|\E(\Z_p)|$ оценивается как
    \[ (\sqrt{p}-1)^2 \leq |\E(\Z_p)| \leq (\sqrt{p}+1)^2, \]
или в другой записи
    \[ \Big| |\E(\Z_p)| - p - 1 \Big| \le 2 \sqrt{p}. \]

\subsection{Примеры группы точек}

\subsubsection{Пример 1}

Пусть эллиптическая кривая задана уравнением
    \[ E: ~ y^2 = x^3 + 1 \mod 7. \]
Найдём все решения этого уравнения, а также количество точек $|\E(\Z_p)|$ на этой эллиптической кривой. Для нахождения решений уравнения составим следующую таблицу:

\begin{center} \begin{tabular}{|c|c|c|c|c|c|c|c|}
    \hline
    $x$ & 0 & 1 & 2 & 3 & 4 & 5 & 6 \\
    \hline
    $y^2$ & 1 & 2 & 2 & 0 & 2 & 0 & 0 \\
    \hline
    $y_1$ & 1 & 3 & 3 & 0 & 3 & 0 & 0 \\
    \hline
    $y_2 = - y_1 \mod p$ & 6 & 4 & 4 &   & 4 &   &   \\
    \hline
\end{tabular} \end{center}

Выпишем все точки, принадлежащие данной эллиптической кривой $\E(\Z_p)$:
\[
    \begin{array}{cccc}
        P_1 = O, & P_2 = (0,1), & P_3 = (0,6), & P_4 = (1,3), \\
        P_5 = (1,4), & P_6 = (2,3), & P_7 = (2,4), & P_8 = (3,0), \\
        P_9 = (4,3), & P_{10} = (4,4), & P_{11} = (5,0), & P_{12} = (6,0). \\
    \end{array}
\]

Получили
    \[ |\E(\Z_p)| = 12. \]

Проверим выполнение неравенства Хассе:
    \[ \left| 12 - 7 - 1 \right| = 4 < 2 \sqrt{7}. \]
Следовательно, неравенство Хассе выполняется.

Минимальное натуральное число $s$ такое, что
\[ \underbrace{P + P + \ldots + P}_{s} \equiv s P = O \]
будем называть \emph{порядком точки $P$}.

%Теорема Лагранжа определяет порядок подгруппы.

\subsubsection{Пример 2}

Группа точек эллиптической кривой
    \[ y^2 = x^3 + 5 x + 6 \mod 17 \]
состоит из точек
\[ \begin{array}{ccccccc}
    \E(\Z_p) & =~ \Big\{ & (-8, \pm 7), & (-7, \pm 6), & (-6, \pm 7), &   & \\
             &           & (-5, \pm 3), & (-3, \pm 7), & (-1, 0),     & O & \Big\}. \\
\end{array} \]

Порядок группы:
    \[ |\E(\Z_p)| = 12. \]

Порядок группы точек по теореме Хассе:
    \[ (\sqrt{p}-1)^2 \leq |\E(\Z_p)| \leq (\sqrt{p}+1)^2, \]
    \[ 10 \leq 12 \leq 26. \]

Порядки возможных подгрупп: 2, 3, 4, 6 (все возможные делители порядка группы 12).

В табл.~\ref{tab:ellipic-point-order-sample} найден порядок точки $P = (-8, 7)$ той же кривой
    \[ y^2 = x^3 + 5 x + 6 \mod 17. \]
Проверяются только степени точки, равные всем делителям порядка группы 12, отличным от 1: 2, 3, 4, 6. Найденный порядок точки $(-8,7)$ равен 12, следовательно, она -- генератор всей группы.

\begin{table}[!ht]
    \centering
    \caption{Пример нахождения порядка точки\label{tab:ellipic-point-order-sample}}
    \resizebox{\textwidth}{!}{ \begin{tabular}{|c|p{0.9\textwidth}|}
        \hline
        2 & $2 P = P + P = 2 \cdot (-8,7) = (-8,7) + (-8,7) = R$, \\
        & $\lambda = \frac{3 x_P^2 + a}{2y_P} = \frac{3 \cdot (-8)^2 + 5}{2 \cdot 7} = 8 \mod 17$, \\
        & $x_R = \lambda^2 - 2x_P = 8^2 - 2 \cdot (-8) = -5 \mod 17$, \\
        & $y_R = \lambda (x_P - x_R) - y_P = 8 \cdot ((-8) - (-5)) - 7 = 3 \mod 17$, \\
        & $R = 2P = (-5, 3)$ \\
        \hline
        3 & $3 P = 2 P + P = Q + P = R$, \\
        & $Q = 2P = (-5, 3)$, \\
        & $\lambda = \frac{y_Q - y_P}{x_Q - x_P} = \frac{3 - 7}{-5 - (-8)} = -7 \mod 17$, \\
        & $x_R = \lambda^2 - x_P - x_Q = (-7)^2 - (-8) - (-5) = -6 \mod 17$, \\
        & $y_R = \lambda (x_P - x_R) - y_P = -7 \cdot (-8 - (-6)) - 7 = 7 \mod 17$, \\
        & $R = 3P = (-6, 7)$ \\
        \hline
        4 & $4 P = 2 \cdot (2 P) = 2 \cdot (-5,3) = (-3, -7)$ \\
        \hline
        6 & $6 P = 2 P +  4 P = (-5,3) + (-3, -7) = (-1, 0)$ \\
        \hline
        12 & $12 P = 2 \cdot (6 P) = 2 \cdot (-1, 0) = O$ \\
        \hline
    \end{tabular} }
\end{table}

В табл.~\ref{tab:elliptic-group-sample} найдены порядки точек и циклические подгруппы группы точек $\E(\Z_p)$ такой же эллиптической кривой
    \[ y^2 = x^3 + 5 x + 6 \mod 17. \]
Группа циклическая, число генераторов:
    \[ \varphi(12) = 4. \]
Циклические подгруппы:
    \[ \Gr^{(2)}, ~ \Gr^{(3)}, ~ \Gr^{(4)}, ~ \Gr^{(6)}, \]
верхний индекс обозначает порядок подгруппы.

\begin{table}[!ht]
    \centering
    \caption{Генераторы и циклические подгруппы группы точек эллиптической кривой\label{tab:elliptic-group-sample}}
    \resizebox{\textwidth}{!}{
    \begin{tabular}{|c|l|c|}
        \hline
        Элемент & Порождаемая группа или подгруппа & Порядок \\
        \hline
        $(-8,  \pm 7) $ & Вся группа $\E(\Z_p)$ & 12, генератор \\
        $(-7, \pm 6) $ & Вся группа $\E(\Z_p)$ & 12, генератор \\
        $(-6, \pm 7) $ & $\Gr^{(4)} ~=~ \big\{ ~ (-6, \pm 7), ~ (-1,0), ~ O ~ \big\}$ & 4 \\
        $(-5, \pm 3) $ & $\Gr^{(6)} ~=~ \big\{ ~ (-5, \pm 3), ~ (-3, \pm 7), ~ (-1,0), ~ O ~ \big\}$ & 6 \\
        $(-3, \pm 7) $ & $\Gr^{(3)} ~=~ \big\{ ~ (-3, \pm 7), ~ O ~ \big\}$ & 3\\
        $(-1, 0)     $ & $\Gr^{(2)} ~=~ \big\{ ~ (-1, 0), ~ O ~ \big\}$ & 2\\
        \hline
    \end{tabular}
    }
\end{table}


\section[Полиномиальные и экспоненциальные алгоритмы]{Полиномиальные и \\ экспоненциальные алгоритмы}

Данный раздел поясняет обоснованность стойкости криптосистем с открытым ключом и имеет лишь косвенное отношение к дискретной математике.

Машина Тьюринга (МТ) (модель, представляющая любой вычислительный алгоритм) состоит из следующих частей:
\begin{itemize}
    \item неограниченной ленты, разделенной на клетки; в каждой клетке содержится символ из конечного алфавита, содержащего пустой символ blank; если символ ранее не был записан на ленту, то он считается blank;
    \item печатающей головки, которая может считать, записать символ $a_i$ и передвинуть ленту на 1 клетку влево-вправо $d_k$;
    \item конечной таблицы действий
    \[ (q_i, a_j) \rightarrow (q_{i1}, a_{j1}, d_k), \]
где $q$ -- состояние машины.
\end{itemize}

Если таблица переходов однозначна, то машина Тьюринга\index{машина Тьюринга} называется детерминированной. \textbf{Детерминированная} машина Тьюринга может \emph{имитировать} любую существующую детерминированную ЭВМ. Если таблица переходов не однозначна, то есть $(q_i, a_j)$ может переходить по нескольким правилам, то машина \textbf{недетерминированная}. \emph{Квантовый компьютер} является примером недетерминированной машины Тьюринга.

Класс задач $\set{P}$ -- задачи, которые могут быть решены за \emph{полиномиальное} время\index{задача!полиномиальная} на \emph{детерминированной} машине Тьюринга. Пример полиномиальной сложности (количество битовых операций)
    \[ O(k^{\textrm{const}}), \]
где $k$ -- длина входных параметров алгоритма. Операция возведения в степень в модульной арифметике $a^b \mod n$ имеет кубическую сложность $O(k^3)$, где $k$ -- двоичная длина чисел $a,b,n$.

Класс задач $\set{NP}$ -- обобщение класса $\set{P} \subseteq \set{NP}$, включает задачи, которые могут быть решены за \emph{полиномиальное} время на \emph{недетерминированной} машине Тьюринга. Пример сложности задач из $\set{NP}$ -- экспоненциальная сложность\index{задача!экспоненциальная}
    \[ O(\textrm{const}^k). \]
Описанный алгоритм Гельфонда (в разделе криптостойкости системы Эль-Гамаля\index{криптосистема!Эль-Гамаля}) решения задачи дискретного логарифма по нахождению $x$ для заданных $g \mod p$ и $a = g^x \mod p$ имеет сложность $O(e^{k/2})$, где $k$ -- двоичная длина чисел.

В криптографии полиномиальные $\set{P}$ алгоритмы считаются \emph{легкими и вычислимыми} на ЭВМ, которые являются детерминированными машинами Тьюринга. Неполиномиальные (экспоненциальные) $\set{NP}$ алгоритмы считаются \emph{трудными и невычислимыми} на ЭВМ, так как из-за экспоненциального роста сложности всегда можно выбрать такой параметр $k$, что время вычисления станет сравнимым с возрастом Вселенной.

Задачи факторизации числа и дискретного логарифмирования в группе считаются $\set{NP}$-задачами.

Класс $\set{NP}$-полных задач -- подмножество задач из $\set{NP}$, для которых не известен полиномиальный алгоритм для детерминированной машины Тьюринга, и все задачи могут быть сведены друг к другу за полиномиальное время на \emph{детерминированной} машине Тьюринга. Например, задача об укладке рюкзака является $\set{NP}$-полной.

Стойкость криптосистем с \emph{открытым} ключом, как правило, основана на $\set{NP}$ или $\set{NP}$-полных задачах:
\begin{enumerate}
    \item RSA\index{криптосистема!RSA} -- $\set{NP}$-задача факторизации (строго говоря, на трудности извлечения корня степени $e$ по модулю $n$).
    \item Криптосистемы типа Эль-Гамаля\index{криптосистема!Эль-Гамаля} -- $\set{NP}$-задача дискретного логарифмирования.
\end{enumerate}

\emph{Нерешенной} проблемой является доказательство неравенства
    \[ \set{P} \neq \set{NP}. \]
Именно на гипотезе о том, что для некоторых задач не существует полиномиальных алгоритмов, и основана стойкость криптосистем с открытым ключом.

\section{Метод индекса совпадений}
\selectlanguage{russian}
\label{chap:coincide-index}

Приведём теоретическое обоснование метода индекса совпадений. Пусть алфавит имеет размер $A$. Перенумеруем его буквы числами от $1$ до $A$. Пусть заданы вероятности появления каждой буквы
    \[ \mathcal{P} = \left\{ {p_1 ,p_2 ,  \ldots , p_A } \right\}. \]
В простейшей модели языка предполагается, что тексты состоят из последовательности букв, порождаемых источником независимо друг от друга с известным распределением $\mathcal{P}$.

Найдём индекс совпадений для различных предположений относительно распределений букв последовательности. Сначала рассмотрим случай, когда вероятности всех букв одинаковы. Пусть
    \[ \mathbf{X} = \left[ X_1, X_2, \dots, X_L \right] \]
случайный текст с распределением
    \[ \mathcal{P}_1 = \left\{ p_{11}, p_{12}, \dots, p_{1A} \right\}. \]
Найдём индекс совпадений
    \[ I_c(\mathcal{P}_1), \]
то есть вероятность того, что в случайно выбранной паре позиций находятся одинаковые буквы.

Для пары позиций $(k,j)$ найдём условную вероятность $P \left( X_k  = X_j \mid (k,j) \right)$:
    \[ P \left( X_k  = X_j \mid (k,j) \right) ~=~ \sum\limits_{i=1}^A p_{1i}^2 ~\equiv~ k_{p_1}. \]
Эта вероятность не зависит от выбора пары позиций $(k,j)$.

Так как число различных пар равно $\frac{L(L - 1)}{2}$, то вероятность случайного выбора пары $(k,j)$  равна
    \[ P_{(K,J)} (k,j) = \frac{2}{L(L - 1)}. \]
Следовательно,
\[
    I(\mathcal{P}_1) ~= \sum \limits_{1 \leq k < j \leq L} P_{(K,J)}(k,j) ~\cdot~ P(X_k  = X_j \mid (k,j)) =
\] \[
    = \sum \limits_{1 \leq k < j \leq L} \frac{2}{L(L - 1)} k_{p_1} = k_{p_1}.
\]

Найдём теперь аналогичную вероятность $I\left( {\mathcal{P}_1 ,\mathcal{P}_2 } \right)$  для случая, когда последовательность независимых случайных букв может быть представлена в виде
\[
\mathbf{X} = \left[ {\begin{array}{*{20}c}
   {X_1 ,}  \\
   {Y_1 ,}  \\
 \end{array} \begin{array}{*{20}c}
   {X_2 ,}  \\
   {Y_2 ,}  \\
 \end{array} \begin{array}{*{20}c}
   { \ldots ,}  \\
   { \ldots ,}  \\
 \end{array} \begin{array}{*{20}c}
   {X_{L/2} }  \\
   {Y_{L/2} }  \\
 \end{array} } \right],
\]
где одинаково распределенные случайные буквы в первой строке имеют распределение
    \[ \mathcal{P}_1  = \left\{ {p_{11} ,p_{12} ,  \ldots , p_{1A} } \right\}, \]
а одинаково распределенные случайные буквы во второй строке имеют другое распределение
    \[ \mathcal{P}_2  = \left\{ {p_{21} ,p_{22} ,  \ldots , p_{2A} } \right\}. \]
В этом случае сумму по всем парам мы разделяем на три суммы: по парам внутри позиций первой строки; по парам внутри позиций второй строки и по парам, в которых первая позиция берётся из первой строки, а вторая -- из второй:
{ \small
\[
    I(\mathcal{P}_1, \mathcal{P}_2) =
        \frac{2}{L(L - 1)} \cdot \left(
        \sum \limits_{1 \leq k < j \leq L/2} P( X_k  = X_j \mid ( k,j )) ~~ + \right.
\] \[
        \left. + \sum\limits_{1 \leq k < j \leq L/2} P(Y_k  = Y_j \mid (k,j)) ~~+~~
            \sum\limits_{k=1}^{L/2} \sum\limits_{j=1}^{L/2} {P(X_k = Y_j \mid (k,j))} \right) =
\] \[
    = \frac{2}{L(L - 1)} \left( \frac{1}{2} \frac{L}{2} \left( \frac{L}{2} - 1 \right) k_{p_1} +
        \frac{1}{2} \frac{L}{2} \left( \frac{L}{2} - 1 \right) k_{p_2} +
        \left( \frac{L}{2} \right)^2 \sum \limits_{i = 1}^A p_{1,i} p_{2,i} \right) =
\] \[
    = \frac{2}{L(L - 1)} \left( \frac{1}{2} \frac{L}{2} \left( \frac{L}{2} - 1 \right) k_{p_1} +
        \frac{1}{2} \frac{L}{2} \left( \frac{L}{2} - 1 \right) k_{p_2} +
        \left( \frac{L}{2} \right)^2 k_{p_1, p_2} \right),
\] }
где обозначено
    \[ k_{p_1, p_2}  = \sum\limits_{i=1}^A p_{1,i} p_{2,i}. \]


В общем случае рассмотрим последовательность, представленную в виде матрицы, состоящей из $m$  строк и $\frac{L}{m}$ столбцов, где
\[
{\mathbf X} = \left[ {\begin{array}{*{20}c}
   {X_1 } & {X_2 } &  \ldots  & {X_{L/m} }  \\
   {Y_1 } & {Y_2 } &  \ldots  & {Y_{L/m} }  \\
    \vdots  &  \vdots  &  \vdots  &  \vdots   \\
   {Z_1 } & {Z_2 } &  \ldots  & {Z_{L/m} }  \\
\end{array}} \right].
\]
Считаем, что одинаково распределенные случайные буквы в первой строке имеют распределение
    \[ P_1  = \left\{ {p_{11} ,p_{12} ,  \ldots , p_{1A} } \right\}, \]
одинаково распределенные случайные буквы во второй строке имеют распределение
    \[ P_2  = \left\{ {p_{21} ,p_{22} ,  \ldots , p_{2A} } \right\} \]
и т.~д., одинаково распределенные случайные буквы $m$-й строки имеют распределение
    \[ P_m  = \left\{ {p_{m1},p_{m2} ,  \ldots , p_{mA} } \right\}. \]

Для вычисления вероятности того, что в случайно выбранной паре позиций будут одинаковые буквы, выполним суммирование по различным парам внутри строк и по парам между различными строками. Аналогично предыдущему случаю получим
{ \small
\[
    I(\mathcal{P}_1, \mathcal{P}_2, \ldots, \mathcal{P}_m ) ~=
\] \[
    =~ \frac{2}{L(L - 1)} \left( \frac{1}{2} \frac{L}{m} \left( \frac{L}{m} - 1 \right) k_{p_1} ~+~
        \frac{1}{2} \frac{L}{m} \left( \frac{L}{m} - 1 \right) k_{p_2} ~+ \right.
\] \[
        +~ \dots ~+~ \left. \frac{1}{2} \frac{L}{m} \left( \frac{L}{m} - 1 \right) k_{p_m} \right) ~+
\] \[
       +~ \frac{2}{L(L - 1)} \left( \left( \frac{L}{m} \right)^2 k_{p_1, p_2} +
         \left( \frac{L}{m} \right)^2 k_{p_1, p_3} + \dots +
        \left( \frac{L}{m} \right)^2 k_{p_{m - 1}, p_m } \right).
\] }
Первая сумма содержит  $m$ слагаемых, вторая -- $ \frac{m(m-1)}{2}$ слагаемых. Полагая
    \[ k_{p_1} = k_{p_2} = \dots = k_{p_m} = k_p, \]
    \[ k_{p_i p_j } = k_r = \frac{1}{A}, ~ i \ne j, \]
получим после несложных выкладок
    \[ m = \frac{k_p  - k_r}{I - k_r  + \frac{k_p  - I}{L}}. \]


%\chapter{Задачи и упражнения}
%
%К \textbf{примерам 1, 2, 3} \textbf{упражнение 1}. Указать способ расшифрования для легального получателя шифротекста и указать способ дешифрования для криптоаналитика, не знающего ключа.
%
%\textbf{Упражнение 2}. Пусть $M_1, M_2, M_3,\ldots,M_s$ -- набор перестановок. Показать, что существует единственная перестановка $M=M_1,M_2,M_3,\ldots M_s$.
%
%\textbf{Упражнение 3}. Вскрыть одиночную ячейку Фейстеля. Для этого задать конкретную функцию $F(K,R)$ и по конкретным значениям $L_{1}$ и $R_{1}$ найти $K$.
%
%\textbf{Упражнение 4}. Разделим последовательность на блоки, каждый из которых содержит 2 бита.
%
%\[\begin{array}{cc} {z_{1} } & {z_{2} } \end{array}|\begin{array}{cc} {z_{3} } & {z_{4} } \end{array}| \ldots |\begin{array}{cc} {z_{N} } & {z_{N+1} } \end{array}\]
%Блок может принимать значения $z_{1} z_{2} =\begin{array}{c} {11} \\ {10} \\ {01} \\ {00} \end{array}$
%Преобразуем последовательность символов:
% если $z_{1} z_{2} =11$ или $z_{1} z_{2} =00$, то пара выбрасывается;
%если $[z_{1} z_{2} =10$, то записываем новый символ $u=1$; если
%$z_{1} z_{2} =01$, то записываем новый символ $u=0$.
%Получаем новую двоичную последовательность.
%
%Показать, что вероятностное распределение символов в новой последовательности является равномерным.
%
%\textbf{Упражнение 5}.Предположим, что криптоаналитик знает, что период генерируемой $M$ -последовательности равен $T=2^{L} -1$. Пусть ему известна часть последовательности длины, меньшей периода: $T_{1}<2^{L} -1$.
% При каком значении $T_{1}$  криптоаналитик может найти многочлен обратной связи.
%
%\textbf{Упражнение 6}. Ответить на вопрос: <<Как подделать ЭП, не зная закрытого ключа?>>
%
%\textbf{Упражнение 7}. При помощи формул Виета найти дискриминант многочлена, представляющего эллиптическую кривую.

\input{tasks}

\printindex

\chapter*{Литература}
\addcontentsline{toc}{chapter}{Литература}
\begingroup
\renewcommand{\chapter}[2]{}%
%\bibliographystyle{ugost2008s}
%\bibliography{bibliography}
\printbibliography
\endgroup

\end{document}
